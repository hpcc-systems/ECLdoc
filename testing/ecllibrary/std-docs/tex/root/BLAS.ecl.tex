\chapter*{\color{headfile}
BLAS
}
\hypertarget{ecldoc:toc:BLAS}{}
\hyperlink{ecldoc:toc:root}{Go Up}

\section*{\underline{\textsf{IMPORTS}}}
\begin{doublespace}
{\large
lib\_eclblas |
}
\end{doublespace}

\section*{\underline{\textsf{DESCRIPTIONS}}}
\subsection*{\textsf{\colorbox{headtoc}{\color{white} MODULE}
BLAS}}

\hypertarget{ecldoc:blas}{}

{\renewcommand{\arraystretch}{1.5}
\begin{tabularx}{\textwidth}{|>{\raggedright\arraybackslash}l|X|}
\hline
\hspace{0pt}\mytexttt{\color{red} } & \textbf{BLAS} \\
\hline
\end{tabularx}
}

\par





No Documentation Found







\textbf{Children}
\begin{enumerate}
\item \hyperlink{ecldoc:BLAS.Types}{Types}
: No Documentation Found
\item \hyperlink{ecldoc:blas.icellfunc}{ICellFunc}
: Function prototype for Apply2Cell
\item \hyperlink{ecldoc:blas.apply2cells}{Apply2Cells}
: Iterate matrix and apply function to each cell
\item \hyperlink{ecldoc:blas.dasum}{dasum}
: Absolute sum, the 1 norm of a vector
\item \hyperlink{ecldoc:blas.daxpy}{daxpy}
: alpha*X + Y
\item \hyperlink{ecldoc:blas.dgemm}{dgemm}
: alpha*op(A) op(B) + beta*C where op() is transpose
\item \hyperlink{ecldoc:blas.dgetf2}{dgetf2}
: Compute LU Factorization of matrix A
\item \hyperlink{ecldoc:blas.dpotf2}{dpotf2}
: DPOTF2 computes the Cholesky factorization of a real symmetric positive definite matrix A
\item \hyperlink{ecldoc:blas.dscal}{dscal}
: Scale a vector alpha
\item \hyperlink{ecldoc:blas.dsyrk}{dsyrk}
: Implements symmetric rank update C 
\item \hyperlink{ecldoc:blas.dtrsm}{dtrsm}
: Triangular matrix solver
\item \hyperlink{ecldoc:blas.extract_diag}{extract\_diag}
: Extract the diagonal of he matrix
\item \hyperlink{ecldoc:blas.extract_tri}{extract\_tri}
: Extract the upper or lower triangle
\item \hyperlink{ecldoc:blas.make_diag}{make\_diag}
: Generate a diagonal matrix
\item \hyperlink{ecldoc:blas.make_vector}{make\_vector}
: Make a vector of dimension m
\item \hyperlink{ecldoc:blas.trace}{trace}
: The trace of the input matrix
\end{enumerate}

\rule{\linewidth}{0.5pt}

\subsection*{\textsf{\colorbox{headtoc}{\color{white} MODULE}
Types}}

\hypertarget{ecldoc:BLAS.Types}{}
\hspace{0pt} \hyperlink{ecldoc:blas}{BLAS} \textbackslash 

{\renewcommand{\arraystretch}{1.5}
\begin{tabularx}{\textwidth}{|>{\raggedright\arraybackslash}l|X|}
\hline
\hspace{0pt}\mytexttt{\color{red} } & \textbf{Types} \\
\hline
\end{tabularx}
}

\par





No Documentation Found







\textbf{Children}
\begin{enumerate}
\item \hyperlink{ecldoc:blas.types.value_t}{value\_t}
: No Documentation Found
\item \hyperlink{ecldoc:blas.types.dimension_t}{dimension\_t}
: No Documentation Found
\item \hyperlink{ecldoc:blas.types.matrix_t}{matrix\_t}
: No Documentation Found
\item \hyperlink{ecldoc:ecldoc-Triangle}{Triangle}
: No Documentation Found
\item \hyperlink{ecldoc:ecldoc-Diagonal}{Diagonal}
: No Documentation Found
\item \hyperlink{ecldoc:ecldoc-Side}{Side}
: No Documentation Found
\end{enumerate}

\rule{\linewidth}{0.5pt}

\subsection*{\textsf{\colorbox{headtoc}{\color{white} ATTRIBUTE}
value\_t}}

\hypertarget{ecldoc:blas.types.value_t}{}
\hspace{0pt} \hyperlink{ecldoc:blas}{BLAS} \textbackslash 
\hspace{0pt} \hyperlink{ecldoc:BLAS.Types}{Types} \textbackslash 

{\renewcommand{\arraystretch}{1.5}
\begin{tabularx}{\textwidth}{|>{\raggedright\arraybackslash}l|X|}
\hline
\hspace{0pt}\mytexttt{\color{red} } & \textbf{value\_t} \\
\hline
\end{tabularx}
}

\par





No Documentation Found








\par
\begin{description}
\item [\colorbox{tagtype}{\color{white} \textbf{\textsf{RETURN}}}] \textbf{REAL8} --- 
\end{description}




\rule{\linewidth}{0.5pt}
\subsection*{\textsf{\colorbox{headtoc}{\color{white} ATTRIBUTE}
dimension\_t}}

\hypertarget{ecldoc:blas.types.dimension_t}{}
\hspace{0pt} \hyperlink{ecldoc:blas}{BLAS} \textbackslash 
\hspace{0pt} \hyperlink{ecldoc:BLAS.Types}{Types} \textbackslash 

{\renewcommand{\arraystretch}{1.5}
\begin{tabularx}{\textwidth}{|>{\raggedright\arraybackslash}l|X|}
\hline
\hspace{0pt}\mytexttt{\color{red} } & \textbf{dimension\_t} \\
\hline
\end{tabularx}
}

\par





No Documentation Found








\par
\begin{description}
\item [\colorbox{tagtype}{\color{white} \textbf{\textsf{RETURN}}}] \textbf{UNSIGNED4} --- 
\end{description}




\rule{\linewidth}{0.5pt}
\subsection*{\textsf{\colorbox{headtoc}{\color{white} ATTRIBUTE}
matrix\_t}}

\hypertarget{ecldoc:blas.types.matrix_t}{}
\hspace{0pt} \hyperlink{ecldoc:blas}{BLAS} \textbackslash 
\hspace{0pt} \hyperlink{ecldoc:BLAS.Types}{Types} \textbackslash 

{\renewcommand{\arraystretch}{1.5}
\begin{tabularx}{\textwidth}{|>{\raggedright\arraybackslash}l|X|}
\hline
\hspace{0pt}\mytexttt{\color{red} } & \textbf{matrix\_t} \\
\hline
\end{tabularx}
}

\par





No Documentation Found








\par
\begin{description}
\item [\colorbox{tagtype}{\color{white} \textbf{\textsf{RETURN}}}] \textbf{SET ( REAL8 )} --- 
\end{description}




\rule{\linewidth}{0.5pt}
\subsection*{\textsf{\colorbox{headtoc}{\color{white} ATTRIBUTE}
Triangle}}

\hypertarget{ecldoc:ecldoc-Triangle}{}
\hspace{0pt} \hyperlink{ecldoc:blas}{BLAS} \textbackslash 
\hspace{0pt} \hyperlink{ecldoc:BLAS.Types}{Types} \textbackslash 

{\renewcommand{\arraystretch}{1.5}
\begin{tabularx}{\textwidth}{|>{\raggedright\arraybackslash}l|X|}
\hline
\hspace{0pt}\mytexttt{\color{red} } & \textbf{Triangle} \\
\hline
\end{tabularx}
}

\par





No Documentation Found








\par
\begin{description}
\item [\colorbox{tagtype}{\color{white} \textbf{\textsf{RETURN}}}] \textbf{UNSIGNED1} --- 
\end{description}




\rule{\linewidth}{0.5pt}
\subsection*{\textsf{\colorbox{headtoc}{\color{white} ATTRIBUTE}
Diagonal}}

\hypertarget{ecldoc:ecldoc-Diagonal}{}
\hspace{0pt} \hyperlink{ecldoc:blas}{BLAS} \textbackslash 
\hspace{0pt} \hyperlink{ecldoc:BLAS.Types}{Types} \textbackslash 

{\renewcommand{\arraystretch}{1.5}
\begin{tabularx}{\textwidth}{|>{\raggedright\arraybackslash}l|X|}
\hline
\hspace{0pt}\mytexttt{\color{red} } & \textbf{Diagonal} \\
\hline
\end{tabularx}
}

\par





No Documentation Found








\par
\begin{description}
\item [\colorbox{tagtype}{\color{white} \textbf{\textsf{RETURN}}}] \textbf{UNSIGNED1} --- 
\end{description}




\rule{\linewidth}{0.5pt}
\subsection*{\textsf{\colorbox{headtoc}{\color{white} ATTRIBUTE}
Side}}

\hypertarget{ecldoc:ecldoc-Side}{}
\hspace{0pt} \hyperlink{ecldoc:blas}{BLAS} \textbackslash 
\hspace{0pt} \hyperlink{ecldoc:BLAS.Types}{Types} \textbackslash 

{\renewcommand{\arraystretch}{1.5}
\begin{tabularx}{\textwidth}{|>{\raggedright\arraybackslash}l|X|}
\hline
\hspace{0pt}\mytexttt{\color{red} } & \textbf{Side} \\
\hline
\end{tabularx}
}

\par





No Documentation Found








\par
\begin{description}
\item [\colorbox{tagtype}{\color{white} \textbf{\textsf{RETURN}}}] \textbf{UNSIGNED1} --- 
\end{description}




\rule{\linewidth}{0.5pt}


\subsection*{\textsf{\colorbox{headtoc}{\color{white} FUNCTION}
ICellFunc}}

\hypertarget{ecldoc:blas.icellfunc}{}
\hspace{0pt} \hyperlink{ecldoc:blas}{BLAS} \textbackslash 

{\renewcommand{\arraystretch}{1.5}
\begin{tabularx}{\textwidth}{|>{\raggedright\arraybackslash}l|X|}
\hline
\hspace{0pt}\mytexttt{\color{red} Types.value\_t} & \textbf{ICellFunc} \\
\hline
\multicolumn{2}{|>{\raggedright\arraybackslash}X|}{\hspace{0pt}\mytexttt{\color{param} (Types.value\_t v, Types.dimension\_t r, Types.dimension\_t c)}} \\
\hline
\end{tabularx}
}

\par





Function prototype for Apply2Cell.






\par
\begin{description}
\item [\colorbox{tagtype}{\color{white} \textbf{\textsf{PARAMETER}}}] \textbf{\underline{r}} ||| UNSIGNED4 --- the row ordinal
\item [\colorbox{tagtype}{\color{white} \textbf{\textsf{PARAMETER}}}] \textbf{\underline{v}} ||| REAL8 --- the value
\item [\colorbox{tagtype}{\color{white} \textbf{\textsf{PARAMETER}}}] \textbf{\underline{c}} ||| UNSIGNED4 --- the column ordinal
\end{description}







\par
\begin{description}
\item [\colorbox{tagtype}{\color{white} \textbf{\textsf{RETURN}}}] \textbf{REAL8} --- the updated value
\end{description}




\rule{\linewidth}{0.5pt}
\subsection*{\textsf{\colorbox{headtoc}{\color{white} FUNCTION}
Apply2Cells}}

\hypertarget{ecldoc:blas.apply2cells}{}
\hspace{0pt} \hyperlink{ecldoc:blas}{BLAS} \textbackslash 

{\renewcommand{\arraystretch}{1.5}
\begin{tabularx}{\textwidth}{|>{\raggedright\arraybackslash}l|X|}
\hline
\hspace{0pt}\mytexttt{\color{red} Types.matrix\_t} & \textbf{Apply2Cells} \\
\hline
\multicolumn{2}{|>{\raggedright\arraybackslash}X|}{\hspace{0pt}\mytexttt{\color{param} (Types.dimension\_t m, Types.dimension\_t n, Types.matrix\_t x, ICellFunc f)}} \\
\hline
\end{tabularx}
}

\par





Iterate matrix and apply function to each cell






\par
\begin{description}
\item [\colorbox{tagtype}{\color{white} \textbf{\textsf{PARAMETER}}}] \textbf{\underline{f}} ||| FUNCTION [ REAL8 , UNSIGNED4 , UNSIGNED4 ] ( REAL8 ) --- function to apply
\item [\colorbox{tagtype}{\color{white} \textbf{\textsf{PARAMETER}}}] \textbf{\underline{m}} ||| UNSIGNED4 --- number of rows
\item [\colorbox{tagtype}{\color{white} \textbf{\textsf{PARAMETER}}}] \textbf{\underline{x}} ||| SET ( REAL8 ) --- matrix
\item [\colorbox{tagtype}{\color{white} \textbf{\textsf{PARAMETER}}}] \textbf{\underline{n}} ||| UNSIGNED4 --- number of columns
\end{description}







\par
\begin{description}
\item [\colorbox{tagtype}{\color{white} \textbf{\textsf{RETURN}}}] \textbf{SET ( REAL8 )} --- updated matrix
\end{description}




\rule{\linewidth}{0.5pt}
\subsection*{\textsf{\colorbox{headtoc}{\color{white} FUNCTION}
dasum}}

\hypertarget{ecldoc:blas.dasum}{}
\hspace{0pt} \hyperlink{ecldoc:blas}{BLAS} \textbackslash 

{\renewcommand{\arraystretch}{1.5}
\begin{tabularx}{\textwidth}{|>{\raggedright\arraybackslash}l|X|}
\hline
\hspace{0pt}\mytexttt{\color{red} Types.value\_t} & \textbf{dasum} \\
\hline
\multicolumn{2}{|>{\raggedright\arraybackslash}X|}{\hspace{0pt}\mytexttt{\color{param} (Types.dimension\_t m, Types.matrix\_t x, Types.dimension\_t incx, Types.dimension\_t skipped=0)}} \\
\hline
\end{tabularx}
}

\par





Absolute sum, the 1 norm of a vector.






\par
\begin{description}
\item [\colorbox{tagtype}{\color{white} \textbf{\textsf{PARAMETER}}}] \textbf{\underline{m}} ||| UNSIGNED4 --- the number of entries
\item [\colorbox{tagtype}{\color{white} \textbf{\textsf{PARAMETER}}}] \textbf{\underline{x}} ||| SET ( REAL8 ) --- the column major matrix holding the vector
\item [\colorbox{tagtype}{\color{white} \textbf{\textsf{PARAMETER}}}] \textbf{\underline{incx}} ||| UNSIGNED4 --- the increment for x, 1 in the case of an actual vector
\item [\colorbox{tagtype}{\color{white} \textbf{\textsf{PARAMETER}}}] \textbf{\underline{skipped}} ||| UNSIGNED4 --- default is zero, the number of entries stepped over to get to the first entry
\end{description}







\par
\begin{description}
\item [\colorbox{tagtype}{\color{white} \textbf{\textsf{RETURN}}}] \textbf{REAL8} --- the sum of the absolute values
\end{description}




\rule{\linewidth}{0.5pt}
\subsection*{\textsf{\colorbox{headtoc}{\color{white} FUNCTION}
daxpy}}

\hypertarget{ecldoc:blas.daxpy}{}
\hspace{0pt} \hyperlink{ecldoc:blas}{BLAS} \textbackslash 

{\renewcommand{\arraystretch}{1.5}
\begin{tabularx}{\textwidth}{|>{\raggedright\arraybackslash}l|X|}
\hline
\hspace{0pt}\mytexttt{\color{red} Types.matrix\_t} & \textbf{daxpy} \\
\hline
\multicolumn{2}{|>{\raggedright\arraybackslash}X|}{\hspace{0pt}\mytexttt{\color{param} (Types.dimension\_t N, Types.value\_t alpha, Types.matrix\_t X, Types.dimension\_t incX, Types.matrix\_t Y, Types.dimension\_t incY, Types.dimension\_t x\_skipped=0, Types.dimension\_t y\_skipped=0)}} \\
\hline
\end{tabularx}
}

\par





alpha*X + Y






\par
\begin{description}
\item [\colorbox{tagtype}{\color{white} \textbf{\textsf{PARAMETER}}}] \textbf{\underline{y\_skipped}} ||| UNSIGNED4 --- number of entries skipped to get to the first Y
\item [\colorbox{tagtype}{\color{white} \textbf{\textsf{PARAMETER}}}] \textbf{\underline{Y}} ||| SET ( REAL8 ) --- the column major matrix holding the vector Y
\item [\colorbox{tagtype}{\color{white} \textbf{\textsf{PARAMETER}}}] \textbf{\underline{N}} ||| UNSIGNED4 --- number of elements in vector
\item [\colorbox{tagtype}{\color{white} \textbf{\textsf{PARAMETER}}}] \textbf{\underline{x\_skipped}} ||| UNSIGNED4 --- number of entries skipped to get to the first X
\item [\colorbox{tagtype}{\color{white} \textbf{\textsf{PARAMETER}}}] \textbf{\underline{X}} ||| SET ( REAL8 ) --- the column major matrix holding the vector X
\item [\colorbox{tagtype}{\color{white} \textbf{\textsf{PARAMETER}}}] \textbf{\underline{alpha}} ||| REAL8 --- the scalar multiplier
\item [\colorbox{tagtype}{\color{white} \textbf{\textsf{PARAMETER}}}] \textbf{\underline{incX}} ||| UNSIGNED4 --- the increment or stride for the vector
\item [\colorbox{tagtype}{\color{white} \textbf{\textsf{PARAMETER}}}] \textbf{\underline{incY}} ||| UNSIGNED4 --- the increment or stride of Y
\end{description}







\par
\begin{description}
\item [\colorbox{tagtype}{\color{white} \textbf{\textsf{RETURN}}}] \textbf{SET ( REAL8 )} --- the updated matrix
\end{description}




\rule{\linewidth}{0.5pt}
\subsection*{\textsf{\colorbox{headtoc}{\color{white} FUNCTION}
dgemm}}

\hypertarget{ecldoc:blas.dgemm}{}
\hspace{0pt} \hyperlink{ecldoc:blas}{BLAS} \textbackslash 

{\renewcommand{\arraystretch}{1.5}
\begin{tabularx}{\textwidth}{|>{\raggedright\arraybackslash}l|X|}
\hline
\hspace{0pt}\mytexttt{\color{red} Types.matrix\_t} & \textbf{dgemm} \\
\hline
\multicolumn{2}{|>{\raggedright\arraybackslash}X|}{\hspace{0pt}\mytexttt{\color{param} (BOOLEAN transposeA, BOOLEAN transposeB, Types.dimension\_t M, Types.dimension\_t N, Types.dimension\_t K, Types.value\_t alpha, Types.matrix\_t A, Types.matrix\_t B, Types.value\_t beta=0.0, Types.matrix\_t C=[])}} \\
\hline
\end{tabularx}
}

\par





alpha*op(A) op(B) + beta*C where op() is transpose






\par
\begin{description}
\item [\colorbox{tagtype}{\color{white} \textbf{\textsf{PARAMETER}}}] \textbf{\underline{beta}} ||| REAL8 --- scalar for matrix C
\item [\colorbox{tagtype}{\color{white} \textbf{\textsf{PARAMETER}}}] \textbf{\underline{transposeA}} ||| BOOLEAN --- true when transpose of A is used
\item [\colorbox{tagtype}{\color{white} \textbf{\textsf{PARAMETER}}}] \textbf{\underline{N}} ||| UNSIGNED4 --- number of columns in product
\item [\colorbox{tagtype}{\color{white} \textbf{\textsf{PARAMETER}}}] \textbf{\underline{K}} ||| UNSIGNED4 --- number of columns/rows for the multiplier/multiplicand
\item [\colorbox{tagtype}{\color{white} \textbf{\textsf{PARAMETER}}}] \textbf{\underline{B}} ||| SET ( REAL8 ) --- matrix B
\item [\colorbox{tagtype}{\color{white} \textbf{\textsf{PARAMETER}}}] \textbf{\underline{A}} ||| SET ( REAL8 ) --- matrix A
\item [\colorbox{tagtype}{\color{white} \textbf{\textsf{PARAMETER}}}] \textbf{\underline{transposeB}} ||| BOOLEAN --- true when transpose of B is used
\item [\colorbox{tagtype}{\color{white} \textbf{\textsf{PARAMETER}}}] \textbf{\underline{C}} ||| SET ( REAL8 ) --- matrix C or empty
\item [\colorbox{tagtype}{\color{white} \textbf{\textsf{PARAMETER}}}] \textbf{\underline{M}} ||| UNSIGNED4 --- number of rows in product
\item [\colorbox{tagtype}{\color{white} \textbf{\textsf{PARAMETER}}}] \textbf{\underline{alpha}} ||| REAL8 --- scalar used on A
\end{description}







\par
\begin{description}
\item [\colorbox{tagtype}{\color{white} \textbf{\textsf{RETURN}}}] \textbf{SET ( REAL8 )} --- 
\end{description}




\rule{\linewidth}{0.5pt}
\subsection*{\textsf{\colorbox{headtoc}{\color{white} FUNCTION}
dgetf2}}

\hypertarget{ecldoc:blas.dgetf2}{}
\hspace{0pt} \hyperlink{ecldoc:blas}{BLAS} \textbackslash 

{\renewcommand{\arraystretch}{1.5}
\begin{tabularx}{\textwidth}{|>{\raggedright\arraybackslash}l|X|}
\hline
\hspace{0pt}\mytexttt{\color{red} Types.matrix\_t} & \textbf{dgetf2} \\
\hline
\multicolumn{2}{|>{\raggedright\arraybackslash}X|}{\hspace{0pt}\mytexttt{\color{param} (Types.dimension\_t m, Types.dimension\_t n, Types.matrix\_t a)}} \\
\hline
\end{tabularx}
}

\par





Compute LU Factorization of matrix A.






\par
\begin{description}
\item [\colorbox{tagtype}{\color{white} \textbf{\textsf{PARAMETER}}}] \textbf{\underline{m}} ||| UNSIGNED4 --- number of rows of A
\item [\colorbox{tagtype}{\color{white} \textbf{\textsf{PARAMETER}}}] \textbf{\underline{n}} ||| UNSIGNED4 --- number of columns of A
\item [\colorbox{tagtype}{\color{white} \textbf{\textsf{PARAMETER}}}] \textbf{\underline{a}} ||| SET ( REAL8 ) --- No Doc
\end{description}







\par
\begin{description}
\item [\colorbox{tagtype}{\color{white} \textbf{\textsf{RETURN}}}] \textbf{SET ( REAL8 )} --- composite matrix of factors, lower triangle has an implied diagonal of ones. Upper triangle has the diagonal of the composite.
\end{description}




\rule{\linewidth}{0.5pt}
\subsection*{\textsf{\colorbox{headtoc}{\color{white} FUNCTION}
dpotf2}}

\hypertarget{ecldoc:blas.dpotf2}{}
\hspace{0pt} \hyperlink{ecldoc:blas}{BLAS} \textbackslash 

{\renewcommand{\arraystretch}{1.5}
\begin{tabularx}{\textwidth}{|>{\raggedright\arraybackslash}l|X|}
\hline
\hspace{0pt}\mytexttt{\color{red} Types.matrix\_t} & \textbf{dpotf2} \\
\hline
\multicolumn{2}{|>{\raggedright\arraybackslash}X|}{\hspace{0pt}\mytexttt{\color{param} (Types.Triangle tri, Types.dimension\_t r, Types.matrix\_t A, BOOLEAN clear=TRUE)}} \\
\hline
\end{tabularx}
}

\par





DPOTF2 computes the Cholesky factorization of a real symmetric positive definite matrix A. The factorization has the form A = U**T * U , if UPLO = 'U', or A = L * L**T, if UPLO = 'L', where U is an upper triangular matrix and L is lower triangular. This is the unblocked version of the algorithm, calling Level 2 BLAS.






\par
\begin{description}
\item [\colorbox{tagtype}{\color{white} \textbf{\textsf{PARAMETER}}}] \textbf{\underline{tri}} ||| UNSIGNED1 --- indicate whether upper or lower triangle is used
\item [\colorbox{tagtype}{\color{white} \textbf{\textsf{PARAMETER}}}] \textbf{\underline{r}} ||| UNSIGNED4 --- number of rows/columns in the square matrix
\item [\colorbox{tagtype}{\color{white} \textbf{\textsf{PARAMETER}}}] \textbf{\underline{clear}} ||| BOOLEAN --- clears the unused triangle
\item [\colorbox{tagtype}{\color{white} \textbf{\textsf{PARAMETER}}}] \textbf{\underline{A}} ||| SET ( REAL8 ) --- the square matrix
\end{description}







\par
\begin{description}
\item [\colorbox{tagtype}{\color{white} \textbf{\textsf{RETURN}}}] \textbf{SET ( REAL8 )} --- the triangular matrix requested.
\end{description}




\rule{\linewidth}{0.5pt}
\subsection*{\textsf{\colorbox{headtoc}{\color{white} FUNCTION}
dscal}}

\hypertarget{ecldoc:blas.dscal}{}
\hspace{0pt} \hyperlink{ecldoc:blas}{BLAS} \textbackslash 

{\renewcommand{\arraystretch}{1.5}
\begin{tabularx}{\textwidth}{|>{\raggedright\arraybackslash}l|X|}
\hline
\hspace{0pt}\mytexttt{\color{red} Types.matrix\_t} & \textbf{dscal} \\
\hline
\multicolumn{2}{|>{\raggedright\arraybackslash}X|}{\hspace{0pt}\mytexttt{\color{param} (Types.dimension\_t N, Types.value\_t alpha, Types.matrix\_t X, Types.dimension\_t incX, Types.dimension\_t skipped=0)}} \\
\hline
\end{tabularx}
}

\par





Scale a vector alpha






\par
\begin{description}
\item [\colorbox{tagtype}{\color{white} \textbf{\textsf{PARAMETER}}}] \textbf{\underline{X}} ||| SET ( REAL8 ) --- the column major matrix holding the vector
\item [\colorbox{tagtype}{\color{white} \textbf{\textsf{PARAMETER}}}] \textbf{\underline{incX}} ||| UNSIGNED4 --- the stride to get to the next element in the vector
\item [\colorbox{tagtype}{\color{white} \textbf{\textsf{PARAMETER}}}] \textbf{\underline{N}} ||| UNSIGNED4 --- number of elements in the vector
\item [\colorbox{tagtype}{\color{white} \textbf{\textsf{PARAMETER}}}] \textbf{\underline{alpha}} ||| REAL8 --- the scaling factor
\item [\colorbox{tagtype}{\color{white} \textbf{\textsf{PARAMETER}}}] \textbf{\underline{skipped}} ||| UNSIGNED4 --- the number of elements skipped to get to the first element
\end{description}







\par
\begin{description}
\item [\colorbox{tagtype}{\color{white} \textbf{\textsf{RETURN}}}] \textbf{SET ( REAL8 )} --- the updated matrix
\end{description}




\rule{\linewidth}{0.5pt}
\subsection*{\textsf{\colorbox{headtoc}{\color{white} FUNCTION}
dsyrk}}

\hypertarget{ecldoc:blas.dsyrk}{}
\hspace{0pt} \hyperlink{ecldoc:blas}{BLAS} \textbackslash 

{\renewcommand{\arraystretch}{1.5}
\begin{tabularx}{\textwidth}{|>{\raggedright\arraybackslash}l|X|}
\hline
\hspace{0pt}\mytexttt{\color{red} Types.matrix\_t} & \textbf{dsyrk} \\
\hline
\multicolumn{2}{|>{\raggedright\arraybackslash}X|}{\hspace{0pt}\mytexttt{\color{param} (Types.Triangle tri, BOOLEAN transposeA, Types.dimension\_t N, Types.dimension\_t K, Types.value\_t alpha, Types.matrix\_t A, Types.value\_t beta, Types.matrix\_t C, BOOLEAN clear=FALSE)}} \\
\hline
\end{tabularx}
}

\par





Implements symmetric rank update C 






\par
\begin{description}
\item [\colorbox{tagtype}{\color{white} \textbf{\textsf{PARAMETER}}}] \textbf{\underline{A}} ||| SET ( REAL8 ) --- the update matrix, either NxK or KxN
\item [\colorbox{tagtype}{\color{white} \textbf{\textsf{PARAMETER}}}] \textbf{\underline{transposeA}} ||| BOOLEAN --- Transpose the A matrix to be NxK
\item [\colorbox{tagtype}{\color{white} \textbf{\textsf{PARAMETER}}}] \textbf{\underline{N}} ||| UNSIGNED4 --- number of rows
\item [\colorbox{tagtype}{\color{white} \textbf{\textsf{PARAMETER}}}] \textbf{\underline{K}} ||| UNSIGNED4 --- number of columns in the update matrix or transpose
\item [\colorbox{tagtype}{\color{white} \textbf{\textsf{PARAMETER}}}] \textbf{\underline{beta}} ||| REAL8 --- the beta scalar
\item [\colorbox{tagtype}{\color{white} \textbf{\textsf{PARAMETER}}}] \textbf{\underline{tri}} ||| UNSIGNED1 --- update upper or lower triangle
\item [\colorbox{tagtype}{\color{white} \textbf{\textsf{PARAMETER}}}] \textbf{\underline{C}} ||| SET ( REAL8 ) --- the matrix to update
\item [\colorbox{tagtype}{\color{white} \textbf{\textsf{PARAMETER}}}] \textbf{\underline{alpha}} ||| REAL8 --- the alpha scalar
\item [\colorbox{tagtype}{\color{white} \textbf{\textsf{PARAMETER}}}] \textbf{\underline{clear}} ||| BOOLEAN --- clear the triangle that is not updated. BLAS assumes that symmetric matrices have only one of the triangles and this option lets you make that true.
\end{description}







\par
\begin{description}
\item [\colorbox{tagtype}{\color{white} \textbf{\textsf{RETURN}}}] \textbf{SET ( REAL8 )} --- 
\end{description}




\rule{\linewidth}{0.5pt}
\subsection*{\textsf{\colorbox{headtoc}{\color{white} FUNCTION}
dtrsm}}

\hypertarget{ecldoc:blas.dtrsm}{}
\hspace{0pt} \hyperlink{ecldoc:blas}{BLAS} \textbackslash 

{\renewcommand{\arraystretch}{1.5}
\begin{tabularx}{\textwidth}{|>{\raggedright\arraybackslash}l|X|}
\hline
\hspace{0pt}\mytexttt{\color{red} Types.matrix\_t} & \textbf{dtrsm} \\
\hline
\multicolumn{2}{|>{\raggedright\arraybackslash}X|}{\hspace{0pt}\mytexttt{\color{param} (Types.Side side, Types.Triangle tri, BOOLEAN transposeA, Types.Diagonal diag, Types.dimension\_t M, Types.dimension\_t N, Types.dimension\_t lda, Types.value\_t alpha, Types.matrix\_t A, Types.matrix\_t B)}} \\
\hline
\end{tabularx}
}

\par





Triangular matrix solver. op(A) X = alpha B or X op(A) = alpha B where op is Transpose, X and B is MxN






\par
\begin{description}
\item [\colorbox{tagtype}{\color{white} \textbf{\textsf{PARAMETER}}}] \textbf{\underline{lda}} ||| UNSIGNED4 --- the leading dimension of the A matrix, either M or N
\item [\colorbox{tagtype}{\color{white} \textbf{\textsf{PARAMETER}}}] \textbf{\underline{transposeA}} ||| BOOLEAN --- is op(A) the transpose of A
\item [\colorbox{tagtype}{\color{white} \textbf{\textsf{PARAMETER}}}] \textbf{\underline{diag}} ||| UNSIGNED1 --- is the diagonal an implied unit diagonal or supplied
\item [\colorbox{tagtype}{\color{white} \textbf{\textsf{PARAMETER}}}] \textbf{\underline{side}} ||| UNSIGNED1 --- side for A, Side.Ax is op(A) X = alpha B
\item [\colorbox{tagtype}{\color{white} \textbf{\textsf{PARAMETER}}}] \textbf{\underline{tri}} ||| UNSIGNED1 --- Says whether A is Upper or Lower triangle
\item [\colorbox{tagtype}{\color{white} \textbf{\textsf{PARAMETER}}}] \textbf{\underline{alpha}} ||| REAL8 --- the scalar multiplier for B
\item [\colorbox{tagtype}{\color{white} \textbf{\textsf{PARAMETER}}}] \textbf{\underline{B}} ||| SET ( REAL8 ) --- the matrix of values for the solve
\item [\colorbox{tagtype}{\color{white} \textbf{\textsf{PARAMETER}}}] \textbf{\underline{M}} ||| UNSIGNED4 --- number of rows
\item [\colorbox{tagtype}{\color{white} \textbf{\textsf{PARAMETER}}}] \textbf{\underline{N}} ||| UNSIGNED4 --- number of columns
\item [\colorbox{tagtype}{\color{white} \textbf{\textsf{PARAMETER}}}] \textbf{\underline{A}} ||| SET ( REAL8 ) --- a triangular matrix
\end{description}







\par
\begin{description}
\item [\colorbox{tagtype}{\color{white} \textbf{\textsf{RETURN}}}] \textbf{SET ( REAL8 )} --- the matrix of coefficients to get B.
\end{description}




\rule{\linewidth}{0.5pt}
\subsection*{\textsf{\colorbox{headtoc}{\color{white} FUNCTION}
extract\_diag}}

\hypertarget{ecldoc:blas.extract_diag}{}
\hspace{0pt} \hyperlink{ecldoc:blas}{BLAS} \textbackslash 

{\renewcommand{\arraystretch}{1.5}
\begin{tabularx}{\textwidth}{|>{\raggedright\arraybackslash}l|X|}
\hline
\hspace{0pt}\mytexttt{\color{red} Types.matrix\_t} & \textbf{extract\_diag} \\
\hline
\multicolumn{2}{|>{\raggedright\arraybackslash}X|}{\hspace{0pt}\mytexttt{\color{param} (Types.dimension\_t m, Types.dimension\_t n, Types.matrix\_t x)}} \\
\hline
\end{tabularx}
}

\par





Extract the diagonal of he matrix






\par
\begin{description}
\item [\colorbox{tagtype}{\color{white} \textbf{\textsf{PARAMETER}}}] \textbf{\underline{m}} ||| UNSIGNED4 --- number of rows
\item [\colorbox{tagtype}{\color{white} \textbf{\textsf{PARAMETER}}}] \textbf{\underline{x}} ||| SET ( REAL8 ) --- matrix from which to extract the diagonal
\item [\colorbox{tagtype}{\color{white} \textbf{\textsf{PARAMETER}}}] \textbf{\underline{n}} ||| UNSIGNED4 --- number of columns
\end{description}







\par
\begin{description}
\item [\colorbox{tagtype}{\color{white} \textbf{\textsf{RETURN}}}] \textbf{SET ( REAL8 )} --- diagonal matrix
\end{description}




\rule{\linewidth}{0.5pt}
\subsection*{\textsf{\colorbox{headtoc}{\color{white} FUNCTION}
extract\_tri}}

\hypertarget{ecldoc:blas.extract_tri}{}
\hspace{0pt} \hyperlink{ecldoc:blas}{BLAS} \textbackslash 

{\renewcommand{\arraystretch}{1.5}
\begin{tabularx}{\textwidth}{|>{\raggedright\arraybackslash}l|X|}
\hline
\hspace{0pt}\mytexttt{\color{red} Types.matrix\_t} & \textbf{extract\_tri} \\
\hline
\multicolumn{2}{|>{\raggedright\arraybackslash}X|}{\hspace{0pt}\mytexttt{\color{param} (Types.dimension\_t m, Types.dimension\_t n, Types.Triangle tri, Types.Diagonal dt, Types.matrix\_t a)}} \\
\hline
\end{tabularx}
}

\par





Extract the upper or lower triangle. Diagonal can be actual or implied unit diagonal.






\par
\begin{description}
\item [\colorbox{tagtype}{\color{white} \textbf{\textsf{PARAMETER}}}] \textbf{\underline{tri}} ||| UNSIGNED1 --- Upper or Lower specifier, Triangle.Lower or Triangle.Upper
\item [\colorbox{tagtype}{\color{white} \textbf{\textsf{PARAMETER}}}] \textbf{\underline{m}} ||| UNSIGNED4 --- number of rows
\item [\colorbox{tagtype}{\color{white} \textbf{\textsf{PARAMETER}}}] \textbf{\underline{n}} ||| UNSIGNED4 --- number of columns
\item [\colorbox{tagtype}{\color{white} \textbf{\textsf{PARAMETER}}}] \textbf{\underline{a}} ||| SET ( REAL8 ) --- Matrix, usually a composite from factoring
\item [\colorbox{tagtype}{\color{white} \textbf{\textsf{PARAMETER}}}] \textbf{\underline{dt}} ||| UNSIGNED1 --- Use Diagonal.NotUnitTri or Diagonal.UnitTri
\end{description}







\par
\begin{description}
\item [\colorbox{tagtype}{\color{white} \textbf{\textsf{RETURN}}}] \textbf{SET ( REAL8 )} --- the triangle
\end{description}




\rule{\linewidth}{0.5pt}
\subsection*{\textsf{\colorbox{headtoc}{\color{white} FUNCTION}
make\_diag}}

\hypertarget{ecldoc:blas.make_diag}{}
\hspace{0pt} \hyperlink{ecldoc:blas}{BLAS} \textbackslash 

{\renewcommand{\arraystretch}{1.5}
\begin{tabularx}{\textwidth}{|>{\raggedright\arraybackslash}l|X|}
\hline
\hspace{0pt}\mytexttt{\color{red} Types.matrix\_t} & \textbf{make\_diag} \\
\hline
\multicolumn{2}{|>{\raggedright\arraybackslash}X|}{\hspace{0pt}\mytexttt{\color{param} (Types.dimension\_t m, Types.value\_t v=1.0, Types.matrix\_t X=[])}} \\
\hline
\end{tabularx}
}

\par





Generate a diagonal matrix.






\par
\begin{description}
\item [\colorbox{tagtype}{\color{white} \textbf{\textsf{PARAMETER}}}] \textbf{\underline{X}} ||| SET ( REAL8 ) --- optional input of diagonal values, multiplied by v.
\item [\colorbox{tagtype}{\color{white} \textbf{\textsf{PARAMETER}}}] \textbf{\underline{m}} ||| UNSIGNED4 --- number of diagonal entries
\item [\colorbox{tagtype}{\color{white} \textbf{\textsf{PARAMETER}}}] \textbf{\underline{v}} ||| REAL8 --- option value, defaults to 1
\end{description}







\par
\begin{description}
\item [\colorbox{tagtype}{\color{white} \textbf{\textsf{RETURN}}}] \textbf{SET ( REAL8 )} --- a diagonal matrix
\end{description}




\rule{\linewidth}{0.5pt}
\subsection*{\textsf{\colorbox{headtoc}{\color{white} FUNCTION}
make\_vector}}

\hypertarget{ecldoc:blas.make_vector}{}
\hspace{0pt} \hyperlink{ecldoc:blas}{BLAS} \textbackslash 

{\renewcommand{\arraystretch}{1.5}
\begin{tabularx}{\textwidth}{|>{\raggedright\arraybackslash}l|X|}
\hline
\hspace{0pt}\mytexttt{\color{red} Types.matrix\_t} & \textbf{make\_vector} \\
\hline
\multicolumn{2}{|>{\raggedright\arraybackslash}X|}{\hspace{0pt}\mytexttt{\color{param} (Types.dimension\_t m, Types.value\_t v=1.0)}} \\
\hline
\end{tabularx}
}

\par





Make a vector of dimension m






\par
\begin{description}
\item [\colorbox{tagtype}{\color{white} \textbf{\textsf{PARAMETER}}}] \textbf{\underline{m}} ||| UNSIGNED4 --- number of elements
\item [\colorbox{tagtype}{\color{white} \textbf{\textsf{PARAMETER}}}] \textbf{\underline{v}} ||| REAL8 --- the values, defaults to 1
\end{description}







\par
\begin{description}
\item [\colorbox{tagtype}{\color{white} \textbf{\textsf{RETURN}}}] \textbf{SET ( REAL8 )} --- the vector
\end{description}




\rule{\linewidth}{0.5pt}
\subsection*{\textsf{\colorbox{headtoc}{\color{white} FUNCTION}
trace}}

\hypertarget{ecldoc:blas.trace}{}
\hspace{0pt} \hyperlink{ecldoc:blas}{BLAS} \textbackslash 

{\renewcommand{\arraystretch}{1.5}
\begin{tabularx}{\textwidth}{|>{\raggedright\arraybackslash}l|X|}
\hline
\hspace{0pt}\mytexttt{\color{red} Types.value\_t} & \textbf{trace} \\
\hline
\multicolumn{2}{|>{\raggedright\arraybackslash}X|}{\hspace{0pt}\mytexttt{\color{param} (Types.dimension\_t m, Types.dimension\_t n, Types.matrix\_t x)}} \\
\hline
\end{tabularx}
}

\par





The trace of the input matrix






\par
\begin{description}
\item [\colorbox{tagtype}{\color{white} \textbf{\textsf{PARAMETER}}}] \textbf{\underline{m}} ||| UNSIGNED4 --- number of rows
\item [\colorbox{tagtype}{\color{white} \textbf{\textsf{PARAMETER}}}] \textbf{\underline{x}} ||| SET ( REAL8 ) --- the matrix
\item [\colorbox{tagtype}{\color{white} \textbf{\textsf{PARAMETER}}}] \textbf{\underline{n}} ||| UNSIGNED4 --- number of columns
\end{description}







\par
\begin{description}
\item [\colorbox{tagtype}{\color{white} \textbf{\textsf{RETURN}}}] \textbf{REAL8} --- the trace (sum of the diagonal entries)
\end{description}




\rule{\linewidth}{0.5pt}


