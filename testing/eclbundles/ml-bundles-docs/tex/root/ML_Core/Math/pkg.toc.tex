\chapter*{\color{headtoc} Math}
\hypertarget{ecldoc:toc:root/ML_Core/Math}{}
\hyperlink{ecldoc:toc:root/ML_Core}{Go Up}


\section*{Table of Contents}
{\renewcommand{\arraystretch}{1.5}
\begin{longtable}{|p{\textwidth}|}
\hline
\hyperlink{ecldoc:toc:ML_Core.Math.Beta}{Beta.ecl} \\
Return the beta value of two positive real numbers, x and y \\
\hline
\hyperlink{ecldoc:toc:ML_Core.Math.Distributions}{Distributions.ecl} \\
\hline
\hyperlink{ecldoc:toc:ML_Core.Math.DoubleFac}{DoubleFac.ecl} \\
The 'double' factorial is defined for ODD n and is the product of all the odd numbers up to and including that number \\
\hline
\hyperlink{ecldoc:toc:ML_Core.Math.Fac}{Fac.ecl} \\
Factorial function \\
\hline
\hyperlink{ecldoc:toc:ML_Core.Math.gamma}{gamma.ecl} \\
Return the value of gamma function of real number x A wrapper for the standard C tgamma function \\
\hline
\hyperlink{ecldoc:toc:ML_Core.Math.log_gamma}{log\_gamma.ecl} \\
Return the value of the log gamma function of the absolute value of X \\
\hline
\hyperlink{ecldoc:toc:ML_Core.Math.lowerGamma}{lowerGamma.ecl} \\
Return the lower incomplete gamma value of two real numbers, \\
\hline
\hyperlink{ecldoc:toc:ML_Core.Math.NCK}{NCK.ecl} \\
\hline
\hyperlink{ecldoc:toc:ML_Core.Math.Poly}{Poly.ecl} \\
Evaluate a polynomial from a set of co-effs \\
\hline
\hyperlink{ecldoc:toc:ML_Core.Math.StirlingFormula}{StirlingFormula.ecl} \\
Stirling's formula \\
\hline
\hyperlink{ecldoc:toc:ML_Core.Math.upperGamma}{upperGamma.ecl} \\
Return the upper incomplete gamma value of two real numbers, x and y \\
\hline
\end{longtable}
}

\chapter*{\color{headfile}
{\large ML\_Core\slash\hspace{0pt}}
{\large Math\slash\hspace{0pt}}
 \\
Beta
}
\hypertarget{ecldoc:toc:ML_Core.Math.Beta}{}
\hyperlink{ecldoc:toc:root/ML_Core/Math}{Go Up}

\section*{\underline{\textsf{IMPORTS}}}
\begin{doublespace}
{\large
ML\_Core.Math |
}
\end{doublespace}

\section*{\underline{\textsf{DESCRIPTIONS}}}
\subsection*{\textsf{\colorbox{headtoc}{\color{white} FUNCTION}
Beta}}

\hypertarget{ecldoc:ml_core.math.beta}{}

{\renewcommand{\arraystretch}{1.5}
\begin{tabularx}{\textwidth}{|>{\raggedright\arraybackslash}l|X|}
\hline
\hspace{0pt}\mytexttt{\color{red} } & \textbf{Beta} \\
\hline
\multicolumn{2}{|>{\raggedright\arraybackslash}X|}{\hspace{0pt}\mytexttt{\color{param} (REAL8 x, REAL8 y)}} \\
\hline
\end{tabularx}
}

\par





Return the beta value of two positive real numbers, x and y






\par
\begin{description}
\item [\colorbox{tagtype}{\color{white} \textbf{\textsf{PARAMETER}}}] \textbf{\underline{y}} ||| REAL8 --- the value of the second number
\item [\colorbox{tagtype}{\color{white} \textbf{\textsf{PARAMETER}}}] \textbf{\underline{x}} ||| REAL8 --- the value of the first number
\end{description}







\par
\begin{description}
\item [\colorbox{tagtype}{\color{white} \textbf{\textsf{RETURN}}}] \textbf{REAL8} --- the beta value
\end{description}




\rule{\linewidth}{0.5pt}

\chapter*{\color{headfile}
{\large LogisticRegression\slash\hspace{0pt}}
 \\
Distributions
}
\hypertarget{ecldoc:toc:LogisticRegression.Distributions}{}
\hyperlink{ecldoc:toc:root/LogisticRegression}{Go Up}

\section*{\underline{\textsf{IMPORTS}}}
\begin{doublespace}
{\large
ML\_Core.Constants |
ML\_Core.Math |
}
\end{doublespace}

\section*{\underline{\textsf{DESCRIPTIONS}}}
\subsection*{\textsf{\colorbox{headtoc}{\color{white} MODULE}
Distributions}}

\hypertarget{ecldoc:LogisticRegression.Distributions}{}

{\renewcommand{\arraystretch}{1.5}
\begin{tabularx}{\textwidth}{|>{\raggedright\arraybackslash}l|X|}
\hline
\hspace{0pt}\mytexttt{\color{red} } & \textbf{Distributions} \\
\hline
\end{tabularx}
}

\par





No Documentation Found







\textbf{Children}
\begin{enumerate}
\item \hyperlink{ecldoc:logisticregression.distributions.normal_cdf}{Normal\_CDF}
: Cumulative Distribution of the standard normal distribution, the probability that a normal random variable will be smaller than x standard deviations above or below the mean
\item \hyperlink{ecldoc:logisticregression.distributions.normal_ppf}{Normal\_PPF}
: Normal Distribution Percentage Point Function
\item \hyperlink{ecldoc:logisticregression.distributions.t_cdf}{T\_CDF}
: Students t distribution integral evaluated between negative infinity and x
\item \hyperlink{ecldoc:logisticregression.distributions.t_ppf}{T\_PPF}
: Percentage point function for the T distribution
\item \hyperlink{ecldoc:logisticregression.distributions.chi2_cdf}{Chi2\_CDF}
: The cumulative distribution function for the Chi Square distribution
\item \hyperlink{ecldoc:logisticregression.distributions.chi2_ppf}{Chi2\_PPF}
: The Chi Squared PPF function
\end{enumerate}

\rule{\linewidth}{0.5pt}

\subsection*{\textsf{\colorbox{headtoc}{\color{white} FUNCTION}
Normal\_CDF}}

\hypertarget{ecldoc:logisticregression.distributions.normal_cdf}{}
\hspace{0pt} \hyperlink{ecldoc:LogisticRegression.Distributions}{Distributions} \textbackslash 

{\renewcommand{\arraystretch}{1.5}
\begin{tabularx}{\textwidth}{|>{\raggedright\arraybackslash}l|X|}
\hline
\hspace{0pt}\mytexttt{\color{red} REAL8} & \textbf{Normal\_CDF} \\
\hline
\multicolumn{2}{|>{\raggedright\arraybackslash}X|}{\hspace{0pt}\mytexttt{\color{param} (REAL8 x)}} \\
\hline
\end{tabularx}
}

\par





Cumulative Distribution of the standard normal distribution, the probability that a normal random variable will be smaller than x standard deviations above or below the mean. Taken from C/C++ Mathematical Algorithms for Scientists and Engineers, n. Shammas, McGraw-Hill, 1995






\par
\begin{description}
\item [\colorbox{tagtype}{\color{white} \textbf{\textsf{PARAMETER}}}] \textbf{\underline{x}} ||| REAL8 --- the number of standard deviations
\end{description}







\par
\begin{description}
\item [\colorbox{tagtype}{\color{white} \textbf{\textsf{RETURN}}}] \textbf{REAL8} --- 
\end{description}






\par
\begin{description}
\item [\colorbox{tagtype}{\color{white} \textbf{\textsf{RETURNS}}}] probability of exceeding x.
\end{description}




\rule{\linewidth}{0.5pt}
\subsection*{\textsf{\colorbox{headtoc}{\color{white} FUNCTION}
Normal\_PPF}}

\hypertarget{ecldoc:logisticregression.distributions.normal_ppf}{}
\hspace{0pt} \hyperlink{ecldoc:LogisticRegression.Distributions}{Distributions} \textbackslash 

{\renewcommand{\arraystretch}{1.5}
\begin{tabularx}{\textwidth}{|>{\raggedright\arraybackslash}l|X|}
\hline
\hspace{0pt}\mytexttt{\color{red} REAL8} & \textbf{Normal\_PPF} \\
\hline
\multicolumn{2}{|>{\raggedright\arraybackslash}X|}{\hspace{0pt}\mytexttt{\color{param} (REAL8 x)}} \\
\hline
\end{tabularx}
}

\par





Normal Distribution Percentage Point Function. Translated from C/C++ Mathematical Algorithms for Scientists and Engineers, N. Shammas, McGraw-Hill, 1995






\par
\begin{description}
\item [\colorbox{tagtype}{\color{white} \textbf{\textsf{PARAMETER}}}] \textbf{\underline{x}} ||| REAL8 --- probability
\end{description}







\par
\begin{description}
\item [\colorbox{tagtype}{\color{white} \textbf{\textsf{RETURN}}}] \textbf{REAL8} --- 
\end{description}






\par
\begin{description}
\item [\colorbox{tagtype}{\color{white} \textbf{\textsf{RETURNS}}}] number of standard deviations from the mean
\end{description}




\rule{\linewidth}{0.5pt}
\subsection*{\textsf{\colorbox{headtoc}{\color{white} FUNCTION}
T\_CDF}}

\hypertarget{ecldoc:logisticregression.distributions.t_cdf}{}
\hspace{0pt} \hyperlink{ecldoc:LogisticRegression.Distributions}{Distributions} \textbackslash 

{\renewcommand{\arraystretch}{1.5}
\begin{tabularx}{\textwidth}{|>{\raggedright\arraybackslash}l|X|}
\hline
\hspace{0pt}\mytexttt{\color{red} REAL8} & \textbf{T\_CDF} \\
\hline
\multicolumn{2}{|>{\raggedright\arraybackslash}X|}{\hspace{0pt}\mytexttt{\color{param} (REAL8 x, REAL8 df)}} \\
\hline
\end{tabularx}
}

\par





Students t distribution integral evaluated between negative infinity and x. Translated from NIST SEL DATAPAC Fortran TCDF.f source






\par
\begin{description}
\item [\colorbox{tagtype}{\color{white} \textbf{\textsf{PARAMETER}}}] \textbf{\underline{df}} ||| REAL8 --- degrees of freedom
\item [\colorbox{tagtype}{\color{white} \textbf{\textsf{PARAMETER}}}] \textbf{\underline{x}} ||| REAL8 --- value of the evaluation
\end{description}







\par
\begin{description}
\item [\colorbox{tagtype}{\color{white} \textbf{\textsf{RETURN}}}] \textbf{REAL8} --- 
\end{description}






\par
\begin{description}
\item [\colorbox{tagtype}{\color{white} \textbf{\textsf{RETURNS}}}] the probability that a value will be less than the specified value
\end{description}




\rule{\linewidth}{0.5pt}
\subsection*{\textsf{\colorbox{headtoc}{\color{white} FUNCTION}
T\_PPF}}

\hypertarget{ecldoc:logisticregression.distributions.t_ppf}{}
\hspace{0pt} \hyperlink{ecldoc:LogisticRegression.Distributions}{Distributions} \textbackslash 

{\renewcommand{\arraystretch}{1.5}
\begin{tabularx}{\textwidth}{|>{\raggedright\arraybackslash}l|X|}
\hline
\hspace{0pt}\mytexttt{\color{red} REAL8} & \textbf{T\_PPF} \\
\hline
\multicolumn{2}{|>{\raggedright\arraybackslash}X|}{\hspace{0pt}\mytexttt{\color{param} (REAL8 x, REAL8 df)}} \\
\hline
\end{tabularx}
}

\par





Percentage point function for the T distribution. Translated from NIST SEL DATAPAC Fortran TPPF.f source






\par
\begin{description}
\item [\colorbox{tagtype}{\color{white} \textbf{\textsf{PARAMETER}}}] \textbf{\underline{df}} ||| REAL8 --- No Doc
\item [\colorbox{tagtype}{\color{white} \textbf{\textsf{PARAMETER}}}] \textbf{\underline{x}} ||| REAL8 --- No Doc
\end{description}







\par
\begin{description}
\item [\colorbox{tagtype}{\color{white} \textbf{\textsf{RETURN}}}] \textbf{REAL8} --- 
\end{description}




\rule{\linewidth}{0.5pt}
\subsection*{\textsf{\colorbox{headtoc}{\color{white} FUNCTION}
Chi2\_CDF}}

\hypertarget{ecldoc:logisticregression.distributions.chi2_cdf}{}
\hspace{0pt} \hyperlink{ecldoc:LogisticRegression.Distributions}{Distributions} \textbackslash 

{\renewcommand{\arraystretch}{1.5}
\begin{tabularx}{\textwidth}{|>{\raggedright\arraybackslash}l|X|}
\hline
\hspace{0pt}\mytexttt{\color{red} REAL8} & \textbf{Chi2\_CDF} \\
\hline
\multicolumn{2}{|>{\raggedright\arraybackslash}X|}{\hspace{0pt}\mytexttt{\color{param} (REAL8 x, REAL8 df)}} \\
\hline
\end{tabularx}
}

\par





The cumulative distribution function for the Chi Square distribution. the CDF for the specfied degrees of freedom. Translated from the NIST SEL DATAPAC Fortran subroutine CHSCDF.






\par
\begin{description}
\item [\colorbox{tagtype}{\color{white} \textbf{\textsf{PARAMETER}}}] \textbf{\underline{df}} ||| REAL8 --- No Doc
\item [\colorbox{tagtype}{\color{white} \textbf{\textsf{PARAMETER}}}] \textbf{\underline{x}} ||| REAL8 --- No Doc
\end{description}







\par
\begin{description}
\item [\colorbox{tagtype}{\color{white} \textbf{\textsf{RETURN}}}] \textbf{REAL8} --- 
\end{description}




\rule{\linewidth}{0.5pt}
\subsection*{\textsf{\colorbox{headtoc}{\color{white} FUNCTION}
Chi2\_PPF}}

\hypertarget{ecldoc:logisticregression.distributions.chi2_ppf}{}
\hspace{0pt} \hyperlink{ecldoc:LogisticRegression.Distributions}{Distributions} \textbackslash 

{\renewcommand{\arraystretch}{1.5}
\begin{tabularx}{\textwidth}{|>{\raggedright\arraybackslash}l|X|}
\hline
\hspace{0pt}\mytexttt{\color{red} REAL8} & \textbf{Chi2\_PPF} \\
\hline
\multicolumn{2}{|>{\raggedright\arraybackslash}X|}{\hspace{0pt}\mytexttt{\color{param} (REAL8 x, REAL8 df)}} \\
\hline
\end{tabularx}
}

\par





The Chi Squared PPF function. Translated from the NIST SEL DATAPAC Fortran subroutine CHSPPF.






\par
\begin{description}
\item [\colorbox{tagtype}{\color{white} \textbf{\textsf{PARAMETER}}}] \textbf{\underline{df}} ||| REAL8 --- No Doc
\item [\colorbox{tagtype}{\color{white} \textbf{\textsf{PARAMETER}}}] \textbf{\underline{x}} ||| REAL8 --- No Doc
\end{description}







\par
\begin{description}
\item [\colorbox{tagtype}{\color{white} \textbf{\textsf{RETURN}}}] \textbf{REAL8} --- 
\end{description}




\rule{\linewidth}{0.5pt}



\chapter*{\color{headfile}
{\large ML\_Core\slash\hspace{0pt}}
{\large Math\slash\hspace{0pt}}
 \\
DoubleFac
}
\hypertarget{ecldoc:toc:ML_Core.Math.DoubleFac}{}
\hyperlink{ecldoc:toc:root/ML_Core/Math}{Go Up}


\section*{\underline{\textsf{DESCRIPTIONS}}}
\subsection*{\textsf{\colorbox{headtoc}{\color{white} EMBED}
DoubleFac}}

\hypertarget{ecldoc:ml_core.math.doublefac}{}

{\renewcommand{\arraystretch}{1.5}
\begin{tabularx}{\textwidth}{|>{\raggedright\arraybackslash}l|X|}
\hline
\hspace{0pt}\mytexttt{\color{red} REAL8} & \textbf{DoubleFac} \\
\hline
\multicolumn{2}{|>{\raggedright\arraybackslash}X|}{\hspace{0pt}\mytexttt{\color{param} (INTEGER2 i)}} \\
\hline
\end{tabularx}
}

\par





The 'double' factorial is defined for ODD n and is the product of all the odd numbers up to and including that number. We are extending the meaning to even numbers to mean the product of the even numbers up to and including that number. Thus DoubleFac(8) = 8*6*4*2 We also defend against i < 2 (returning 1.0)






\par
\begin{description}
\item [\colorbox{tagtype}{\color{white} \textbf{\textsf{PARAMETER}}}] \textbf{\underline{i}} ||| INTEGER2 --- the value used in the calculation
\end{description}







\par
\begin{description}
\item [\colorbox{tagtype}{\color{white} \textbf{\textsf{RETURN}}}] \textbf{REAL8} --- the factorial of the sequence, declining by 2
\end{description}




\rule{\linewidth}{0.5pt}

\chapter*{\color{headfile}
{\large ML\_Core\slash\hspace{0pt}}
{\large Math\slash\hspace{0pt}}
 \\
Fac
}
\hypertarget{ecldoc:toc:ML_Core.Math.Fac}{}
\hyperlink{ecldoc:toc:root/ML_Core/Math}{Go Up}


\section*{\underline{\textsf{DESCRIPTIONS}}}
\subsection*{\textsf{\colorbox{headtoc}{\color{white} EMBED}
Fac}}

\hypertarget{ecldoc:ml_core.math.fac}{}

{\renewcommand{\arraystretch}{1.5}
\begin{tabularx}{\textwidth}{|>{\raggedright\arraybackslash}l|X|}
\hline
\hspace{0pt}\mytexttt{\color{red} REAL8} & \textbf{Fac} \\
\hline
\multicolumn{2}{|>{\raggedright\arraybackslash}X|}{\hspace{0pt}\mytexttt{\color{param} (UNSIGNED2 i)}} \\
\hline
\end{tabularx}
}

\par





Factorial function






\par
\begin{description}
\item [\colorbox{tagtype}{\color{white} \textbf{\textsf{PARAMETER}}}] \textbf{\underline{i}} ||| UNSIGNED2 --- the value used, (i)(i-1)(i-2)\ldots(2)
\end{description}







\par
\begin{description}
\item [\colorbox{tagtype}{\color{white} \textbf{\textsf{RETURN}}}] \textbf{REAL8} --- the factorial i!
\end{description}




\rule{\linewidth}{0.5pt}

\chapter*{\color{headfile}
{\large ML\_Core\slash\hspace{0pt}}
{\large Math\slash\hspace{0pt}}
 \\
gamma
}
\hypertarget{ecldoc:toc:ML_Core.Math.gamma}{}
\hyperlink{ecldoc:toc:root/ML_Core/Math}{Go Up}


\section*{\underline{\textsf{DESCRIPTIONS}}}
\subsection*{\textsf{\colorbox{headtoc}{\color{white} EMBED}
gamma}}

\hypertarget{ecldoc:ml_core.math.gamma}{}

{\renewcommand{\arraystretch}{1.5}
\begin{tabularx}{\textwidth}{|>{\raggedright\arraybackslash}l|X|}
\hline
\hspace{0pt}\mytexttt{\color{red} REAL8} & \textbf{gamma} \\
\hline
\multicolumn{2}{|>{\raggedright\arraybackslash}X|}{\hspace{0pt}\mytexttt{\color{param} (REAL8 x)}} \\
\hline
\end{tabularx}
}

\par





Return the value of gamma function of real number x A wrapper for the standard C tgamma function.






\par
\begin{description}
\item [\colorbox{tagtype}{\color{white} \textbf{\textsf{PARAMETER}}}] \textbf{\underline{x}} ||| REAL8 --- the input x
\end{description}







\par
\begin{description}
\item [\colorbox{tagtype}{\color{white} \textbf{\textsf{RETURN}}}] \textbf{REAL8} --- the value of GAMMA evaluated at x
\end{description}




\rule{\linewidth}{0.5pt}

\chapter*{\color{headfile}
{\large ML\_Core\slash\hspace{0pt}}
{\large Math\slash\hspace{0pt}}
 \\
log_gamma
}
\hypertarget{ecldoc:toc:ML_Core.Math.log_gamma}{}
\hyperlink{ecldoc:toc:root/ML_Core/Math}{Go Up}


\section*{\underline{\textsf{DESCRIPTIONS}}}
\subsection*{\textsf{\colorbox{headtoc}{\color{white} EMBED}
log\_gamma}}

\hypertarget{ecldoc:ml_core.math.log_gamma}{}

{\renewcommand{\arraystretch}{1.5}
\begin{tabularx}{\textwidth}{|>{\raggedright\arraybackslash}l|X|}
\hline
\hspace{0pt}\mytexttt{\color{red} REAL8} & \textbf{log\_gamma} \\
\hline
\multicolumn{2}{|>{\raggedright\arraybackslash}X|}{\hspace{0pt}\mytexttt{\color{param} (REAL8 x)}} \\
\hline
\end{tabularx}
}

\par





Return the value of the log gamma function of the absolute value of X. A wrapper for the standard C lgamma function. Avoids the race condition found on some platforms by taking the absolute value of the of the input argument.






\par
\begin{description}
\item [\colorbox{tagtype}{\color{white} \textbf{\textsf{PARAMETER}}}] \textbf{\underline{x}} ||| REAL8 --- the input x
\end{description}







\par
\begin{description}
\item [\colorbox{tagtype}{\color{white} \textbf{\textsf{RETURN}}}] \textbf{REAL8} --- the value of the log of the GAMMA evaluated at ABS(x)
\end{description}




\rule{\linewidth}{0.5pt}

\chapter*{\color{headfile}
{\large ML\_Core\slash\hspace{0pt}}
{\large Math\slash\hspace{0pt}}
 \\
lowerGamma
}
\hypertarget{ecldoc:toc:ML_Core.Math.lowerGamma}{}
\hyperlink{ecldoc:toc:root/ML_Core/Math}{Go Up}


\section*{\underline{\textsf{DESCRIPTIONS}}}
\subsection*{\textsf{\colorbox{headtoc}{\color{white} EMBED}
lowerGamma}}

\hypertarget{ecldoc:ml_core.math.lowergamma}{}

{\renewcommand{\arraystretch}{1.5}
\begin{tabularx}{\textwidth}{|>{\raggedright\arraybackslash}l|X|}
\hline
\hspace{0pt}\mytexttt{\color{red} REAL8} & \textbf{lowerGamma} \\
\hline
\multicolumn{2}{|>{\raggedright\arraybackslash}X|}{\hspace{0pt}\mytexttt{\color{param} (REAL8 x, REAL8 y)}} \\
\hline
\end{tabularx}
}

\par





Return the lower incomplete gamma value of two real numbers, x and y






\par
\begin{description}
\item [\colorbox{tagtype}{\color{white} \textbf{\textsf{PARAMETER}}}] \textbf{\underline{y}} ||| REAL8 --- the value of the second number
\item [\colorbox{tagtype}{\color{white} \textbf{\textsf{PARAMETER}}}] \textbf{\underline{x}} ||| REAL8 --- the value of the first number
\end{description}







\par
\begin{description}
\item [\colorbox{tagtype}{\color{white} \textbf{\textsf{RETURN}}}] \textbf{REAL8} --- the lower incomplete gamma value
\end{description}




\rule{\linewidth}{0.5pt}

\chapter*{\color{headfile}
{\large ML\_Core\slash\hspace{0pt}}
{\large Math\slash\hspace{0pt}}
 \\
NCK
}
\hypertarget{ecldoc:toc:ML_Core.Math.NCK}{}
\hyperlink{ecldoc:toc:root/ML_Core/Math}{Go Up}

\section*{\underline{\textsf{IMPORTS}}}
\begin{doublespace}
{\large
ML\_Core.Math |
}
\end{doublespace}

\section*{\underline{\textsf{DESCRIPTIONS}}}
\subsection*{\textsf{\colorbox{headtoc}{\color{white} FUNCTION}
NCK}}

\hypertarget{ecldoc:ml_core.math.nck}{}

{\renewcommand{\arraystretch}{1.5}
\begin{tabularx}{\textwidth}{|>{\raggedright\arraybackslash}l|X|}
\hline
\hspace{0pt}\mytexttt{\color{red} REAL8} & \textbf{NCK} \\
\hline
\multicolumn{2}{|>{\raggedright\arraybackslash}X|}{\hspace{0pt}\mytexttt{\color{param} (INTEGER2 N, INTEGER2 K)}} \\
\hline
\end{tabularx}
}

\par





No Documentation Found






\par
\begin{description}
\item [\colorbox{tagtype}{\color{white} \textbf{\textsf{PARAMETER}}}] \textbf{\underline{k}} ||| INTEGER2 --- No Doc
\item [\colorbox{tagtype}{\color{white} \textbf{\textsf{PARAMETER}}}] \textbf{\underline{n}} ||| INTEGER2 --- No Doc
\end{description}







\par
\begin{description}
\item [\colorbox{tagtype}{\color{white} \textbf{\textsf{RETURN}}}] \textbf{REAL8} --- 
\end{description}




\rule{\linewidth}{0.5pt}

\chapter*{\color{headfile}
{\large ML\_Core\slash\hspace{0pt}}
{\large Math\slash\hspace{0pt}}
 \\
Poly
}
\hypertarget{ecldoc:toc:ML_Core.Math.Poly}{}
\hyperlink{ecldoc:toc:root/ML_Core/Math}{Go Up}


\section*{\underline{\textsf{DESCRIPTIONS}}}
\subsection*{\textsf{\colorbox{headtoc}{\color{white} EMBED}
Poly}}

\hypertarget{ecldoc:ml_core.math.poly}{}

{\renewcommand{\arraystretch}{1.5}
\begin{tabularx}{\textwidth}{|>{\raggedright\arraybackslash}l|X|}
\hline
\hspace{0pt}\mytexttt{\color{red} REAL8} & \textbf{Poly} \\
\hline
\multicolumn{2}{|>{\raggedright\arraybackslash}X|}{\hspace{0pt}\mytexttt{\color{param} (REAL8 x, SET OF REAL8 Coeffs)}} \\
\hline
\end{tabularx}
}

\par





Evaluate a polynomial from a set of co-effs. Co-effs 1 is assumed to be the HIGH order of the equation. Thus for ax\^{}2+bx+c - the set would need to be Coef := [a,b,c];






\par
\begin{description}
\item [\colorbox{tagtype}{\color{white} \textbf{\textsf{PARAMETER}}}] \textbf{\underline{Coeffs}} ||| SET ( REAL8 ) --- a set of coefficients forthe polynomial. The ALL set is considered to be all zero values
\item [\colorbox{tagtype}{\color{white} \textbf{\textsf{PARAMETER}}}] \textbf{\underline{x}} ||| REAL8 --- the value of x in the polynomial
\end{description}







\par
\begin{description}
\item [\colorbox{tagtype}{\color{white} \textbf{\textsf{RETURN}}}] \textbf{REAL8} --- value of the polynomial at x
\end{description}




\rule{\linewidth}{0.5pt}

\chapter*{\color{headfile}
{\large ML\_Core\slash\hspace{0pt}}
{\large Math\slash\hspace{0pt}}
 \\
StirlingFormula
}
\hypertarget{ecldoc:toc:ML_Core.Math.StirlingFormula}{}
\hyperlink{ecldoc:toc:root/ML_Core/Math}{Go Up}

\section*{\underline{\textsf{IMPORTS}}}
\begin{doublespace}
{\large
ML\_Core.Math |
ML\_Core.Constants |
}
\end{doublespace}

\section*{\underline{\textsf{DESCRIPTIONS}}}
\subsection*{\textsf{\colorbox{headtoc}{\color{white} FUNCTION}
StirlingFormula}}

\hypertarget{ecldoc:ml_core.math.stirlingformula}{}

{\renewcommand{\arraystretch}{1.5}
\begin{tabularx}{\textwidth}{|>{\raggedright\arraybackslash}l|X|}
\hline
\hspace{0pt}\mytexttt{\color{red} } & \textbf{StirlingFormula} \\
\hline
\multicolumn{2}{|>{\raggedright\arraybackslash}X|}{\hspace{0pt}\mytexttt{\color{param} (REAL x)}} \\
\hline
\end{tabularx}
}

\par





Stirling's formula






\par
\begin{description}
\item [\colorbox{tagtype}{\color{white} \textbf{\textsf{PARAMETER}}}] \textbf{\underline{x}} ||| REAL8 --- the point of evaluation
\end{description}







\par
\begin{description}
\item [\colorbox{tagtype}{\color{white} \textbf{\textsf{RETURN}}}] \textbf{REAL8} --- evaluation result
\end{description}




\rule{\linewidth}{0.5pt}

\chapter*{\color{headfile}
{\large ML\_Core\slash\hspace{0pt}}
{\large Math\slash\hspace{0pt}}
 \\
upperGamma
}
\hypertarget{ecldoc:toc:ML_Core.Math.upperGamma}{}
\hyperlink{ecldoc:toc:root/ML_Core/Math}{Go Up}


\section*{\underline{\textsf{DESCRIPTIONS}}}
\subsection*{\textsf{\colorbox{headtoc}{\color{white} EMBED}
upperGamma}}

\hypertarget{ecldoc:ml_core.math.uppergamma}{}

{\renewcommand{\arraystretch}{1.5}
\begin{tabularx}{\textwidth}{|>{\raggedright\arraybackslash}l|X|}
\hline
\hspace{0pt}\mytexttt{\color{red} REAL8} & \textbf{upperGamma} \\
\hline
\multicolumn{2}{|>{\raggedright\arraybackslash}X|}{\hspace{0pt}\mytexttt{\color{param} (REAL8 x, REAL8 y)}} \\
\hline
\end{tabularx}
}

\par





Return the upper incomplete gamma value of two real numbers, x and y.






\par
\begin{description}
\item [\colorbox{tagtype}{\color{white} \textbf{\textsf{PARAMETER}}}] \textbf{\underline{y}} ||| REAL8 --- the value of the second number
\item [\colorbox{tagtype}{\color{white} \textbf{\textsf{PARAMETER}}}] \textbf{\underline{x}} ||| REAL8 --- the value of the first number
\end{description}







\par
\begin{description}
\item [\colorbox{tagtype}{\color{white} \textbf{\textsf{RETURN}}}] \textbf{REAL8} --- the upper incomplete gamma value
\end{description}




\rule{\linewidth}{0.5pt}

