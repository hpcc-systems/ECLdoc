\chapter*{\color{headtoc} intest}
\hypertarget{ecldoc:toc:root/intest}{}
\hyperlink{ecldoc:toc:root}{Go Up}


\section*{Table of Contents}
{\renewcommand{\arraystretch}{1.5}
\begin{longtable}{|p{\textwidth}|}
\hline
\hyperlink{ecldoc:toc:intest.example_11}{example\_11.ecl} \\
\hline
\hyperlink{ecldoc:toc:intest.example_2}{example\_2.ecl} \\
Basic Inheritance documentation : mod\_3 inherits both mod\_1 and mod\_2 \\
\hline
\hyperlink{ecldoc:toc:intest.example_3}{example\_3.ecl} \\
Example : Inheritance across files \\
\hline
\hyperlink{ecldoc:toc:intest.example_4}{example\_4.ecl} \\
Example : Inheritance across files \\
\hline
\hyperlink{ecldoc:toc:intest.example_6}{example\_6.ecl} \\
Module Hierarchy Example : mod\_1 -\&gt; mod\_11 -\&gt; mod\_111 \\
\hline
\hyperlink{ecldoc:toc:intest.example_7}{example\_7.ecl} \\
Basic Type Example \\
\hline
\hyperlink{ecldoc:toc:intest.example_8}{example\_8.ecl} \\
Three level Hierarchy Example \\
\hline
\hyperlink{ecldoc:toc:root/intest/in1intest}{in1intest} \\
\hline
\hyperlink{ecldoc:toc:root/intest/inintest}{inintest} \\
\hline
\end{longtable}
}

\chapter*{\color{headfile}
{\large intest\slash\hspace{0pt}}
 \\
example_11
}
\hypertarget{ecldoc:toc:intest.example_11}{}
\hyperlink{ecldoc:toc:root/intest}{Go Up}

\section*{\underline{\textsf{IMPORTS}}}
\begin{doublespace}
{\large
std |
intest |
Example\_3 |
intest.Example\_3 |
intest.inintest |
intest.inintest.Example\_3 |
test |
Inintest |
Inintest.Example\_3 |
}
\end{doublespace}

\section*{\underline{\textsf{DESCRIPTIONS}}}
\subsection*{\textsf{\colorbox{headtoc}{\color{white} MODULE}
example\_11}}

\hypertarget{ecldoc:intest.example_11}{}

{\renewcommand{\arraystretch}{1.5}
\begin{tabularx}{\textwidth}{|>{\raggedright\arraybackslash}l|X|}
\hline
\hspace{0pt}\mytexttt{\color{red} } & \textbf{example\_11} \\
\hline
\end{tabularx}
}

\par





No Documentation Found







\rule{\linewidth}{0.5pt}

\chapter*{\color{headfile}
{\large intest\slash\hspace{0pt}}
{\large inintest\slash\hspace{0pt}}
 \\
example_2
}
\hypertarget{ecldoc:toc:intest.inintest.example_2}{}
\hyperlink{ecldoc:toc:root/intest/inintest}{Go Up}


\section*{\underline{\textsf{DESCRIPTIONS}}}
\subsection*{\textsf{\colorbox{headtoc}{\color{white} MODULE}
example\_2}}

\hypertarget{ecldoc:intest.inintest.example_2}{}

{\renewcommand{\arraystretch}{1.5}
\begin{tabularx}{\textwidth}{|>{\raggedright\arraybackslash}l|X|}
\hline
\hspace{0pt}\mytexttt{\color{red} } & \textbf{example\_2} \\
\hline
\end{tabularx}
}

\par





Basic Inheritance documentation : mod\_3 inherits both mod\_1 and mod\_2 . Inherits v2\_m1, v2\_m2, Overrides v1\_m1, new locals v2\_m3 . Interface Inheritance : mod\_4 inherits interface iface\_1, overrides v1\_i1







\textbf{Children}
\begin{enumerate}
\item \hyperlink{ecldoc:intest.inintest.example_2.rec_1}{rec\_1}
: No Documentation Found
\item \hyperlink{ecldoc:intest.inintest.example_2.rec_2}{rec\_2}
: No Documentation Found
\item \hyperlink{ecldoc:intest.inintest.example_2.rec_3}{rec\_3}
: No Documentation Found
\item \hyperlink{ecldoc:intest.inintest.example_2.mod_1}{mod\_1}
: No Documentation Found
\item \hyperlink{ecldoc:intest.inintest.example_2.mod_2}{mod\_2}
: No Documentation Found
\item \hyperlink{ecldoc:intest.inintest.example_2.mod_3}{mod\_3}
: No Documentation Found
\item \hyperlink{ecldoc:intest.inintest.example_2.iface_1}{iface\_1}
: No Documentation Found
\item \hyperlink{ecldoc:intest.inintest.example_2.mod_4}{mod\_4}
: No Documentation Found
\end{enumerate}

\rule{\linewidth}{0.5pt}

\subsection*{\textsf{\colorbox{headtoc}{\color{white} RECORD}
rec\_1}}

\hypertarget{ecldoc:intest.inintest.example_2.rec_1}{}
\hspace{0pt} \hyperlink{ecldoc:intest.inintest.example_2}{example_2} \textbackslash 

{\renewcommand{\arraystretch}{1.5}
\begin{tabularx}{\textwidth}{|>{\raggedright\arraybackslash}l|X|}
\hline
\hspace{0pt}\mytexttt{\color{red} } & \textbf{rec\_1} \\
\hline
\end{tabularx}
}

\par





No Documentation Found







\par
\begin{description}
\item [\colorbox{tagtype}{\color{white} \textbf{\textsf{FIELD}}}] \textbf{\underline{v1}} ||| REAL8 --- No Doc
\end{description}





\rule{\linewidth}{0.5pt}
\subsection*{\textsf{\colorbox{headtoc}{\color{white} RECORD}
rec\_2}}

\hypertarget{ecldoc:intest.inintest.example_2.rec_2}{}
\hspace{0pt} \hyperlink{ecldoc:intest.inintest.example_2}{example_2} \textbackslash 

{\renewcommand{\arraystretch}{1.5}
\begin{tabularx}{\textwidth}{|>{\raggedright\arraybackslash}l|X|}
\hline
\hspace{0pt}\mytexttt{\color{red} } & \textbf{rec\_2} \\
\hline
\end{tabularx}
}

\par





No Documentation Found







\par
\begin{description}
\item [\colorbox{tagtype}{\color{white} \textbf{\textsf{FIELD}}}] \textbf{\underline{v1}} ||| REAL8 --- No Doc
\item [\colorbox{tagtype}{\color{white} \textbf{\textsf{FIELD}}}] \textbf{\underline{v2}} ||| REAL8 --- No Doc
\end{description}





\rule{\linewidth}{0.5pt}
\subsection*{\textsf{\colorbox{headtoc}{\color{white} RECORD}
rec\_3}}

\hypertarget{ecldoc:intest.inintest.example_2.rec_3}{}
\hspace{0pt} \hyperlink{ecldoc:intest.inintest.example_2}{example_2} \textbackslash 

{\renewcommand{\arraystretch}{1.5}
\begin{tabularx}{\textwidth}{|>{\raggedright\arraybackslash}l|X|}
\hline
\hspace{0pt}\mytexttt{\color{red} } & \textbf{rec\_3} \\
\hline
\end{tabularx}
}

\par





No Documentation Found







\par
\begin{description}
\item [\colorbox{tagtype}{\color{white} \textbf{\textsf{FIELD}}}] \textbf{\underline{v1}} ||| REAL8 --- No Doc
\item [\colorbox{tagtype}{\color{white} \textbf{\textsf{FIELD}}}] \textbf{\underline{v3}} ||| REAL8 --- No Doc
\end{description}





\rule{\linewidth}{0.5pt}
\subsection*{\textsf{\colorbox{headtoc}{\color{white} MODULE}
mod\_1}}

\hypertarget{ecldoc:intest.inintest.example_2.mod_1}{}
\hspace{0pt} \hyperlink{ecldoc:intest.inintest.example_2}{example_2} \textbackslash 

{\renewcommand{\arraystretch}{1.5}
\begin{tabularx}{\textwidth}{|>{\raggedright\arraybackslash}l|X|}
\hline
\hspace{0pt}\mytexttt{\color{red} } & \textbf{mod\_1} \\
\hline
\end{tabularx}
}

\par





No Documentation Found







\textbf{Children}
\begin{enumerate}
\item \hyperlink{ecldoc:intest.inintest.example_2.mod_1.v1_m1}{v1\_m1}
: No Documentation Found
\item \hyperlink{ecldoc:intest.inintest.example_2.mod_1.v2_m1}{v2\_m1}
: No Documentation Found
\end{enumerate}

\rule{\linewidth}{0.5pt}

\subsection*{\textsf{\colorbox{headtoc}{\color{white} ATTRIBUTE}
v1\_m1}}

\hypertarget{ecldoc:intest.inintest.example_2.mod_1.v1_m1}{}
\hspace{0pt} \hyperlink{ecldoc:intest.inintest.example_2}{example_2} \textbackslash 
\hspace{0pt} \hyperlink{ecldoc:intest.inintest.example_2.mod_1}{mod_1} \textbackslash 

{\renewcommand{\arraystretch}{1.5}
\begin{tabularx}{\textwidth}{|>{\raggedright\arraybackslash}l|X|}
\hline
\hspace{0pt}\mytexttt{\color{red} real8} & \textbf{v1\_m1} \\
\hline
\end{tabularx}
}

\par





No Documentation Found








\par
\begin{description}
\item [\colorbox{tagtype}{\color{white} \textbf{\textsf{RETURN}}}] \textbf{REAL8} --- 
\end{description}




\rule{\linewidth}{0.5pt}
\subsection*{\textsf{\colorbox{headtoc}{\color{white} ATTRIBUTE}
v2\_m1}}

\hypertarget{ecldoc:intest.inintest.example_2.mod_1.v2_m1}{}
\hspace{0pt} \hyperlink{ecldoc:intest.inintest.example_2}{example_2} \textbackslash 
\hspace{0pt} \hyperlink{ecldoc:intest.inintest.example_2.mod_1}{mod_1} \textbackslash 

{\renewcommand{\arraystretch}{1.5}
\begin{tabularx}{\textwidth}{|>{\raggedright\arraybackslash}l|X|}
\hline
\hspace{0pt}\mytexttt{\color{red} } & \textbf{v2\_m1} \\
\hline
\end{tabularx}
}

\par





No Documentation Found








\par
\begin{description}
\item [\colorbox{tagtype}{\color{white} \textbf{\textsf{RETURN}}}] \textbf{REAL8} --- 
\end{description}




\rule{\linewidth}{0.5pt}


\subsection*{\textsf{\colorbox{headtoc}{\color{white} MODULE}
mod\_2}}

\hypertarget{ecldoc:intest.inintest.example_2.mod_2}{}
\hspace{0pt} \hyperlink{ecldoc:intest.inintest.example_2}{example_2} \textbackslash 

{\renewcommand{\arraystretch}{1.5}
\begin{tabularx}{\textwidth}{|>{\raggedright\arraybackslash}l|X|}
\hline
\hspace{0pt}\mytexttt{\color{red} } & \textbf{mod\_2} \\
\hline
\end{tabularx}
}

\par





No Documentation Found







\textbf{Children}
\begin{enumerate}
\item \hyperlink{ecldoc:intest.inintest.example_2.mod_2.v1_m1}{v1\_m1}
: No Documentation Found
\item \hyperlink{ecldoc:intest.inintest.example_2.mod_2.v2_m2}{v2\_m2}
: No Documentation Found
\end{enumerate}

\rule{\linewidth}{0.5pt}

\subsection*{\textsf{\colorbox{headtoc}{\color{white} ATTRIBUTE}
v1\_m1}}

\hypertarget{ecldoc:intest.inintest.example_2.mod_2.v1_m1}{}
\hspace{0pt} \hyperlink{ecldoc:intest.inintest.example_2}{example_2} \textbackslash 
\hspace{0pt} \hyperlink{ecldoc:intest.inintest.example_2.mod_2}{mod_2} \textbackslash 

{\renewcommand{\arraystretch}{1.5}
\begin{tabularx}{\textwidth}{|>{\raggedright\arraybackslash}l|X|}
\hline
\hspace{0pt}\mytexttt{\color{red} } & \textbf{v1\_m1} \\
\hline
\end{tabularx}
}

\par





No Documentation Found








\par
\begin{description}
\item [\colorbox{tagtype}{\color{white} \textbf{\textsf{RETURN}}}] \textbf{REAL8} --- 
\end{description}




\rule{\linewidth}{0.5pt}
\subsection*{\textsf{\colorbox{headtoc}{\color{white} ATTRIBUTE}
v2\_m2}}

\hypertarget{ecldoc:intest.inintest.example_2.mod_2.v2_m2}{}
\hspace{0pt} \hyperlink{ecldoc:intest.inintest.example_2}{example_2} \textbackslash 
\hspace{0pt} \hyperlink{ecldoc:intest.inintest.example_2.mod_2}{mod_2} \textbackslash 

{\renewcommand{\arraystretch}{1.5}
\begin{tabularx}{\textwidth}{|>{\raggedright\arraybackslash}l|X|}
\hline
\hspace{0pt}\mytexttt{\color{red} } & \textbf{v2\_m2} \\
\hline
\end{tabularx}
}

\par





No Documentation Found








\par
\begin{description}
\item [\colorbox{tagtype}{\color{white} \textbf{\textsf{RETURN}}}] \textbf{REAL8} --- 
\end{description}




\rule{\linewidth}{0.5pt}


\subsection*{\textsf{\colorbox{headtoc}{\color{white} MODULE}
mod\_3}}

\hypertarget{ecldoc:intest.inintest.example_2.mod_3}{}
\hspace{0pt} \hyperlink{ecldoc:intest.inintest.example_2}{example_2} \textbackslash 

{\renewcommand{\arraystretch}{1.5}
\begin{tabularx}{\textwidth}{|>{\raggedright\arraybackslash}l|X|}
\hline
\hspace{0pt}\mytexttt{\color{red} } & \textbf{mod\_3} \\
\hline
\end{tabularx}
}

\par





No Documentation Found










\par
\begin{description}
\item [\colorbox{tagtype}{\color{white} \textbf{\textsf{PARENT}}}] \textbf{intest.inintest.example\_2.mod\_1} <example\_2.ecl.tex>
\item [\colorbox{tagtype}{\color{white} \textbf{\textsf{PARENT}}}] \textbf{intest.inintest.example\_2.mod\_2} <example\_2.ecl.tex>
\end{description}


\textbf{Children}
\begin{enumerate}
\item \hyperlink{ecldoc:intest.inintest.example_2.mod_1.v2_m1}{v2\_m1}
: No Documentation Found
\item \hyperlink{ecldoc:intest.inintest.example_2.mod_2.v2_m2}{v2\_m2}
: No Documentation Found
\item \hyperlink{ecldoc:intest.inintest.example_2.mod_3.v1_m1}{v1\_m1}
: No Documentation Found
\item \hyperlink{ecldoc:intest.inintest.example_2.mod_3.v2_m3}{v2\_m3}
: No Documentation Found
\end{enumerate}

\rule{\linewidth}{0.5pt}

\subsection*{\textsf{\colorbox{headtoc}{\color{white} ATTRIBUTE}
v2\_m1}}

\hypertarget{ecldoc:intest.inintest.example_2.mod_1.v2_m1}{}
\hspace{0pt} \hyperlink{ecldoc:intest.inintest.example_2}{example_2} \textbackslash 
\hspace{0pt} \hyperlink{ecldoc:intest.inintest.example_2.mod_3}{mod_3} \textbackslash 

{\renewcommand{\arraystretch}{1.5}
\begin{tabularx}{\textwidth}{|>{\raggedright\arraybackslash}l|X|}
\hline
\hspace{0pt}\mytexttt{\color{red} } & \textbf{v2\_m1} \\
\hline
\end{tabularx}
}

\par





No Documentation Found








\par
\begin{description}
\item [\colorbox{tagtype}{\color{white} \textbf{\textsf{RETURN}}}] \textbf{REAL8} --- 
\end{description}






\par
\begin{description}
\item [\colorbox{tagtype}{\color{white} \textbf{\textsf{INHERITED}}}] 
\end{description}



\rule{\linewidth}{0.5pt}
\subsection*{\textsf{\colorbox{headtoc}{\color{white} ATTRIBUTE}
v2\_m2}}

\hypertarget{ecldoc:intest.inintest.example_2.mod_2.v2_m2}{}
\hspace{0pt} \hyperlink{ecldoc:intest.inintest.example_2}{example_2} \textbackslash 
\hspace{0pt} \hyperlink{ecldoc:intest.inintest.example_2.mod_3}{mod_3} \textbackslash 

{\renewcommand{\arraystretch}{1.5}
\begin{tabularx}{\textwidth}{|>{\raggedright\arraybackslash}l|X|}
\hline
\hspace{0pt}\mytexttt{\color{red} } & \textbf{v2\_m2} \\
\hline
\end{tabularx}
}

\par





No Documentation Found








\par
\begin{description}
\item [\colorbox{tagtype}{\color{white} \textbf{\textsf{RETURN}}}] \textbf{REAL8} --- 
\end{description}






\par
\begin{description}
\item [\colorbox{tagtype}{\color{white} \textbf{\textsf{INHERITED}}}] 
\end{description}



\rule{\linewidth}{0.5pt}
\subsection*{\textsf{\colorbox{headtoc}{\color{white} ATTRIBUTE}
v1\_m1}}

\hypertarget{ecldoc:intest.inintest.example_2.mod_3.v1_m1}{}
\hspace{0pt} \hyperlink{ecldoc:intest.inintest.example_2}{example_2} \textbackslash 
\hspace{0pt} \hyperlink{ecldoc:intest.inintest.example_2.mod_3}{mod_3} \textbackslash 

{\renewcommand{\arraystretch}{1.5}
\begin{tabularx}{\textwidth}{|>{\raggedright\arraybackslash}l|X|}
\hline
\hspace{0pt}\mytexttt{\color{red} } & \textbf{v1\_m1} \\
\hline
\end{tabularx}
}

\par





No Documentation Found








\par
\begin{description}
\item [\colorbox{tagtype}{\color{white} \textbf{\textsf{RETURN}}}] \textbf{REAL8} --- 
\end{description}






\par
\begin{description}
\item [\colorbox{tagtype}{\color{white} \textbf{\textsf{OVERRIDE}}}] 
\end{description}



\rule{\linewidth}{0.5pt}
\subsection*{\textsf{\colorbox{headtoc}{\color{white} ATTRIBUTE}
v2\_m3}}

\hypertarget{ecldoc:intest.inintest.example_2.mod_3.v2_m3}{}
\hspace{0pt} \hyperlink{ecldoc:intest.inintest.example_2}{example_2} \textbackslash 
\hspace{0pt} \hyperlink{ecldoc:intest.inintest.example_2.mod_3}{mod_3} \textbackslash 

{\renewcommand{\arraystretch}{1.5}
\begin{tabularx}{\textwidth}{|>{\raggedright\arraybackslash}l|X|}
\hline
\hspace{0pt}\mytexttt{\color{red} } & \textbf{v2\_m3} \\
\hline
\end{tabularx}
}

\par





No Documentation Found








\par
\begin{description}
\item [\colorbox{tagtype}{\color{white} \textbf{\textsf{RETURN}}}] \textbf{REAL8} --- 
\end{description}




\rule{\linewidth}{0.5pt}


\subsection*{\textsf{\colorbox{headtoc}{\color{white} INTERFACE}
iface\_1}}

\hypertarget{ecldoc:intest.inintest.example_2.iface_1}{}
\hspace{0pt} \hyperlink{ecldoc:intest.inintest.example_2}{example_2} \textbackslash 

{\renewcommand{\arraystretch}{1.5}
\begin{tabularx}{\textwidth}{|>{\raggedright\arraybackslash}l|X|}
\hline
\hspace{0pt}\mytexttt{\color{red} } & \textbf{iface\_1} \\
\hline
\end{tabularx}
}

\par





No Documentation Found







\textbf{Children}
\begin{enumerate}
\item \hyperlink{ecldoc:intest.inintest.example_2.iface_1.v1_i1}{v1\_i1}
: No Documentation Found
\end{enumerate}

\rule{\linewidth}{0.5pt}

\subsection*{\textsf{\colorbox{headtoc}{\color{white} ATTRIBUTE}
v1\_i1}}

\hypertarget{ecldoc:intest.inintest.example_2.iface_1.v1_i1}{}
\hspace{0pt} \hyperlink{ecldoc:intest.inintest.example_2}{example_2} \textbackslash 
\hspace{0pt} \hyperlink{ecldoc:intest.inintest.example_2.iface_1}{iface_1} \textbackslash 

{\renewcommand{\arraystretch}{1.5}
\begin{tabularx}{\textwidth}{|>{\raggedright\arraybackslash}l|X|}
\hline
\hspace{0pt}\mytexttt{\color{red} real8} & \textbf{v1\_i1} \\
\hline
\end{tabularx}
}

\par





No Documentation Found








\par
\begin{description}
\item [\colorbox{tagtype}{\color{white} \textbf{\textsf{RETURN}}}] \textbf{REAL8} --- 
\end{description}




\rule{\linewidth}{0.5pt}


\subsection*{\textsf{\colorbox{headtoc}{\color{white} MODULE}
mod\_4}}

\hypertarget{ecldoc:intest.inintest.example_2.mod_4}{}
\hspace{0pt} \hyperlink{ecldoc:intest.inintest.example_2}{example_2} \textbackslash 

{\renewcommand{\arraystretch}{1.5}
\begin{tabularx}{\textwidth}{|>{\raggedright\arraybackslash}l|X|}
\hline
\hspace{0pt}\mytexttt{\color{red} } & \textbf{mod\_4} \\
\hline
\end{tabularx}
}

\par





No Documentation Found










\par
\begin{description}
\item [\colorbox{tagtype}{\color{white} \textbf{\textsf{PARENT}}}] \textbf{intest.inintest.example\_2.iface\_1} <example\_2.ecl.tex>
\end{description}


\textbf{Children}
\begin{enumerate}
\item \hyperlink{ecldoc:intest.inintest.example_2.mod_4.v1_i1}{v1\_i1}
: No Documentation Found
\item \hyperlink{ecldoc:intest.inintest.example_2.mod_4.v2_m4}{v2\_m4}
: No Documentation Found
\end{enumerate}

\rule{\linewidth}{0.5pt}

\subsection*{\textsf{\colorbox{headtoc}{\color{white} ATTRIBUTE}
v1\_i1}}

\hypertarget{ecldoc:intest.inintest.example_2.mod_4.v1_i1}{}
\hspace{0pt} \hyperlink{ecldoc:intest.inintest.example_2}{example_2} \textbackslash 
\hspace{0pt} \hyperlink{ecldoc:intest.inintest.example_2.mod_4}{mod_4} \textbackslash 

{\renewcommand{\arraystretch}{1.5}
\begin{tabularx}{\textwidth}{|>{\raggedright\arraybackslash}l|X|}
\hline
\hspace{0pt}\mytexttt{\color{red} } & \textbf{v1\_i1} \\
\hline
\end{tabularx}
}

\par





No Documentation Found








\par
\begin{description}
\item [\colorbox{tagtype}{\color{white} \textbf{\textsf{RETURN}}}] \textbf{REAL8} --- 
\end{description}






\par
\begin{description}
\item [\colorbox{tagtype}{\color{white} \textbf{\textsf{OVERRIDE}}}] 
\end{description}



\rule{\linewidth}{0.5pt}
\subsection*{\textsf{\colorbox{headtoc}{\color{white} ATTRIBUTE}
v2\_m4}}

\hypertarget{ecldoc:intest.inintest.example_2.mod_4.v2_m4}{}
\hspace{0pt} \hyperlink{ecldoc:intest.inintest.example_2}{example_2} \textbackslash 
\hspace{0pt} \hyperlink{ecldoc:intest.inintest.example_2.mod_4}{mod_4} \textbackslash 

{\renewcommand{\arraystretch}{1.5}
\begin{tabularx}{\textwidth}{|>{\raggedright\arraybackslash}l|X|}
\hline
\hspace{0pt}\mytexttt{\color{red} STRING20} & \textbf{v2\_m4} \\
\hline
\end{tabularx}
}

\par





No Documentation Found








\par
\begin{description}
\item [\colorbox{tagtype}{\color{white} \textbf{\textsf{RETURN}}}] \textbf{STRING20} --- 
\end{description}




\rule{\linewidth}{0.5pt}





\chapter*{\color{headfile}
example_3
}
\hypertarget{ecldoc:toc:example_3}{}
\hyperlink{ecldoc:toc:root}{Go Up}


\section*{\underline{\textsf{DESCRIPTIONS}}}
\subsection*{\textsf{\colorbox{headtoc}{\color{white} MODULE}
Example\_3}}

\hypertarget{ecldoc:Example_3}{}

{\renewcommand{\arraystretch}{1.5}
\begin{tabularx}{\textwidth}{|>{\raggedright\arraybackslash}l|X|}
\hline
\hspace{0pt}\mytexttt{\color{red} } & \textbf{Example\_3} \\
\hline
\end{tabularx}
}

\par





Documentation Testing Multiline Title. link@myspace.com 
\par
 Sentence 1 blablalbla bbblaaaa 


\par
 Sentence 2 


\begin{verbatim}

 blablalbla                    bbbblaaaaa

 bblaaaaaaaaaa
 \end{verbatim}








\par
\begin{description}
\item [\colorbox{tagtype}{\color{white} \textbf{\textsf{PARAMETER}}}] \textbf{\underline{second}} |||  --- okay\_2
\item [\colorbox{tagtype}{\color{white} \textbf{\textsf{PARAMETER}}}] \textbf{\underline{third}} |||  --- okay\_3
\item [\colorbox{tagtype}{\color{white} \textbf{\textsf{PARAMETER}}}] \textbf{\underline{first}} |||  --- okay\_1
\end{description}






\par
\begin{description}
\item [\colorbox{tagtype}{\color{white} \textbf{\textsf{FIELD}}}] \textbf{\underline{f2}} |||  --- oka\_f2
\item [\colorbox{tagtype}{\color{white} \textbf{\textsf{FIELD}}}] \textbf{\underline{f1}} |||  --- oka\_f1
\end{description}






\par
\begin{description}
\item [\colorbox{tagtype}{\color{white} \textbf{\textsf{RETURN}}}] \textbf{} --- rec\_1
\end{description}






\par
\begin{description}
\item [\colorbox{tagtype}{\color{white} \textbf{\textsf{AUTHOR}}}] example\_1.sarthakjain
\end{description}






\par
\begin{description}
\item [\colorbox{tagtype}{\color{white} \textbf{\textsf{SEE}}}] example\_1.mod\_1
\end{description}




\textbf{Children}
\begin{enumerate}
\item \hyperlink{ecldoc:Example_3.mod_1}{mod\_1}
: No Documentation Found
\end{enumerate}

\rule{\linewidth}{0.5pt}

\subsection*{\textsf{\colorbox{headtoc}{\color{white} MODULE}
mod\_1}}

\hypertarget{ecldoc:Example_3.mod_1}{}
\hspace{0pt} \hyperlink{ecldoc:Example_3}{Example_3} \textbackslash 

{\renewcommand{\arraystretch}{1.5}
\begin{tabularx}{\textwidth}{|>{\raggedright\arraybackslash}l|X|}
\hline
\hspace{0pt}\mytexttt{\color{red} } & \textbf{mod\_1} \\
\hline
\end{tabularx}
}

\par





No Documentation Found







\textbf{Children}
\begin{enumerate}
\item \hyperlink{ecldoc:example_3.mod_1.v1_m1}{v1\_m1}
: Doc test 2
\item \hyperlink{ecldoc:example_3.mod_1.v2_m1_ex3}{v2\_m1\_ex3}
: DOC Test 3
\item \hyperlink{ecldoc:example_3.mod_1.abc}{abc}
: No Documentation Found
\item \hyperlink{ecldoc:example_3.mod_1.long_name}{long\_name}
: No Documentation Found
\end{enumerate}

\rule{\linewidth}{0.5pt}

\subsection*{\textsf{\colorbox{headtoc}{\color{white} ATTRIBUTE}
v1\_m1}}

\hypertarget{ecldoc:example_3.mod_1.v1_m1}{}
\hspace{0pt} \hyperlink{ecldoc:Example_3}{Example_3} \textbackslash 
\hspace{0pt} \hyperlink{ecldoc:Example_3.mod_1}{mod_1} \textbackslash 

{\renewcommand{\arraystretch}{1.5}
\begin{tabularx}{\textwidth}{|>{\raggedright\arraybackslash}l|X|}
\hline
\hspace{0pt}\mytexttt{\color{red} } & \textbf{v1\_m1} \\
\hline
\end{tabularx}
}

\par





Doc test 2. Title end by period not newline 
\begin{verbatim}

 ABCD ||||
 CDEF ||||\end{verbatim}










\par
\begin{description}
\item [\colorbox{tagtype}{\color{white} \textbf{\textsf{RETURN}}}] \textbf{REAL8} --- 
\end{description}




\rule{\linewidth}{0.5pt}
\subsection*{\textsf{\colorbox{headtoc}{\color{white} ATTRIBUTE}
v2\_m1\_ex3}}

\hypertarget{ecldoc:example_3.mod_1.v2_m1_ex3}{}
\hspace{0pt} \hyperlink{ecldoc:Example_3}{Example_3} \textbackslash 
\hspace{0pt} \hyperlink{ecldoc:Example_3.mod_1}{mod_1} \textbackslash 

{\renewcommand{\arraystretch}{1.5}
\begin{tabularx}{\textwidth}{|>{\raggedright\arraybackslash}l|X|}
\hline
\hspace{0pt}\mytexttt{\color{red} } & \textbf{v2\_m1\_ex3} \\
\hline
\end{tabularx}
}

\par





DOC Test 3 No Period title








\par
\begin{description}
\item [\colorbox{tagtype}{\color{white} \textbf{\textsf{RETURN}}}] \textbf{REAL8} --- 
\end{description}




\rule{\linewidth}{0.5pt}
\subsection*{\textsf{\colorbox{headtoc}{\color{white} FUNCTION}
abc}}

\hypertarget{ecldoc:example_3.mod_1.abc}{}
\hspace{0pt} \hyperlink{ecldoc:Example_3}{Example_3} \textbackslash 
\hspace{0pt} \hyperlink{ecldoc:Example_3.mod_1}{mod_1} \textbackslash 

{\renewcommand{\arraystretch}{1.5}
\begin{tabularx}{\textwidth}{|>{\raggedright\arraybackslash}l|X|}
\hline
\hspace{0pt}\mytexttt{\color{red} REAL8} & \textbf{abc} \\
\hline
\multicolumn{2}{|>{\raggedright\arraybackslash}X|}{\hspace{0pt}\mytexttt{\color{param} (REAL8 x)}} \\
\hline
\end{tabularx}
}

\par





No Documentation Found






\par
\begin{description}
\item [\colorbox{tagtype}{\color{white} \textbf{\textsf{PARAMETER}}}] \textbf{\underline{x}} ||| REAL8 --- No Doc
\end{description}







\par
\begin{description}
\item [\colorbox{tagtype}{\color{white} \textbf{\textsf{RETURN}}}] \textbf{REAL8} --- 
\end{description}




\rule{\linewidth}{0.5pt}
\subsection*{\textsf{\colorbox{headtoc}{\color{white} FUNCTION}
long\_name}}

\hypertarget{ecldoc:example_3.mod_1.long_name}{}
\hspace{0pt} \hyperlink{ecldoc:Example_3}{Example_3} \textbackslash 
\hspace{0pt} \hyperlink{ecldoc:Example_3.mod_1}{mod_1} \textbackslash 

{\renewcommand{\arraystretch}{1.5}
\begin{tabularx}{\textwidth}{|>{\raggedright\arraybackslash}l|X|}
\hline
\hspace{0pt}\mytexttt{\color{red} } & \textbf{long\_name} \\
\hline
\multicolumn{2}{|>{\raggedright\arraybackslash}X|}{\hspace{0pt}\mytexttt{\color{param} (DATASET(\{REAL8 u\}) X, DATASET(\{REAL8 u\}) IntW, DATASET(\{REAL8 u\}) Intb, REAL8 BETA=0.1, REAL8 sparsityParam=0.1 , REAL8 LAMBDA=0.001, REAL8 ALPHA=0.1, UNSIGNED2 MaxIter=100)}} \\
\hline
\end{tabularx}
}

\par





No Documentation Found






\par
\begin{description}
\item [\colorbox{tagtype}{\color{white} \textbf{\textsf{PARAMETER}}}] \textbf{\underline{x}} ||| TABLE ( \{ REAL8 u \} ) --- No Doc
\item [\colorbox{tagtype}{\color{white} \textbf{\textsf{PARAMETER}}}] \textbf{\underline{intw}} ||| TABLE ( \{ REAL8 u \} ) --- No Doc
\item [\colorbox{tagtype}{\color{white} \textbf{\textsf{PARAMETER}}}] \textbf{\underline{maxiter}} ||| UNSIGNED2 --- No Doc
\item [\colorbox{tagtype}{\color{white} \textbf{\textsf{PARAMETER}}}] \textbf{\underline{alpha}} ||| REAL8 --- No Doc
\item [\colorbox{tagtype}{\color{white} \textbf{\textsf{PARAMETER}}}] \textbf{\underline{lambda}} ||| REAL8 --- No Doc
\item [\colorbox{tagtype}{\color{white} \textbf{\textsf{PARAMETER}}}] \textbf{\underline{intb}} ||| TABLE ( \{ REAL8 u \} ) --- No Doc
\item [\colorbox{tagtype}{\color{white} \textbf{\textsf{PARAMETER}}}] \textbf{\underline{beta}} ||| REAL8 --- No Doc
\item [\colorbox{tagtype}{\color{white} \textbf{\textsf{PARAMETER}}}] \textbf{\underline{sparsityparam}} ||| REAL8 --- No Doc
\end{description}







\par
\begin{description}
\item [\colorbox{tagtype}{\color{white} \textbf{\textsf{RETURN}}}] \textbf{REAL8} --- 
\end{description}




\rule{\linewidth}{0.5pt}





\chapter*{\color{headfile}
{\large intest\slash\hspace{0pt}}
{\large in1intest\slash\hspace{0pt}}
 \\
example_4
}
\hypertarget{ecldoc:toc:intest.in1intest.example_4}{}
\hyperlink{ecldoc:toc:root/intest/in1intest}{Go Up}

\section*{\underline{\textsf{IMPORTS}}}
\begin{doublespace}
{\large
Example\_3.mod\_1 |
}
\end{doublespace}

\section*{\underline{\textsf{DESCRIPTIONS}}}
\subsection*{\textsf{\colorbox{headtoc}{\color{white} MODULE}
example\_4}}

\hypertarget{ecldoc:intest.in1intest.example_4}{}

{\renewcommand{\arraystretch}{1.5}
\begin{tabularx}{\textwidth}{|>{\raggedright\arraybackslash}l|X|}
\hline
\hspace{0pt}\mytexttt{\color{red} } & \textbf{example\_4} \\
\hline
\end{tabularx}
}

\par





Example : Inheritance across files mod\_1 in Example\_4 inherits mod\_1 in Example\_3







\textbf{Children}
\begin{enumerate}
\item \hyperlink{ecldoc:intest.in1intest.example_4.mod_1}{mod\_1}
: No Documentation Found
\end{enumerate}

\rule{\linewidth}{0.5pt}

\subsection*{\textsf{\colorbox{headtoc}{\color{white} MODULE}
mod\_1}}

\hypertarget{ecldoc:intest.in1intest.example_4.mod_1}{}
\hspace{0pt} \hyperlink{ecldoc:intest.in1intest.example_4}{example_4} \textbackslash 

{\renewcommand{\arraystretch}{1.5}
\begin{tabularx}{\textwidth}{|>{\raggedright\arraybackslash}l|X|}
\hline
\hspace{0pt}\mytexttt{\color{red} } & \textbf{mod\_1} \\
\hline
\end{tabularx}
}

\par





No Documentation Found










\par
\begin{description}
\item [\colorbox{tagtype}{\color{white} \textbf{\textsf{PARENT}}}] \textbf{Example\_3.mod\_1} <../../example\_3.ecl.tex>
\end{description}


\textbf{Children}
\begin{enumerate}
\item \hyperlink{ecldoc:intest.in1intest.example_4.mod_1.v2_m1_ex4}{v2\_m1\_ex4}
: No Documentation Found
\item \hyperlink{ecldoc:example_3.mod_1.v1_m1}{v1\_m1}
: Doc test 2
\item \hyperlink{ecldoc:example_3.mod_1.v2_m1_ex3}{v2\_m1\_ex3}
: DOC Test 3
\item \hyperlink{ecldoc:example_3.mod_1.abc}{abc}
: No Documentation Found
\item \hyperlink{ecldoc:example_3.mod_1.long_name}{long\_name}
: No Documentation Found
\end{enumerate}

\rule{\linewidth}{0.5pt}

\subsection*{\textsf{\colorbox{headtoc}{\color{white} ATTRIBUTE}
v2\_m1\_ex4}}

\hypertarget{ecldoc:intest.in1intest.example_4.mod_1.v2_m1_ex4}{}
\hspace{0pt} \hyperlink{ecldoc:intest.in1intest.example_4}{example_4} \textbackslash 
\hspace{0pt} \hyperlink{ecldoc:intest.in1intest.example_4.mod_1}{mod_1} \textbackslash 

{\renewcommand{\arraystretch}{1.5}
\begin{tabularx}{\textwidth}{|>{\raggedright\arraybackslash}l|X|}
\hline
\hspace{0pt}\mytexttt{\color{red} } & \textbf{v2\_m1\_ex4} \\
\hline
\end{tabularx}
}

\par





No Documentation Found








\par
\begin{description}
\item [\colorbox{tagtype}{\color{white} \textbf{\textsf{RETURN}}}] \textbf{REAL8} --- 
\end{description}




\rule{\linewidth}{0.5pt}
\subsection*{\textsf{\colorbox{headtoc}{\color{white} ATTRIBUTE}
v1\_m1}}

\hypertarget{ecldoc:example_3.mod_1.v1_m1}{}
\hspace{0pt} \hyperlink{ecldoc:intest.in1intest.example_4}{example_4} \textbackslash 
\hspace{0pt} \hyperlink{ecldoc:intest.in1intest.example_4.mod_1}{mod_1} \textbackslash 

{\renewcommand{\arraystretch}{1.5}
\begin{tabularx}{\textwidth}{|>{\raggedright\arraybackslash}l|X|}
\hline
\hspace{0pt}\mytexttt{\color{red} } & \textbf{v1\_m1} \\
\hline
\end{tabularx}
}

\par





Doc test 2. Title end by period not newline 
\begin{verbatim}

 ABCD ||||
 CDEF ||||\end{verbatim}










\par
\begin{description}
\item [\colorbox{tagtype}{\color{white} \textbf{\textsf{RETURN}}}] \textbf{REAL8} --- 
\end{description}






\par
\begin{description}
\item [\colorbox{tagtype}{\color{white} \textbf{\textsf{INHERITED}}}] 
\end{description}



\rule{\linewidth}{0.5pt}
\subsection*{\textsf{\colorbox{headtoc}{\color{white} ATTRIBUTE}
v2\_m1\_ex3}}

\hypertarget{ecldoc:example_3.mod_1.v2_m1_ex3}{}
\hspace{0pt} \hyperlink{ecldoc:intest.in1intest.example_4}{example_4} \textbackslash 
\hspace{0pt} \hyperlink{ecldoc:intest.in1intest.example_4.mod_1}{mod_1} \textbackslash 

{\renewcommand{\arraystretch}{1.5}
\begin{tabularx}{\textwidth}{|>{\raggedright\arraybackslash}l|X|}
\hline
\hspace{0pt}\mytexttt{\color{red} } & \textbf{v2\_m1\_ex3} \\
\hline
\end{tabularx}
}

\par





DOC Test 3 No Period title








\par
\begin{description}
\item [\colorbox{tagtype}{\color{white} \textbf{\textsf{RETURN}}}] \textbf{REAL8} --- 
\end{description}






\par
\begin{description}
\item [\colorbox{tagtype}{\color{white} \textbf{\textsf{INHERITED}}}] 
\end{description}



\rule{\linewidth}{0.5pt}
\subsection*{\textsf{\colorbox{headtoc}{\color{white} FUNCTION}
abc}}

\hypertarget{ecldoc:example_3.mod_1.abc}{}
\hspace{0pt} \hyperlink{ecldoc:intest.in1intest.example_4}{example_4} \textbackslash 
\hspace{0pt} \hyperlink{ecldoc:intest.in1intest.example_4.mod_1}{mod_1} \textbackslash 

{\renewcommand{\arraystretch}{1.5}
\begin{tabularx}{\textwidth}{|>{\raggedright\arraybackslash}l|X|}
\hline
\hspace{0pt}\mytexttt{\color{red} REAL8} & \textbf{abc} \\
\hline
\multicolumn{2}{|>{\raggedright\arraybackslash}X|}{\hspace{0pt}\mytexttt{\color{param} (REAL8 x)}} \\
\hline
\end{tabularx}
}

\par





No Documentation Found






\par
\begin{description}
\item [\colorbox{tagtype}{\color{white} \textbf{\textsf{PARAMETER}}}] \textbf{\underline{x}} ||| REAL8 --- No Doc
\end{description}







\par
\begin{description}
\item [\colorbox{tagtype}{\color{white} \textbf{\textsf{RETURN}}}] \textbf{REAL8} --- 
\end{description}






\par
\begin{description}
\item [\colorbox{tagtype}{\color{white} \textbf{\textsf{INHERITED}}}] 
\end{description}



\rule{\linewidth}{0.5pt}
\subsection*{\textsf{\colorbox{headtoc}{\color{white} FUNCTION}
long\_name}}

\hypertarget{ecldoc:example_3.mod_1.long_name}{}
\hspace{0pt} \hyperlink{ecldoc:intest.in1intest.example_4}{example_4} \textbackslash 
\hspace{0pt} \hyperlink{ecldoc:intest.in1intest.example_4.mod_1}{mod_1} \textbackslash 

{\renewcommand{\arraystretch}{1.5}
\begin{tabularx}{\textwidth}{|>{\raggedright\arraybackslash}l|X|}
\hline
\hspace{0pt}\mytexttt{\color{red} } & \textbf{long\_name} \\
\hline
\multicolumn{2}{|>{\raggedright\arraybackslash}X|}{\hspace{0pt}\mytexttt{\color{param} (DATASET(\{REAL8 u\}) X, DATASET(\{REAL8 u\}) IntW, DATASET(\{REAL8 u\}) Intb, REAL8 BETA=0.1, REAL8 sparsityParam=0.1 , REAL8 LAMBDA=0.001, REAL8 ALPHA=0.1, UNSIGNED2 MaxIter=100)}} \\
\hline
\end{tabularx}
}

\par





No Documentation Found






\par
\begin{description}
\item [\colorbox{tagtype}{\color{white} \textbf{\textsf{PARAMETER}}}] \textbf{\underline{x}} ||| TABLE ( \{ REAL8 u \} ) --- No Doc
\item [\colorbox{tagtype}{\color{white} \textbf{\textsf{PARAMETER}}}] \textbf{\underline{intw}} ||| TABLE ( \{ REAL8 u \} ) --- No Doc
\item [\colorbox{tagtype}{\color{white} \textbf{\textsf{PARAMETER}}}] \textbf{\underline{maxiter}} ||| UNSIGNED2 --- No Doc
\item [\colorbox{tagtype}{\color{white} \textbf{\textsf{PARAMETER}}}] \textbf{\underline{alpha}} ||| REAL8 --- No Doc
\item [\colorbox{tagtype}{\color{white} \textbf{\textsf{PARAMETER}}}] \textbf{\underline{lambda}} ||| REAL8 --- No Doc
\item [\colorbox{tagtype}{\color{white} \textbf{\textsf{PARAMETER}}}] \textbf{\underline{intb}} ||| TABLE ( \{ REAL8 u \} ) --- No Doc
\item [\colorbox{tagtype}{\color{white} \textbf{\textsf{PARAMETER}}}] \textbf{\underline{beta}} ||| REAL8 --- No Doc
\item [\colorbox{tagtype}{\color{white} \textbf{\textsf{PARAMETER}}}] \textbf{\underline{sparsityparam}} ||| REAL8 --- No Doc
\end{description}







\par
\begin{description}
\item [\colorbox{tagtype}{\color{white} \textbf{\textsf{RETURN}}}] \textbf{REAL8} --- 
\end{description}






\par
\begin{description}
\item [\colorbox{tagtype}{\color{white} \textbf{\textsf{INHERITED}}}] 
\end{description}



\rule{\linewidth}{0.5pt}





\chapter*{\color{headfile}
{\large intest\slash\hspace{0pt}}
{\large in1intest\slash\hspace{0pt}}
 \\
example_6
}
\hypertarget{ecldoc:toc:intest.in1intest.example_6}{}
\hyperlink{ecldoc:toc:root/intest/in1intest}{Go Up}


\section*{\underline{\textsf{DESCRIPTIONS}}}
\subsection*{\textsf{\colorbox{headtoc}{\color{white} MODULE}
example\_6}}

\hypertarget{ecldoc:intest.in1intest.example_6}{}

{\renewcommand{\arraystretch}{1.5}
\begin{tabularx}{\textwidth}{|>{\raggedright\arraybackslash}l|X|}
\hline
\hspace{0pt}\mytexttt{\color{red} } & \textbf{example\_6} \\
\hline
\end{tabularx}
}

\par





Module Hierarchy Example : mod\_1 -> mod\_11 -> mod\_111 . Inheritance across Hierarchy : mod\_2 inherits mod\_1.mod\_11 , mod\_3.mod\_31 inherits mod\_1.mod\_11 , mod\_4 inherits mod\_3.mod\_31, mod\_2 , mod\_5 inherits mod\_1 and mod\_1.mod\_11







\textbf{Children}
\begin{enumerate}
\item \hyperlink{ecldoc:intest.in1intest.example_6.mod_1}{mod\_1}
: No Documentation Found
\item \hyperlink{ecldoc:intest.in1intest.example_6.mod_2}{mod\_2}
: No Documentation Found
\item \hyperlink{ecldoc:intest.in1intest.example_6.mod_3}{mod\_3}
: No Documentation Found
\item \hyperlink{ecldoc:intest.in1intest.example_6.mod_4}{mod\_4}
: No Documentation Found
\item \hyperlink{ecldoc:intest.in1intest.example_6.mod_5}{mod\_5}
: No Documentation Found
\end{enumerate}

\rule{\linewidth}{0.5pt}

\subsection*{\textsf{\colorbox{headtoc}{\color{white} MODULE}
mod\_1}}

\hypertarget{ecldoc:intest.in1intest.example_6.mod_1}{}
\hspace{0pt} \hyperlink{ecldoc:intest.in1intest.example_6}{example_6} \textbackslash 

{\renewcommand{\arraystretch}{1.5}
\begin{tabularx}{\textwidth}{|>{\raggedright\arraybackslash}l|X|}
\hline
\hspace{0pt}\mytexttt{\color{red} } & \textbf{mod\_1} \\
\hline
\end{tabularx}
}

\par





No Documentation Found







\textbf{Children}
\begin{enumerate}
\item \hyperlink{ecldoc:intest.in1intest.example_6.mod_1.v1_m1}{v1\_m1}
: No Documentation Found
\item \hyperlink{ecldoc:intest.in1intest.example_6.mod_1.mod_11}{mod\_11}
: No Documentation Found
\end{enumerate}

\rule{\linewidth}{0.5pt}

\subsection*{\textsf{\colorbox{headtoc}{\color{white} ATTRIBUTE}
v1\_m1}}

\hypertarget{ecldoc:intest.in1intest.example_6.mod_1.v1_m1}{}
\hspace{0pt} \hyperlink{ecldoc:intest.in1intest.example_6}{example_6} \textbackslash 
\hspace{0pt} \hyperlink{ecldoc:intest.in1intest.example_6.mod_1}{mod_1} \textbackslash 

{\renewcommand{\arraystretch}{1.5}
\begin{tabularx}{\textwidth}{|>{\raggedright\arraybackslash}l|X|}
\hline
\hspace{0pt}\mytexttt{\color{red} } & \textbf{v1\_m1} \\
\hline
\end{tabularx}
}

\par





No Documentation Found








\par
\begin{description}
\item [\colorbox{tagtype}{\color{white} \textbf{\textsf{RETURN}}}] \textbf{REAL8} --- 
\end{description}




\rule{\linewidth}{0.5pt}
\subsection*{\textsf{\colorbox{headtoc}{\color{white} MODULE}
mod\_11}}

\hypertarget{ecldoc:intest.in1intest.example_6.mod_1.mod_11}{}
\hspace{0pt} \hyperlink{ecldoc:intest.in1intest.example_6}{example_6} \textbackslash 
\hspace{0pt} \hyperlink{ecldoc:intest.in1intest.example_6.mod_1}{mod_1} \textbackslash 

{\renewcommand{\arraystretch}{1.5}
\begin{tabularx}{\textwidth}{|>{\raggedright\arraybackslash}l|X|}
\hline
\hspace{0pt}\mytexttt{\color{red} } & \textbf{mod\_11} \\
\hline
\multicolumn{2}{|>{\raggedright\arraybackslash}X|}{\hspace{0pt}\mytexttt{\color{param} (real8 a\_11)}} \\
\hline
\end{tabularx}
}

\par





No Documentation Found






\par
\begin{description}
\item [\colorbox{tagtype}{\color{white} \textbf{\textsf{PARAMETER}}}] \textbf{\underline{a\_11}} ||| REAL8 --- No Doc
\end{description}






\textbf{Children}
\begin{enumerate}
\item \hyperlink{ecldoc:intest.in1intest.example_6.mod_1.mod_11.v1_m11}{v1\_m11}
: No Documentation Found
\item \hyperlink{ecldoc:intest.in1intest.example_6.mod_1.mod_11.mod_111}{mod\_111}
: No Documentation Found
\end{enumerate}

\rule{\linewidth}{0.5pt}

\subsection*{\textsf{\colorbox{headtoc}{\color{white} ATTRIBUTE}
v1\_m11}}

\hypertarget{ecldoc:intest.in1intest.example_6.mod_1.mod_11.v1_m11}{}
\hspace{0pt} \hyperlink{ecldoc:intest.in1intest.example_6}{example_6} \textbackslash 
\hspace{0pt} \hyperlink{ecldoc:intest.in1intest.example_6.mod_1}{mod_1} \textbackslash 
\hspace{0pt} \hyperlink{ecldoc:intest.in1intest.example_6.mod_1.mod_11}{mod_11} \textbackslash 

{\renewcommand{\arraystretch}{1.5}
\begin{tabularx}{\textwidth}{|>{\raggedright\arraybackslash}l|X|}
\hline
\hspace{0pt}\mytexttt{\color{red} } & \textbf{v1\_m11} \\
\hline
\end{tabularx}
}

\par





No Documentation Found








\par
\begin{description}
\item [\colorbox{tagtype}{\color{white} \textbf{\textsf{RETURN}}}] \textbf{REAL8} --- 
\end{description}




\rule{\linewidth}{0.5pt}
\subsection*{\textsf{\colorbox{headtoc}{\color{white} MODULE}
mod\_111}}

\hypertarget{ecldoc:intest.in1intest.example_6.mod_1.mod_11.mod_111}{}
\hspace{0pt} \hyperlink{ecldoc:intest.in1intest.example_6}{example_6} \textbackslash 
\hspace{0pt} \hyperlink{ecldoc:intest.in1intest.example_6.mod_1}{mod_1} \textbackslash 
\hspace{0pt} \hyperlink{ecldoc:intest.in1intest.example_6.mod_1.mod_11}{mod_11} \textbackslash 

{\renewcommand{\arraystretch}{1.5}
\begin{tabularx}{\textwidth}{|>{\raggedright\arraybackslash}l|X|}
\hline
\hspace{0pt}\mytexttt{\color{red} } & \textbf{mod\_111} \\
\hline
\multicolumn{2}{|>{\raggedright\arraybackslash}X|}{\hspace{0pt}\mytexttt{\color{param} (real8 a\_111)}} \\
\hline
\end{tabularx}
}

\par





No Documentation Found






\par
\begin{description}
\item [\colorbox{tagtype}{\color{white} \textbf{\textsf{PARAMETER}}}] \textbf{\underline{a\_111}} ||| REAL8 --- No Doc
\end{description}






\textbf{Children}
\begin{enumerate}
\item \hyperlink{ecldoc:intest.in1intest.example_6.mod_1.mod_11.mod_111.v1_m111}{v1\_m111}
: No Documentation Found
\end{enumerate}

\rule{\linewidth}{0.5pt}

\subsection*{\textsf{\colorbox{headtoc}{\color{white} ATTRIBUTE}
v1\_m111}}

\hypertarget{ecldoc:intest.in1intest.example_6.mod_1.mod_11.mod_111.v1_m111}{}
\hspace{0pt} \hyperlink{ecldoc:intest.in1intest.example_6}{example_6} \textbackslash 
\hspace{0pt} \hyperlink{ecldoc:intest.in1intest.example_6.mod_1}{mod_1} \textbackslash 
\hspace{0pt} \hyperlink{ecldoc:intest.in1intest.example_6.mod_1.mod_11}{mod_11} \textbackslash 
\hspace{0pt} \hyperlink{ecldoc:intest.in1intest.example_6.mod_1.mod_11.mod_111}{mod_111} \textbackslash 

{\renewcommand{\arraystretch}{1.5}
\begin{tabularx}{\textwidth}{|>{\raggedright\arraybackslash}l|X|}
\hline
\hspace{0pt}\mytexttt{\color{red} } & \textbf{v1\_m111} \\
\hline
\end{tabularx}
}

\par





No Documentation Found








\par
\begin{description}
\item [\colorbox{tagtype}{\color{white} \textbf{\textsf{RETURN}}}] \textbf{REAL8} --- 
\end{description}




\rule{\linewidth}{0.5pt}






\subsection*{\textsf{\colorbox{headtoc}{\color{white} MODULE}
mod\_2}}

\hypertarget{ecldoc:intest.in1intest.example_6.mod_2}{}
\hspace{0pt} \hyperlink{ecldoc:intest.in1intest.example_6}{example_6} \textbackslash 

{\renewcommand{\arraystretch}{1.5}
\begin{tabularx}{\textwidth}{|>{\raggedright\arraybackslash}l|X|}
\hline
\hspace{0pt}\mytexttt{\color{red} } & \textbf{mod\_2} \\
\hline
\end{tabularx}
}

\par





No Documentation Found










\par
\begin{description}
\item [\colorbox{tagtype}{\color{white} \textbf{\textsf{PARENT}}}] \textbf{intest.in1intest.example\_6.mod\_1.mod\_11} <example\_6.ecl.tex>
\end{description}


\textbf{Children}
\begin{enumerate}
\item \hyperlink{ecldoc:intest.in1intest.example_6.mod_1.mod_11.v1_m11}{v1\_m11}
: No Documentation Found
\item \hyperlink{ecldoc:intest.in1intest.example_6.mod_1.mod_11.mod_111}{mod\_111}
: No Documentation Found
\item \hyperlink{ecldoc:intest.in1intest.example_6.mod_2.v1_m2}{v1\_m2}
: No Documentation Found
\end{enumerate}

\rule{\linewidth}{0.5pt}

\subsection*{\textsf{\colorbox{headtoc}{\color{white} ATTRIBUTE}
v1\_m11}}

\hypertarget{ecldoc:intest.in1intest.example_6.mod_1.mod_11.v1_m11}{}
\hspace{0pt} \hyperlink{ecldoc:intest.in1intest.example_6}{example_6} \textbackslash 
\hspace{0pt} \hyperlink{ecldoc:intest.in1intest.example_6.mod_2}{mod_2} \textbackslash 

{\renewcommand{\arraystretch}{1.5}
\begin{tabularx}{\textwidth}{|>{\raggedright\arraybackslash}l|X|}
\hline
\hspace{0pt}\mytexttt{\color{red} } & \textbf{v1\_m11} \\
\hline
\end{tabularx}
}

\par





No Documentation Found








\par
\begin{description}
\item [\colorbox{tagtype}{\color{white} \textbf{\textsf{RETURN}}}] \textbf{REAL8} --- 
\end{description}






\par
\begin{description}
\item [\colorbox{tagtype}{\color{white} \textbf{\textsf{INHERITED}}}] 
\end{description}



\rule{\linewidth}{0.5pt}
\subsection*{\textsf{\colorbox{headtoc}{\color{white} MODULE}
mod\_111}}

\hypertarget{ecldoc:intest.in1intest.example_6.mod_1.mod_11.mod_111}{}
\hspace{0pt} \hyperlink{ecldoc:intest.in1intest.example_6}{example_6} \textbackslash 
\hspace{0pt} \hyperlink{ecldoc:intest.in1intest.example_6.mod_2}{mod_2} \textbackslash 

{\renewcommand{\arraystretch}{1.5}
\begin{tabularx}{\textwidth}{|>{\raggedright\arraybackslash}l|X|}
\hline
\hspace{0pt}\mytexttt{\color{red} } & \textbf{mod\_111} \\
\hline
\multicolumn{2}{|>{\raggedright\arraybackslash}X|}{\hspace{0pt}\mytexttt{\color{param} (real8 a\_111)}} \\
\hline
\end{tabularx}
}

\par





No Documentation Found






\par
\begin{description}
\item [\colorbox{tagtype}{\color{white} \textbf{\textsf{PARAMETER}}}] \textbf{\underline{a\_111}} ||| REAL8 --- No Doc
\end{description}








\par
\begin{description}
\item [\colorbox{tagtype}{\color{white} \textbf{\textsf{INHERITED}}}] 
\end{description}



\rule{\linewidth}{0.5pt}
\subsection*{\textsf{\colorbox{headtoc}{\color{white} ATTRIBUTE}
v1\_m2}}

\hypertarget{ecldoc:intest.in1intest.example_6.mod_2.v1_m2}{}
\hspace{0pt} \hyperlink{ecldoc:intest.in1intest.example_6}{example_6} \textbackslash 
\hspace{0pt} \hyperlink{ecldoc:intest.in1intest.example_6.mod_2}{mod_2} \textbackslash 

{\renewcommand{\arraystretch}{1.5}
\begin{tabularx}{\textwidth}{|>{\raggedright\arraybackslash}l|X|}
\hline
\hspace{0pt}\mytexttt{\color{red} } & \textbf{v1\_m2} \\
\hline
\end{tabularx}
}

\par





No Documentation Found








\par
\begin{description}
\item [\colorbox{tagtype}{\color{white} \textbf{\textsf{RETURN}}}] \textbf{REAL8} --- 
\end{description}




\rule{\linewidth}{0.5pt}


\subsection*{\textsf{\colorbox{headtoc}{\color{white} MODULE}
mod\_3}}

\hypertarget{ecldoc:intest.in1intest.example_6.mod_3}{}
\hspace{0pt} \hyperlink{ecldoc:intest.in1intest.example_6}{example_6} \textbackslash 

{\renewcommand{\arraystretch}{1.5}
\begin{tabularx}{\textwidth}{|>{\raggedright\arraybackslash}l|X|}
\hline
\hspace{0pt}\mytexttt{\color{red} } & \textbf{mod\_3} \\
\hline
\end{tabularx}
}

\par





No Documentation Found







\textbf{Children}
\begin{enumerate}
\item \hyperlink{ecldoc:intest.in1intest.example_6.mod_3.v1_m3}{v1\_m3}
: No Documentation Found
\item \hyperlink{ecldoc:intest.in1intest.example_6.mod_3.mod_31}{mod\_31}
: No Documentation Found
\end{enumerate}

\rule{\linewidth}{0.5pt}

\subsection*{\textsf{\colorbox{headtoc}{\color{white} ATTRIBUTE}
v1\_m3}}

\hypertarget{ecldoc:intest.in1intest.example_6.mod_3.v1_m3}{}
\hspace{0pt} \hyperlink{ecldoc:intest.in1intest.example_6}{example_6} \textbackslash 
\hspace{0pt} \hyperlink{ecldoc:intest.in1intest.example_6.mod_3}{mod_3} \textbackslash 

{\renewcommand{\arraystretch}{1.5}
\begin{tabularx}{\textwidth}{|>{\raggedright\arraybackslash}l|X|}
\hline
\hspace{0pt}\mytexttt{\color{red} } & \textbf{v1\_m3} \\
\hline
\end{tabularx}
}

\par





No Documentation Found








\par
\begin{description}
\item [\colorbox{tagtype}{\color{white} \textbf{\textsf{RETURN}}}] \textbf{REAL8} --- 
\end{description}




\rule{\linewidth}{0.5pt}
\subsection*{\textsf{\colorbox{headtoc}{\color{white} MODULE}
mod\_31}}

\hypertarget{ecldoc:intest.in1intest.example_6.mod_3.mod_31}{}
\hspace{0pt} \hyperlink{ecldoc:intest.in1intest.example_6}{example_6} \textbackslash 
\hspace{0pt} \hyperlink{ecldoc:intest.in1intest.example_6.mod_3}{mod_3} \textbackslash 

{\renewcommand{\arraystretch}{1.5}
\begin{tabularx}{\textwidth}{|>{\raggedright\arraybackslash}l|X|}
\hline
\hspace{0pt}\mytexttt{\color{red} } & \textbf{mod\_31} \\
\hline
\end{tabularx}
}

\par





No Documentation Found










\par
\begin{description}
\item [\colorbox{tagtype}{\color{white} \textbf{\textsf{PARENT}}}] \textbf{intest.in1intest.example\_6.mod\_1.mod\_11} <example\_6.ecl.tex>
\end{description}


\textbf{Children}
\begin{enumerate}
\item \hyperlink{ecldoc:intest.in1intest.example_6.mod_1.mod_11.v1_m11}{v1\_m11}
: No Documentation Found
\item \hyperlink{ecldoc:intest.in1intest.example_6.mod_1.mod_11.mod_111}{mod\_111}
: No Documentation Found
\item \hyperlink{ecldoc:intest.in1intest.example_6.mod_3.mod_31.v1_m31}{v1\_m31}
: No Documentation Found
\end{enumerate}

\rule{\linewidth}{0.5pt}

\subsection*{\textsf{\colorbox{headtoc}{\color{white} ATTRIBUTE}
v1\_m11}}

\hypertarget{ecldoc:intest.in1intest.example_6.mod_1.mod_11.v1_m11}{}
\hspace{0pt} \hyperlink{ecldoc:intest.in1intest.example_6}{example_6} \textbackslash 
\hspace{0pt} \hyperlink{ecldoc:intest.in1intest.example_6.mod_3}{mod_3} \textbackslash 
\hspace{0pt} \hyperlink{ecldoc:intest.in1intest.example_6.mod_3.mod_31}{mod_31} \textbackslash 

{\renewcommand{\arraystretch}{1.5}
\begin{tabularx}{\textwidth}{|>{\raggedright\arraybackslash}l|X|}
\hline
\hspace{0pt}\mytexttt{\color{red} } & \textbf{v1\_m11} \\
\hline
\end{tabularx}
}

\par





No Documentation Found








\par
\begin{description}
\item [\colorbox{tagtype}{\color{white} \textbf{\textsf{RETURN}}}] \textbf{REAL8} --- 
\end{description}






\par
\begin{description}
\item [\colorbox{tagtype}{\color{white} \textbf{\textsf{INHERITED}}}] 
\end{description}



\rule{\linewidth}{0.5pt}
\subsection*{\textsf{\colorbox{headtoc}{\color{white} MODULE}
mod\_111}}

\hypertarget{ecldoc:intest.in1intest.example_6.mod_1.mod_11.mod_111}{}
\hspace{0pt} \hyperlink{ecldoc:intest.in1intest.example_6}{example_6} \textbackslash 
\hspace{0pt} \hyperlink{ecldoc:intest.in1intest.example_6.mod_3}{mod_3} \textbackslash 
\hspace{0pt} \hyperlink{ecldoc:intest.in1intest.example_6.mod_3.mod_31}{mod_31} \textbackslash 

{\renewcommand{\arraystretch}{1.5}
\begin{tabularx}{\textwidth}{|>{\raggedright\arraybackslash}l|X|}
\hline
\hspace{0pt}\mytexttt{\color{red} } & \textbf{mod\_111} \\
\hline
\multicolumn{2}{|>{\raggedright\arraybackslash}X|}{\hspace{0pt}\mytexttt{\color{param} (real8 a\_111)}} \\
\hline
\end{tabularx}
}

\par





No Documentation Found






\par
\begin{description}
\item [\colorbox{tagtype}{\color{white} \textbf{\textsf{PARAMETER}}}] \textbf{\underline{a\_111}} ||| REAL8 --- No Doc
\end{description}








\par
\begin{description}
\item [\colorbox{tagtype}{\color{white} \textbf{\textsf{INHERITED}}}] 
\end{description}



\rule{\linewidth}{0.5pt}
\subsection*{\textsf{\colorbox{headtoc}{\color{white} ATTRIBUTE}
v1\_m31}}

\hypertarget{ecldoc:intest.in1intest.example_6.mod_3.mod_31.v1_m31}{}
\hspace{0pt} \hyperlink{ecldoc:intest.in1intest.example_6}{example_6} \textbackslash 
\hspace{0pt} \hyperlink{ecldoc:intest.in1intest.example_6.mod_3}{mod_3} \textbackslash 
\hspace{0pt} \hyperlink{ecldoc:intest.in1intest.example_6.mod_3.mod_31}{mod_31} \textbackslash 

{\renewcommand{\arraystretch}{1.5}
\begin{tabularx}{\textwidth}{|>{\raggedright\arraybackslash}l|X|}
\hline
\hspace{0pt}\mytexttt{\color{red} } & \textbf{v1\_m31} \\
\hline
\end{tabularx}
}

\par





No Documentation Found








\par
\begin{description}
\item [\colorbox{tagtype}{\color{white} \textbf{\textsf{RETURN}}}] \textbf{REAL8} --- 
\end{description}




\rule{\linewidth}{0.5pt}




\subsection*{\textsf{\colorbox{headtoc}{\color{white} MODULE}
mod\_4}}

\hypertarget{ecldoc:intest.in1intest.example_6.mod_4}{}
\hspace{0pt} \hyperlink{ecldoc:intest.in1intest.example_6}{example_6} \textbackslash 

{\renewcommand{\arraystretch}{1.5}
\begin{tabularx}{\textwidth}{|>{\raggedright\arraybackslash}l|X|}
\hline
\hspace{0pt}\mytexttt{\color{red} } & \textbf{mod\_4} \\
\hline
\end{tabularx}
}

\par





No Documentation Found










\par
\begin{description}
\item [\colorbox{tagtype}{\color{white} \textbf{\textsf{PARENT}}}] \textbf{intest.in1intest.example\_6.mod\_3.mod\_31} <example\_6.ecl.tex>
\item [\colorbox{tagtype}{\color{white} \textbf{\textsf{PARENT}}}] \textbf{intest.in1intest.example\_6.mod\_2} <example\_6.ecl.tex>
\end{description}


\textbf{Children}
\begin{enumerate}
\item \hyperlink{ecldoc:intest.in1intest.example_6.mod_1.mod_11.v1_m11}{v1\_m11}
: No Documentation Found
\item \hyperlink{ecldoc:intest.in1intest.example_6.mod_1.mod_11.mod_111}{mod\_111}
: No Documentation Found
\item \hyperlink{ecldoc:intest.in1intest.example_6.mod_2.v1_m2}{v1\_m2}
: No Documentation Found
\item \hyperlink{ecldoc:intest.in1intest.example_6.mod_3.mod_31.v1_m31}{v1\_m31}
: No Documentation Found
\item \hyperlink{ecldoc:intest.in1intest.example_6.mod_4.v1_m4}{v1\_m4}
: No Documentation Found
\end{enumerate}

\rule{\linewidth}{0.5pt}

\subsection*{\textsf{\colorbox{headtoc}{\color{white} ATTRIBUTE}
v1\_m11}}

\hypertarget{ecldoc:intest.in1intest.example_6.mod_1.mod_11.v1_m11}{}
\hspace{0pt} \hyperlink{ecldoc:intest.in1intest.example_6}{example_6} \textbackslash 
\hspace{0pt} \hyperlink{ecldoc:intest.in1intest.example_6.mod_4}{mod_4} \textbackslash 

{\renewcommand{\arraystretch}{1.5}
\begin{tabularx}{\textwidth}{|>{\raggedright\arraybackslash}l|X|}
\hline
\hspace{0pt}\mytexttt{\color{red} } & \textbf{v1\_m11} \\
\hline
\end{tabularx}
}

\par





No Documentation Found








\par
\begin{description}
\item [\colorbox{tagtype}{\color{white} \textbf{\textsf{RETURN}}}] \textbf{REAL8} --- 
\end{description}






\par
\begin{description}
\item [\colorbox{tagtype}{\color{white} \textbf{\textsf{INHERITED}}}] 
\end{description}



\rule{\linewidth}{0.5pt}
\subsection*{\textsf{\colorbox{headtoc}{\color{white} MODULE}
mod\_111}}

\hypertarget{ecldoc:intest.in1intest.example_6.mod_1.mod_11.mod_111}{}
\hspace{0pt} \hyperlink{ecldoc:intest.in1intest.example_6}{example_6} \textbackslash 
\hspace{0pt} \hyperlink{ecldoc:intest.in1intest.example_6.mod_4}{mod_4} \textbackslash 

{\renewcommand{\arraystretch}{1.5}
\begin{tabularx}{\textwidth}{|>{\raggedright\arraybackslash}l|X|}
\hline
\hspace{0pt}\mytexttt{\color{red} } & \textbf{mod\_111} \\
\hline
\multicolumn{2}{|>{\raggedright\arraybackslash}X|}{\hspace{0pt}\mytexttt{\color{param} (real8 a\_111)}} \\
\hline
\end{tabularx}
}

\par





No Documentation Found






\par
\begin{description}
\item [\colorbox{tagtype}{\color{white} \textbf{\textsf{PARAMETER}}}] \textbf{\underline{a\_111}} ||| REAL8 --- No Doc
\end{description}








\par
\begin{description}
\item [\colorbox{tagtype}{\color{white} \textbf{\textsf{INHERITED}}}] 
\end{description}



\rule{\linewidth}{0.5pt}
\subsection*{\textsf{\colorbox{headtoc}{\color{white} ATTRIBUTE}
v1\_m2}}

\hypertarget{ecldoc:intest.in1intest.example_6.mod_2.v1_m2}{}
\hspace{0pt} \hyperlink{ecldoc:intest.in1intest.example_6}{example_6} \textbackslash 
\hspace{0pt} \hyperlink{ecldoc:intest.in1intest.example_6.mod_4}{mod_4} \textbackslash 

{\renewcommand{\arraystretch}{1.5}
\begin{tabularx}{\textwidth}{|>{\raggedright\arraybackslash}l|X|}
\hline
\hspace{0pt}\mytexttt{\color{red} } & \textbf{v1\_m2} \\
\hline
\end{tabularx}
}

\par





No Documentation Found








\par
\begin{description}
\item [\colorbox{tagtype}{\color{white} \textbf{\textsf{RETURN}}}] \textbf{REAL8} --- 
\end{description}






\par
\begin{description}
\item [\colorbox{tagtype}{\color{white} \textbf{\textsf{INHERITED}}}] 
\end{description}



\rule{\linewidth}{0.5pt}
\subsection*{\textsf{\colorbox{headtoc}{\color{white} ATTRIBUTE}
v1\_m31}}

\hypertarget{ecldoc:intest.in1intest.example_6.mod_3.mod_31.v1_m31}{}
\hspace{0pt} \hyperlink{ecldoc:intest.in1intest.example_6}{example_6} \textbackslash 
\hspace{0pt} \hyperlink{ecldoc:intest.in1intest.example_6.mod_4}{mod_4} \textbackslash 

{\renewcommand{\arraystretch}{1.5}
\begin{tabularx}{\textwidth}{|>{\raggedright\arraybackslash}l|X|}
\hline
\hspace{0pt}\mytexttt{\color{red} } & \textbf{v1\_m31} \\
\hline
\end{tabularx}
}

\par





No Documentation Found








\par
\begin{description}
\item [\colorbox{tagtype}{\color{white} \textbf{\textsf{RETURN}}}] \textbf{REAL8} --- 
\end{description}






\par
\begin{description}
\item [\colorbox{tagtype}{\color{white} \textbf{\textsf{INHERITED}}}] 
\end{description}



\rule{\linewidth}{0.5pt}
\subsection*{\textsf{\colorbox{headtoc}{\color{white} ATTRIBUTE}
v1\_m4}}

\hypertarget{ecldoc:intest.in1intest.example_6.mod_4.v1_m4}{}
\hspace{0pt} \hyperlink{ecldoc:intest.in1intest.example_6}{example_6} \textbackslash 
\hspace{0pt} \hyperlink{ecldoc:intest.in1intest.example_6.mod_4}{mod_4} \textbackslash 

{\renewcommand{\arraystretch}{1.5}
\begin{tabularx}{\textwidth}{|>{\raggedright\arraybackslash}l|X|}
\hline
\hspace{0pt}\mytexttt{\color{red} } & \textbf{v1\_m4} \\
\hline
\end{tabularx}
}

\par





No Documentation Found








\par
\begin{description}
\item [\colorbox{tagtype}{\color{white} \textbf{\textsf{RETURN}}}] \textbf{REAL8} --- 
\end{description}




\rule{\linewidth}{0.5pt}


\subsection*{\textsf{\colorbox{headtoc}{\color{white} MODULE}
mod\_5}}

\hypertarget{ecldoc:intest.in1intest.example_6.mod_5}{}
\hspace{0pt} \hyperlink{ecldoc:intest.in1intest.example_6}{example_6} \textbackslash 

{\renewcommand{\arraystretch}{1.5}
\begin{tabularx}{\textwidth}{|>{\raggedright\arraybackslash}l|X|}
\hline
\hspace{0pt}\mytexttt{\color{red} } & \textbf{mod\_5} \\
\hline
\end{tabularx}
}

\par





No Documentation Found










\par
\begin{description}
\item [\colorbox{tagtype}{\color{white} \textbf{\textsf{PARENT}}}] \textbf{intest.in1intest.example\_6.mod\_1} <example\_6.ecl.tex>
\item [\colorbox{tagtype}{\color{white} \textbf{\textsf{PARENT}}}] \textbf{intest.in1intest.example\_6.mod\_1.mod\_11} <example\_6.ecl.tex>
\end{description}


\textbf{Children}
\begin{enumerate}
\item \hyperlink{ecldoc:intest.in1intest.example_6.mod_1.v1_m1}{v1\_m1}
: No Documentation Found
\item \hyperlink{ecldoc:intest.in1intest.example_6.mod_1.mod_11}{mod\_11}
: No Documentation Found
\item \hyperlink{ecldoc:intest.in1intest.example_6.mod_1.mod_11.v1_m11}{v1\_m11}
: No Documentation Found
\item \hyperlink{ecldoc:intest.in1intest.example_6.mod_1.mod_11.mod_111}{mod\_111}
: No Documentation Found
\item \hyperlink{ecldoc:intest.in1intest.example_6.mod_5.v1_m5}{v1\_m5}
: No Documentation Found
\end{enumerate}

\rule{\linewidth}{0.5pt}

\subsection*{\textsf{\colorbox{headtoc}{\color{white} ATTRIBUTE}
v1\_m1}}

\hypertarget{ecldoc:intest.in1intest.example_6.mod_1.v1_m1}{}
\hspace{0pt} \hyperlink{ecldoc:intest.in1intest.example_6}{example_6} \textbackslash 
\hspace{0pt} \hyperlink{ecldoc:intest.in1intest.example_6.mod_5}{mod_5} \textbackslash 

{\renewcommand{\arraystretch}{1.5}
\begin{tabularx}{\textwidth}{|>{\raggedright\arraybackslash}l|X|}
\hline
\hspace{0pt}\mytexttt{\color{red} } & \textbf{v1\_m1} \\
\hline
\end{tabularx}
}

\par





No Documentation Found








\par
\begin{description}
\item [\colorbox{tagtype}{\color{white} \textbf{\textsf{RETURN}}}] \textbf{REAL8} --- 
\end{description}






\par
\begin{description}
\item [\colorbox{tagtype}{\color{white} \textbf{\textsf{INHERITED}}}] 
\end{description}



\rule{\linewidth}{0.5pt}
\subsection*{\textsf{\colorbox{headtoc}{\color{white} MODULE}
mod\_11}}

\hypertarget{ecldoc:intest.in1intest.example_6.mod_1.mod_11}{}
\hspace{0pt} \hyperlink{ecldoc:intest.in1intest.example_6}{example_6} \textbackslash 
\hspace{0pt} \hyperlink{ecldoc:intest.in1intest.example_6.mod_5}{mod_5} \textbackslash 

{\renewcommand{\arraystretch}{1.5}
\begin{tabularx}{\textwidth}{|>{\raggedright\arraybackslash}l|X|}
\hline
\hspace{0pt}\mytexttt{\color{red} } & \textbf{mod\_11} \\
\hline
\multicolumn{2}{|>{\raggedright\arraybackslash}X|}{\hspace{0pt}\mytexttt{\color{param} (real8 a\_11)}} \\
\hline
\end{tabularx}
}

\par





No Documentation Found






\par
\begin{description}
\item [\colorbox{tagtype}{\color{white} \textbf{\textsf{PARAMETER}}}] \textbf{\underline{a\_11}} ||| REAL8 --- No Doc
\end{description}








\par
\begin{description}
\item [\colorbox{tagtype}{\color{white} \textbf{\textsf{INHERITED}}}] 
\end{description}



\rule{\linewidth}{0.5pt}
\subsection*{\textsf{\colorbox{headtoc}{\color{white} ATTRIBUTE}
v1\_m11}}

\hypertarget{ecldoc:intest.in1intest.example_6.mod_1.mod_11.v1_m11}{}
\hspace{0pt} \hyperlink{ecldoc:intest.in1intest.example_6}{example_6} \textbackslash 
\hspace{0pt} \hyperlink{ecldoc:intest.in1intest.example_6.mod_5}{mod_5} \textbackslash 

{\renewcommand{\arraystretch}{1.5}
\begin{tabularx}{\textwidth}{|>{\raggedright\arraybackslash}l|X|}
\hline
\hspace{0pt}\mytexttt{\color{red} } & \textbf{v1\_m11} \\
\hline
\end{tabularx}
}

\par





No Documentation Found








\par
\begin{description}
\item [\colorbox{tagtype}{\color{white} \textbf{\textsf{RETURN}}}] \textbf{REAL8} --- 
\end{description}






\par
\begin{description}
\item [\colorbox{tagtype}{\color{white} \textbf{\textsf{INHERITED}}}] 
\end{description}



\rule{\linewidth}{0.5pt}
\subsection*{\textsf{\colorbox{headtoc}{\color{white} MODULE}
mod\_111}}

\hypertarget{ecldoc:intest.in1intest.example_6.mod_1.mod_11.mod_111}{}
\hspace{0pt} \hyperlink{ecldoc:intest.in1intest.example_6}{example_6} \textbackslash 
\hspace{0pt} \hyperlink{ecldoc:intest.in1intest.example_6.mod_5}{mod_5} \textbackslash 

{\renewcommand{\arraystretch}{1.5}
\begin{tabularx}{\textwidth}{|>{\raggedright\arraybackslash}l|X|}
\hline
\hspace{0pt}\mytexttt{\color{red} } & \textbf{mod\_111} \\
\hline
\multicolumn{2}{|>{\raggedright\arraybackslash}X|}{\hspace{0pt}\mytexttt{\color{param} (real8 a\_111)}} \\
\hline
\end{tabularx}
}

\par





No Documentation Found






\par
\begin{description}
\item [\colorbox{tagtype}{\color{white} \textbf{\textsf{PARAMETER}}}] \textbf{\underline{a\_111}} ||| REAL8 --- No Doc
\end{description}








\par
\begin{description}
\item [\colorbox{tagtype}{\color{white} \textbf{\textsf{INHERITED}}}] 
\end{description}



\rule{\linewidth}{0.5pt}
\subsection*{\textsf{\colorbox{headtoc}{\color{white} ATTRIBUTE}
v1\_m5}}

\hypertarget{ecldoc:intest.in1intest.example_6.mod_5.v1_m5}{}
\hspace{0pt} \hyperlink{ecldoc:intest.in1intest.example_6}{example_6} \textbackslash 
\hspace{0pt} \hyperlink{ecldoc:intest.in1intest.example_6.mod_5}{mod_5} \textbackslash 

{\renewcommand{\arraystretch}{1.5}
\begin{tabularx}{\textwidth}{|>{\raggedright\arraybackslash}l|X|}
\hline
\hspace{0pt}\mytexttt{\color{red} } & \textbf{v1\_m5} \\
\hline
\end{tabularx}
}

\par





No Documentation Found








\par
\begin{description}
\item [\colorbox{tagtype}{\color{white} \textbf{\textsf{RETURN}}}] \textbf{REAL8} --- 
\end{description}




\rule{\linewidth}{0.5pt}





\chapter*{\color{headfile}
{\large intest\slash\hspace{0pt}}
{\large in1intest\slash\hspace{0pt}}
 \\
example_7
}
\hypertarget{ecldoc:toc:intest.in1intest.example_7}{}
\hyperlink{ecldoc:toc:root/intest/in1intest}{Go Up}


\section*{\underline{\textsf{DESCRIPTIONS}}}
\subsection*{\textsf{\colorbox{headtoc}{\color{white} MODULE}
example\_7}}

\hypertarget{ecldoc:intest.in1intest.example_7}{}

{\renewcommand{\arraystretch}{1.5}
\begin{tabularx}{\textwidth}{|>{\raggedright\arraybackslash}l|X|}
\hline
\hspace{0pt}\mytexttt{\color{red} } & \textbf{example\_7} \\
\hline
\end{tabularx}
}

\par





Basic Type Example Source Code copied from ECL Documentation







\textbf{Children}
\begin{enumerate}
\item \hyperlink{ecldoc:intest.in1intest.example_7.r}{R}
: No Documentation Found
\end{enumerate}

\rule{\linewidth}{0.5pt}

\subsection*{\textsf{\colorbox{headtoc}{\color{white} RECORD}
R}}

\hypertarget{ecldoc:intest.in1intest.example_7.r}{}
\hspace{0pt} \hyperlink{ecldoc:intest.in1intest.example_7}{example_7} \textbackslash 

{\renewcommand{\arraystretch}{1.5}
\begin{tabularx}{\textwidth}{|>{\raggedright\arraybackslash}l|X|}
\hline
\hspace{0pt}\mytexttt{\color{red} } & \textbf{R} \\
\hline
\end{tabularx}
}

\par





No Documentation Found







\par
\begin{description}
\item [\colorbox{tagtype}{\color{white} \textbf{\textsf{FIELD}}}] \textbf{\underline{f3}} ||| SCALEINT --- No Doc
\item [\colorbox{tagtype}{\color{white} \textbf{\textsf{FIELD}}}] \textbf{\underline{f2}} ||| NEEDC --- No Doc
\item [\colorbox{tagtype}{\color{white} \textbf{\textsf{FIELD}}}] \textbf{\underline{f1}} ||| REVERSESTRING4 --- No Doc
\end{description}





\rule{\linewidth}{0.5pt}



\chapter*{\color{headfile}
{\large intest\slash\hspace{0pt}}
{\large in1intest\slash\hspace{0pt}}
 \\
example_8
}
\hypertarget{ecldoc:toc:intest.in1intest.example_8}{}
\hyperlink{ecldoc:toc:root/intest/in1intest}{Go Up}


\section*{\underline{\textsf{DESCRIPTIONS}}}
\subsection*{\textsf{\colorbox{headtoc}{\color{white} MODULE}
example\_8}}

\hypertarget{ecldoc:intest.in1intest.example_8}{}

{\renewcommand{\arraystretch}{1.5}
\begin{tabularx}{\textwidth}{|>{\raggedright\arraybackslash}l|X|}
\hline
\hspace{0pt}\mytexttt{\color{red} } & \textbf{example\_8} \\
\hline
\end{tabularx}
}

\par





Three level Hierarchy Example . Inheritance across Hierarchy . Problems with Type System -- PROJECT Expression does not maintain record typename (rec\_2) but do maintain record structure . IE mod\_2.v1\_m2 should be  but shown  .  has same structure as record  .







\textbf{Children}
\begin{enumerate}
\item \hyperlink{ecldoc:intest.in1intest.example_8.mod_1}{mod\_1}
: No Documentation Found
\item \hyperlink{ecldoc:intest.in1intest.example_8.mod_2}{mod\_2}
: No Documentation Found
\end{enumerate}

\rule{\linewidth}{0.5pt}

\subsection*{\textsf{\colorbox{headtoc}{\color{white} MODULE}
mod\_1}}

\hypertarget{ecldoc:intest.in1intest.example_8.mod_1}{}
\hspace{0pt} \hyperlink{ecldoc:intest.in1intest.example_8}{example_8} \textbackslash 

{\renewcommand{\arraystretch}{1.5}
\begin{tabularx}{\textwidth}{|>{\raggedright\arraybackslash}l|X|}
\hline
\hspace{0pt}\mytexttt{\color{red} } & \textbf{mod\_1} \\
\hline
\end{tabularx}
}

\par





No Documentation Found







\textbf{Children}
\begin{enumerate}
\item \hyperlink{ecldoc:intest.in1intest.example_8.mod_1.rec_1}{rec\_1}
: No Documentation Found
\item \hyperlink{ecldoc:intest.in1intest.example_8.mod_1.mod_11}{mod\_11}
: No Documentation Found
\end{enumerate}

\rule{\linewidth}{0.5pt}

\subsection*{\textsf{\colorbox{headtoc}{\color{white} RECORD}
rec\_1}}

\hypertarget{ecldoc:intest.in1intest.example_8.mod_1.rec_1}{}
\hspace{0pt} \hyperlink{ecldoc:intest.in1intest.example_8}{example_8} \textbackslash 
\hspace{0pt} \hyperlink{ecldoc:intest.in1intest.example_8.mod_1}{mod_1} \textbackslash 

{\renewcommand{\arraystretch}{1.5}
\begin{tabularx}{\textwidth}{|>{\raggedright\arraybackslash}l|X|}
\hline
\hspace{0pt}\mytexttt{\color{red} } & \textbf{rec\_1} \\
\hline
\end{tabularx}
}

\par





No Documentation Found







\par
\begin{description}
\item [\colorbox{tagtype}{\color{white} \textbf{\textsf{FIELD}}}] \textbf{\underline{a}} ||| REAL8 --- No Doc
\end{description}





\rule{\linewidth}{0.5pt}
\subsection*{\textsf{\colorbox{headtoc}{\color{white} MODULE}
mod\_11}}

\hypertarget{ecldoc:intest.in1intest.example_8.mod_1.mod_11}{}
\hspace{0pt} \hyperlink{ecldoc:intest.in1intest.example_8}{example_8} \textbackslash 
\hspace{0pt} \hyperlink{ecldoc:intest.in1intest.example_8.mod_1}{mod_1} \textbackslash 

{\renewcommand{\arraystretch}{1.5}
\begin{tabularx}{\textwidth}{|>{\raggedright\arraybackslash}l|X|}
\hline
\hspace{0pt}\mytexttt{\color{red} } & \textbf{mod\_11} \\
\hline
\end{tabularx}
}

\par





No Documentation Found







\textbf{Children}
\begin{enumerate}
\item \hyperlink{ecldoc:intest.in1intest.example_8.mod_1.mod_11.v1_m11}{v1\_m11}
: No Documentation Found
\end{enumerate}

\rule{\linewidth}{0.5pt}

\subsection*{\textsf{\colorbox{headtoc}{\color{white} ATTRIBUTE}
v1\_m11}}

\hypertarget{ecldoc:intest.in1intest.example_8.mod_1.mod_11.v1_m11}{}
\hspace{0pt} \hyperlink{ecldoc:intest.in1intest.example_8}{example_8} \textbackslash 
\hspace{0pt} \hyperlink{ecldoc:intest.in1intest.example_8.mod_1}{mod_1} \textbackslash 
\hspace{0pt} \hyperlink{ecldoc:intest.in1intest.example_8.mod_1.mod_11}{mod_11} \textbackslash 

{\renewcommand{\arraystretch}{1.5}
\begin{tabularx}{\textwidth}{|>{\raggedright\arraybackslash}l|X|}
\hline
\hspace{0pt}\mytexttt{\color{red} } & \textbf{v1\_m11} \\
\hline
\end{tabularx}
}

\par





No Documentation Found








\par
\begin{description}
\item [\colorbox{tagtype}{\color{white} \textbf{\textsf{RETURN}}}] \textbf{TABLE ( rec\_1 )} --- 
\end{description}




\rule{\linewidth}{0.5pt}




\subsection*{\textsf{\colorbox{headtoc}{\color{white} MODULE}
mod\_2}}

\hypertarget{ecldoc:intest.in1intest.example_8.mod_2}{}
\hspace{0pt} \hyperlink{ecldoc:intest.in1intest.example_8}{example_8} \textbackslash 

{\renewcommand{\arraystretch}{1.5}
\begin{tabularx}{\textwidth}{|>{\raggedright\arraybackslash}l|X|}
\hline
\hspace{0pt}\mytexttt{\color{red} } & \textbf{mod\_2} \\
\hline
\end{tabularx}
}

\par





No Documentation Found










\par
\begin{description}
\item [\colorbox{tagtype}{\color{white} \textbf{\textsf{PARENT}}}] \textbf{intest.in1intest.example\_8.mod\_1.mod\_11} <example\_8.ecl.tex>
\end{description}


\textbf{Children}
\begin{enumerate}
\item \hyperlink{ecldoc:intest.in1intest.example_8.mod_1.mod_11.v1_m11}{v1\_m11}
: No Documentation Found
\item \hyperlink{ecldoc:intest.in1intest.example_8.mod_2.rec_2}{rec\_2}
: No Documentation Found
\item \hyperlink{ecldoc:intest.in1intest.example_8.mod_2.v1_m2}{v1\_m2}
: No Documentation Found
\end{enumerate}

\rule{\linewidth}{0.5pt}

\subsection*{\textsf{\colorbox{headtoc}{\color{white} ATTRIBUTE}
v1\_m11}}

\hypertarget{ecldoc:intest.in1intest.example_8.mod_1.mod_11.v1_m11}{}
\hspace{0pt} \hyperlink{ecldoc:intest.in1intest.example_8}{example_8} \textbackslash 
\hspace{0pt} \hyperlink{ecldoc:intest.in1intest.example_8.mod_2}{mod_2} \textbackslash 

{\renewcommand{\arraystretch}{1.5}
\begin{tabularx}{\textwidth}{|>{\raggedright\arraybackslash}l|X|}
\hline
\hspace{0pt}\mytexttt{\color{red} } & \textbf{v1\_m11} \\
\hline
\end{tabularx}
}

\par





No Documentation Found








\par
\begin{description}
\item [\colorbox{tagtype}{\color{white} \textbf{\textsf{RETURN}}}] \textbf{TABLE ( rec\_1 )} --- 
\end{description}






\par
\begin{description}
\item [\colorbox{tagtype}{\color{white} \textbf{\textsf{INHERITED}}}] 
\end{description}



\rule{\linewidth}{0.5pt}
\subsection*{\textsf{\colorbox{headtoc}{\color{white} RECORD}
rec\_2}}

\hypertarget{ecldoc:intest.in1intest.example_8.mod_2.rec_2}{}
\hspace{0pt} \hyperlink{ecldoc:intest.in1intest.example_8}{example_8} \textbackslash 
\hspace{0pt} \hyperlink{ecldoc:intest.in1intest.example_8.mod_2}{mod_2} \textbackslash 

{\renewcommand{\arraystretch}{1.5}
\begin{tabularx}{\textwidth}{|>{\raggedright\arraybackslash}l|X|}
\hline
\hspace{0pt}\mytexttt{\color{red} } & \textbf{rec\_2} \\
\hline
\end{tabularx}
}

\par





No Documentation Found







\par
\begin{description}
\item [\colorbox{tagtype}{\color{white} \textbf{\textsf{FIELD}}}] \textbf{\underline{b}} ||| REAL8 --- No Doc
\end{description}





\rule{\linewidth}{0.5pt}
\subsection*{\textsf{\colorbox{headtoc}{\color{white} FUNCTION}
v1\_m2}}

\hypertarget{ecldoc:intest.in1intest.example_8.mod_2.v1_m2}{}
\hspace{0pt} \hyperlink{ecldoc:intest.in1intest.example_8}{example_8} \textbackslash 
\hspace{0pt} \hyperlink{ecldoc:intest.in1intest.example_8.mod_2}{mod_2} \textbackslash 

{\renewcommand{\arraystretch}{1.5}
\begin{tabularx}{\textwidth}{|>{\raggedright\arraybackslash}l|X|}
\hline
\hspace{0pt}\mytexttt{\color{red} } & \textbf{v1\_m2} \\
\hline
\multicolumn{2}{|>{\raggedright\arraybackslash}X|}{\hspace{0pt}\mytexttt{\color{param} (REAL8 ag\_1)}} \\
\hline
\end{tabularx}
}

\par





No Documentation Found






\par
\begin{description}
\item [\colorbox{tagtype}{\color{white} \textbf{\textsf{PARAMETER}}}] \textbf{\underline{ag\_1}} ||| REAL8 --- No Doc
\end{description}







\par
\begin{description}
\item [\colorbox{tagtype}{\color{white} \textbf{\textsf{RETURN}}}] \textbf{TABLE ( \{ REAL8 b \} )} --- 
\end{description}




\rule{\linewidth}{0.5pt}





\chapter*{\color{headtoc} Math}
\hypertarget{ecldoc:toc:root/ML_Core/Math}{}
\hyperlink{ecldoc:toc:root/ML_Core}{Go Up}


\section*{Table of Contents}
{\renewcommand{\arraystretch}{1.5}
\begin{longtable}{|p{\textwidth}|}
\hline
\hyperlink{ecldoc:toc:ML_Core.Math.Beta}{Beta.ecl} \\
Return the beta value of two positive real numbers, x and y \\
\hline
\hyperlink{ecldoc:toc:ML_Core.Math.Distributions}{Distributions.ecl} \\
\hline
\hyperlink{ecldoc:toc:ML_Core.Math.DoubleFac}{DoubleFac.ecl} \\
The 'double' factorial is defined for ODD n and is the product of all the odd numbers up to and including that number \\
\hline
\hyperlink{ecldoc:toc:ML_Core.Math.Fac}{Fac.ecl} \\
Factorial function \\
\hline
\hyperlink{ecldoc:toc:ML_Core.Math.gamma}{gamma.ecl} \\
Return the value of gamma function of real number x A wrapper for the standard C tgamma function \\
\hline
\hyperlink{ecldoc:toc:ML_Core.Math.log_gamma}{log\_gamma.ecl} \\
Return the value of the log gamma function of the absolute value of X \\
\hline
\hyperlink{ecldoc:toc:ML_Core.Math.lowerGamma}{lowerGamma.ecl} \\
Return the lower incomplete gamma value of two real numbers, \\
\hline
\hyperlink{ecldoc:toc:ML_Core.Math.NCK}{NCK.ecl} \\
\hline
\hyperlink{ecldoc:toc:ML_Core.Math.Poly}{Poly.ecl} \\
Evaluate a polynomial from a set of co-effs \\
\hline
\hyperlink{ecldoc:toc:ML_Core.Math.StirlingFormula}{StirlingFormula.ecl} \\
Stirling's formula \\
\hline
\hyperlink{ecldoc:toc:ML_Core.Math.upperGamma}{upperGamma.ecl} \\
Return the upper incomplete gamma value of two real numbers, x and y \\
\hline
\end{longtable}
}

\chapter*{\color{headfile}
{\large ML\_Core\slash\hspace{0pt}}
{\large Math\slash\hspace{0pt}}
 \\
Beta
}
\hypertarget{ecldoc:toc:ML_Core.Math.Beta}{}
\hyperlink{ecldoc:toc:root/ML_Core/Math}{Go Up}

\section*{\underline{\textsf{IMPORTS}}}
\begin{doublespace}
{\large
ML\_Core.Math |
}
\end{doublespace}

\section*{\underline{\textsf{DESCRIPTIONS}}}
\subsection*{\textsf{\colorbox{headtoc}{\color{white} FUNCTION}
Beta}}

\hypertarget{ecldoc:ml_core.math.beta}{}

{\renewcommand{\arraystretch}{1.5}
\begin{tabularx}{\textwidth}{|>{\raggedright\arraybackslash}l|X|}
\hline
\hspace{0pt}\mytexttt{\color{red} } & \textbf{Beta} \\
\hline
\multicolumn{2}{|>{\raggedright\arraybackslash}X|}{\hspace{0pt}\mytexttt{\color{param} (REAL8 x, REAL8 y)}} \\
\hline
\end{tabularx}
}

\par





Return the beta value of two positive real numbers, x and y






\par
\begin{description}
\item [\colorbox{tagtype}{\color{white} \textbf{\textsf{PARAMETER}}}] \textbf{\underline{y}} ||| REAL8 --- the value of the second number
\item [\colorbox{tagtype}{\color{white} \textbf{\textsf{PARAMETER}}}] \textbf{\underline{x}} ||| REAL8 --- the value of the first number
\end{description}







\par
\begin{description}
\item [\colorbox{tagtype}{\color{white} \textbf{\textsf{RETURN}}}] \textbf{REAL8} --- the beta value
\end{description}




\rule{\linewidth}{0.5pt}

\chapter*{\color{headfile}
{\large LogisticRegression\slash\hspace{0pt}}
 \\
Distributions
}
\hypertarget{ecldoc:toc:LogisticRegression.Distributions}{}
\hyperlink{ecldoc:toc:root/LogisticRegression}{Go Up}

\section*{\underline{\textsf{IMPORTS}}}
\begin{doublespace}
{\large
ML\_Core.Constants |
ML\_Core.Math |
}
\end{doublespace}

\section*{\underline{\textsf{DESCRIPTIONS}}}
\subsection*{\textsf{\colorbox{headtoc}{\color{white} MODULE}
Distributions}}

\hypertarget{ecldoc:LogisticRegression.Distributions}{}

{\renewcommand{\arraystretch}{1.5}
\begin{tabularx}{\textwidth}{|>{\raggedright\arraybackslash}l|X|}
\hline
\hspace{0pt}\mytexttt{\color{red} } & \textbf{Distributions} \\
\hline
\end{tabularx}
}

\par





No Documentation Found







\textbf{Children}
\begin{enumerate}
\item \hyperlink{ecldoc:logisticregression.distributions.normal_cdf}{Normal\_CDF}
: Cumulative Distribution of the standard normal distribution, the probability that a normal random variable will be smaller than x standard deviations above or below the mean
\item \hyperlink{ecldoc:logisticregression.distributions.normal_ppf}{Normal\_PPF}
: Normal Distribution Percentage Point Function
\item \hyperlink{ecldoc:logisticregression.distributions.t_cdf}{T\_CDF}
: Students t distribution integral evaluated between negative infinity and x
\item \hyperlink{ecldoc:logisticregression.distributions.t_ppf}{T\_PPF}
: Percentage point function for the T distribution
\item \hyperlink{ecldoc:logisticregression.distributions.chi2_cdf}{Chi2\_CDF}
: The cumulative distribution function for the Chi Square distribution
\item \hyperlink{ecldoc:logisticregression.distributions.chi2_ppf}{Chi2\_PPF}
: The Chi Squared PPF function
\end{enumerate}

\rule{\linewidth}{0.5pt}

\subsection*{\textsf{\colorbox{headtoc}{\color{white} FUNCTION}
Normal\_CDF}}

\hypertarget{ecldoc:logisticregression.distributions.normal_cdf}{}
\hspace{0pt} \hyperlink{ecldoc:LogisticRegression.Distributions}{Distributions} \textbackslash 

{\renewcommand{\arraystretch}{1.5}
\begin{tabularx}{\textwidth}{|>{\raggedright\arraybackslash}l|X|}
\hline
\hspace{0pt}\mytexttt{\color{red} REAL8} & \textbf{Normal\_CDF} \\
\hline
\multicolumn{2}{|>{\raggedright\arraybackslash}X|}{\hspace{0pt}\mytexttt{\color{param} (REAL8 x)}} \\
\hline
\end{tabularx}
}

\par





Cumulative Distribution of the standard normal distribution, the probability that a normal random variable will be smaller than x standard deviations above or below the mean. Taken from C/C++ Mathematical Algorithms for Scientists and Engineers, n. Shammas, McGraw-Hill, 1995






\par
\begin{description}
\item [\colorbox{tagtype}{\color{white} \textbf{\textsf{PARAMETER}}}] \textbf{\underline{x}} ||| REAL8 --- the number of standard deviations
\end{description}







\par
\begin{description}
\item [\colorbox{tagtype}{\color{white} \textbf{\textsf{RETURN}}}] \textbf{REAL8} --- 
\end{description}






\par
\begin{description}
\item [\colorbox{tagtype}{\color{white} \textbf{\textsf{RETURNS}}}] probability of exceeding x.
\end{description}




\rule{\linewidth}{0.5pt}
\subsection*{\textsf{\colorbox{headtoc}{\color{white} FUNCTION}
Normal\_PPF}}

\hypertarget{ecldoc:logisticregression.distributions.normal_ppf}{}
\hspace{0pt} \hyperlink{ecldoc:LogisticRegression.Distributions}{Distributions} \textbackslash 

{\renewcommand{\arraystretch}{1.5}
\begin{tabularx}{\textwidth}{|>{\raggedright\arraybackslash}l|X|}
\hline
\hspace{0pt}\mytexttt{\color{red} REAL8} & \textbf{Normal\_PPF} \\
\hline
\multicolumn{2}{|>{\raggedright\arraybackslash}X|}{\hspace{0pt}\mytexttt{\color{param} (REAL8 x)}} \\
\hline
\end{tabularx}
}

\par





Normal Distribution Percentage Point Function. Translated from C/C++ Mathematical Algorithms for Scientists and Engineers, N. Shammas, McGraw-Hill, 1995






\par
\begin{description}
\item [\colorbox{tagtype}{\color{white} \textbf{\textsf{PARAMETER}}}] \textbf{\underline{x}} ||| REAL8 --- probability
\end{description}







\par
\begin{description}
\item [\colorbox{tagtype}{\color{white} \textbf{\textsf{RETURN}}}] \textbf{REAL8} --- 
\end{description}






\par
\begin{description}
\item [\colorbox{tagtype}{\color{white} \textbf{\textsf{RETURNS}}}] number of standard deviations from the mean
\end{description}




\rule{\linewidth}{0.5pt}
\subsection*{\textsf{\colorbox{headtoc}{\color{white} FUNCTION}
T\_CDF}}

\hypertarget{ecldoc:logisticregression.distributions.t_cdf}{}
\hspace{0pt} \hyperlink{ecldoc:LogisticRegression.Distributions}{Distributions} \textbackslash 

{\renewcommand{\arraystretch}{1.5}
\begin{tabularx}{\textwidth}{|>{\raggedright\arraybackslash}l|X|}
\hline
\hspace{0pt}\mytexttt{\color{red} REAL8} & \textbf{T\_CDF} \\
\hline
\multicolumn{2}{|>{\raggedright\arraybackslash}X|}{\hspace{0pt}\mytexttt{\color{param} (REAL8 x, REAL8 df)}} \\
\hline
\end{tabularx}
}

\par





Students t distribution integral evaluated between negative infinity and x. Translated from NIST SEL DATAPAC Fortran TCDF.f source






\par
\begin{description}
\item [\colorbox{tagtype}{\color{white} \textbf{\textsf{PARAMETER}}}] \textbf{\underline{df}} ||| REAL8 --- degrees of freedom
\item [\colorbox{tagtype}{\color{white} \textbf{\textsf{PARAMETER}}}] \textbf{\underline{x}} ||| REAL8 --- value of the evaluation
\end{description}







\par
\begin{description}
\item [\colorbox{tagtype}{\color{white} \textbf{\textsf{RETURN}}}] \textbf{REAL8} --- 
\end{description}






\par
\begin{description}
\item [\colorbox{tagtype}{\color{white} \textbf{\textsf{RETURNS}}}] the probability that a value will be less than the specified value
\end{description}




\rule{\linewidth}{0.5pt}
\subsection*{\textsf{\colorbox{headtoc}{\color{white} FUNCTION}
T\_PPF}}

\hypertarget{ecldoc:logisticregression.distributions.t_ppf}{}
\hspace{0pt} \hyperlink{ecldoc:LogisticRegression.Distributions}{Distributions} \textbackslash 

{\renewcommand{\arraystretch}{1.5}
\begin{tabularx}{\textwidth}{|>{\raggedright\arraybackslash}l|X|}
\hline
\hspace{0pt}\mytexttt{\color{red} REAL8} & \textbf{T\_PPF} \\
\hline
\multicolumn{2}{|>{\raggedright\arraybackslash}X|}{\hspace{0pt}\mytexttt{\color{param} (REAL8 x, REAL8 df)}} \\
\hline
\end{tabularx}
}

\par





Percentage point function for the T distribution. Translated from NIST SEL DATAPAC Fortran TPPF.f source






\par
\begin{description}
\item [\colorbox{tagtype}{\color{white} \textbf{\textsf{PARAMETER}}}] \textbf{\underline{df}} ||| REAL8 --- No Doc
\item [\colorbox{tagtype}{\color{white} \textbf{\textsf{PARAMETER}}}] \textbf{\underline{x}} ||| REAL8 --- No Doc
\end{description}







\par
\begin{description}
\item [\colorbox{tagtype}{\color{white} \textbf{\textsf{RETURN}}}] \textbf{REAL8} --- 
\end{description}




\rule{\linewidth}{0.5pt}
\subsection*{\textsf{\colorbox{headtoc}{\color{white} FUNCTION}
Chi2\_CDF}}

\hypertarget{ecldoc:logisticregression.distributions.chi2_cdf}{}
\hspace{0pt} \hyperlink{ecldoc:LogisticRegression.Distributions}{Distributions} \textbackslash 

{\renewcommand{\arraystretch}{1.5}
\begin{tabularx}{\textwidth}{|>{\raggedright\arraybackslash}l|X|}
\hline
\hspace{0pt}\mytexttt{\color{red} REAL8} & \textbf{Chi2\_CDF} \\
\hline
\multicolumn{2}{|>{\raggedright\arraybackslash}X|}{\hspace{0pt}\mytexttt{\color{param} (REAL8 x, REAL8 df)}} \\
\hline
\end{tabularx}
}

\par





The cumulative distribution function for the Chi Square distribution. the CDF for the specfied degrees of freedom. Translated from the NIST SEL DATAPAC Fortran subroutine CHSCDF.






\par
\begin{description}
\item [\colorbox{tagtype}{\color{white} \textbf{\textsf{PARAMETER}}}] \textbf{\underline{df}} ||| REAL8 --- No Doc
\item [\colorbox{tagtype}{\color{white} \textbf{\textsf{PARAMETER}}}] \textbf{\underline{x}} ||| REAL8 --- No Doc
\end{description}







\par
\begin{description}
\item [\colorbox{tagtype}{\color{white} \textbf{\textsf{RETURN}}}] \textbf{REAL8} --- 
\end{description}




\rule{\linewidth}{0.5pt}
\subsection*{\textsf{\colorbox{headtoc}{\color{white} FUNCTION}
Chi2\_PPF}}

\hypertarget{ecldoc:logisticregression.distributions.chi2_ppf}{}
\hspace{0pt} \hyperlink{ecldoc:LogisticRegression.Distributions}{Distributions} \textbackslash 

{\renewcommand{\arraystretch}{1.5}
\begin{tabularx}{\textwidth}{|>{\raggedright\arraybackslash}l|X|}
\hline
\hspace{0pt}\mytexttt{\color{red} REAL8} & \textbf{Chi2\_PPF} \\
\hline
\multicolumn{2}{|>{\raggedright\arraybackslash}X|}{\hspace{0pt}\mytexttt{\color{param} (REAL8 x, REAL8 df)}} \\
\hline
\end{tabularx}
}

\par





The Chi Squared PPF function. Translated from the NIST SEL DATAPAC Fortran subroutine CHSPPF.






\par
\begin{description}
\item [\colorbox{tagtype}{\color{white} \textbf{\textsf{PARAMETER}}}] \textbf{\underline{df}} ||| REAL8 --- No Doc
\item [\colorbox{tagtype}{\color{white} \textbf{\textsf{PARAMETER}}}] \textbf{\underline{x}} ||| REAL8 --- No Doc
\end{description}







\par
\begin{description}
\item [\colorbox{tagtype}{\color{white} \textbf{\textsf{RETURN}}}] \textbf{REAL8} --- 
\end{description}




\rule{\linewidth}{0.5pt}



\chapter*{\color{headfile}
{\large ML\_Core\slash\hspace{0pt}}
{\large Math\slash\hspace{0pt}}
 \\
DoubleFac
}
\hypertarget{ecldoc:toc:ML_Core.Math.DoubleFac}{}
\hyperlink{ecldoc:toc:root/ML_Core/Math}{Go Up}


\section*{\underline{\textsf{DESCRIPTIONS}}}
\subsection*{\textsf{\colorbox{headtoc}{\color{white} EMBED}
DoubleFac}}

\hypertarget{ecldoc:ml_core.math.doublefac}{}

{\renewcommand{\arraystretch}{1.5}
\begin{tabularx}{\textwidth}{|>{\raggedright\arraybackslash}l|X|}
\hline
\hspace{0pt}\mytexttt{\color{red} REAL8} & \textbf{DoubleFac} \\
\hline
\multicolumn{2}{|>{\raggedright\arraybackslash}X|}{\hspace{0pt}\mytexttt{\color{param} (INTEGER2 i)}} \\
\hline
\end{tabularx}
}

\par





The 'double' factorial is defined for ODD n and is the product of all the odd numbers up to and including that number. We are extending the meaning to even numbers to mean the product of the even numbers up to and including that number. Thus DoubleFac(8) = 8*6*4*2 We also defend against i < 2 (returning 1.0)






\par
\begin{description}
\item [\colorbox{tagtype}{\color{white} \textbf{\textsf{PARAMETER}}}] \textbf{\underline{i}} ||| INTEGER2 --- the value used in the calculation
\end{description}







\par
\begin{description}
\item [\colorbox{tagtype}{\color{white} \textbf{\textsf{RETURN}}}] \textbf{REAL8} --- the factorial of the sequence, declining by 2
\end{description}




\rule{\linewidth}{0.5pt}

\chapter*{\color{headfile}
{\large ML\_Core\slash\hspace{0pt}}
{\large Math\slash\hspace{0pt}}
 \\
Fac
}
\hypertarget{ecldoc:toc:ML_Core.Math.Fac}{}
\hyperlink{ecldoc:toc:root/ML_Core/Math}{Go Up}


\section*{\underline{\textsf{DESCRIPTIONS}}}
\subsection*{\textsf{\colorbox{headtoc}{\color{white} EMBED}
Fac}}

\hypertarget{ecldoc:ml_core.math.fac}{}

{\renewcommand{\arraystretch}{1.5}
\begin{tabularx}{\textwidth}{|>{\raggedright\arraybackslash}l|X|}
\hline
\hspace{0pt}\mytexttt{\color{red} REAL8} & \textbf{Fac} \\
\hline
\multicolumn{2}{|>{\raggedright\arraybackslash}X|}{\hspace{0pt}\mytexttt{\color{param} (UNSIGNED2 i)}} \\
\hline
\end{tabularx}
}

\par





Factorial function






\par
\begin{description}
\item [\colorbox{tagtype}{\color{white} \textbf{\textsf{PARAMETER}}}] \textbf{\underline{i}} ||| UNSIGNED2 --- the value used, (i)(i-1)(i-2)\ldots(2)
\end{description}







\par
\begin{description}
\item [\colorbox{tagtype}{\color{white} \textbf{\textsf{RETURN}}}] \textbf{REAL8} --- the factorial i!
\end{description}




\rule{\linewidth}{0.5pt}

\chapter*{\color{headfile}
{\large ML\_Core\slash\hspace{0pt}}
{\large Math\slash\hspace{0pt}}
 \\
gamma
}
\hypertarget{ecldoc:toc:ML_Core.Math.gamma}{}
\hyperlink{ecldoc:toc:root/ML_Core/Math}{Go Up}


\section*{\underline{\textsf{DESCRIPTIONS}}}
\subsection*{\textsf{\colorbox{headtoc}{\color{white} EMBED}
gamma}}

\hypertarget{ecldoc:ml_core.math.gamma}{}

{\renewcommand{\arraystretch}{1.5}
\begin{tabularx}{\textwidth}{|>{\raggedright\arraybackslash}l|X|}
\hline
\hspace{0pt}\mytexttt{\color{red} REAL8} & \textbf{gamma} \\
\hline
\multicolumn{2}{|>{\raggedright\arraybackslash}X|}{\hspace{0pt}\mytexttt{\color{param} (REAL8 x)}} \\
\hline
\end{tabularx}
}

\par





Return the value of gamma function of real number x A wrapper for the standard C tgamma function.






\par
\begin{description}
\item [\colorbox{tagtype}{\color{white} \textbf{\textsf{PARAMETER}}}] \textbf{\underline{x}} ||| REAL8 --- the input x
\end{description}







\par
\begin{description}
\item [\colorbox{tagtype}{\color{white} \textbf{\textsf{RETURN}}}] \textbf{REAL8} --- the value of GAMMA evaluated at x
\end{description}




\rule{\linewidth}{0.5pt}

\chapter*{\color{headfile}
{\large ML\_Core\slash\hspace{0pt}}
{\large Math\slash\hspace{0pt}}
 \\
log_gamma
}
\hypertarget{ecldoc:toc:ML_Core.Math.log_gamma}{}
\hyperlink{ecldoc:toc:root/ML_Core/Math}{Go Up}


\section*{\underline{\textsf{DESCRIPTIONS}}}
\subsection*{\textsf{\colorbox{headtoc}{\color{white} EMBED}
log\_gamma}}

\hypertarget{ecldoc:ml_core.math.log_gamma}{}

{\renewcommand{\arraystretch}{1.5}
\begin{tabularx}{\textwidth}{|>{\raggedright\arraybackslash}l|X|}
\hline
\hspace{0pt}\mytexttt{\color{red} REAL8} & \textbf{log\_gamma} \\
\hline
\multicolumn{2}{|>{\raggedright\arraybackslash}X|}{\hspace{0pt}\mytexttt{\color{param} (REAL8 x)}} \\
\hline
\end{tabularx}
}

\par





Return the value of the log gamma function of the absolute value of X. A wrapper for the standard C lgamma function. Avoids the race condition found on some platforms by taking the absolute value of the of the input argument.






\par
\begin{description}
\item [\colorbox{tagtype}{\color{white} \textbf{\textsf{PARAMETER}}}] \textbf{\underline{x}} ||| REAL8 --- the input x
\end{description}







\par
\begin{description}
\item [\colorbox{tagtype}{\color{white} \textbf{\textsf{RETURN}}}] \textbf{REAL8} --- the value of the log of the GAMMA evaluated at ABS(x)
\end{description}




\rule{\linewidth}{0.5pt}

\chapter*{\color{headfile}
{\large ML\_Core\slash\hspace{0pt}}
{\large Math\slash\hspace{0pt}}
 \\
lowerGamma
}
\hypertarget{ecldoc:toc:ML_Core.Math.lowerGamma}{}
\hyperlink{ecldoc:toc:root/ML_Core/Math}{Go Up}


\section*{\underline{\textsf{DESCRIPTIONS}}}
\subsection*{\textsf{\colorbox{headtoc}{\color{white} EMBED}
lowerGamma}}

\hypertarget{ecldoc:ml_core.math.lowergamma}{}

{\renewcommand{\arraystretch}{1.5}
\begin{tabularx}{\textwidth}{|>{\raggedright\arraybackslash}l|X|}
\hline
\hspace{0pt}\mytexttt{\color{red} REAL8} & \textbf{lowerGamma} \\
\hline
\multicolumn{2}{|>{\raggedright\arraybackslash}X|}{\hspace{0pt}\mytexttt{\color{param} (REAL8 x, REAL8 y)}} \\
\hline
\end{tabularx}
}

\par





Return the lower incomplete gamma value of two real numbers, x and y






\par
\begin{description}
\item [\colorbox{tagtype}{\color{white} \textbf{\textsf{PARAMETER}}}] \textbf{\underline{y}} ||| REAL8 --- the value of the second number
\item [\colorbox{tagtype}{\color{white} \textbf{\textsf{PARAMETER}}}] \textbf{\underline{x}} ||| REAL8 --- the value of the first number
\end{description}







\par
\begin{description}
\item [\colorbox{tagtype}{\color{white} \textbf{\textsf{RETURN}}}] \textbf{REAL8} --- the lower incomplete gamma value
\end{description}




\rule{\linewidth}{0.5pt}

\chapter*{\color{headfile}
{\large ML\_Core\slash\hspace{0pt}}
{\large Math\slash\hspace{0pt}}
 \\
NCK
}
\hypertarget{ecldoc:toc:ML_Core.Math.NCK}{}
\hyperlink{ecldoc:toc:root/ML_Core/Math}{Go Up}

\section*{\underline{\textsf{IMPORTS}}}
\begin{doublespace}
{\large
ML\_Core.Math |
}
\end{doublespace}

\section*{\underline{\textsf{DESCRIPTIONS}}}
\subsection*{\textsf{\colorbox{headtoc}{\color{white} FUNCTION}
NCK}}

\hypertarget{ecldoc:ml_core.math.nck}{}

{\renewcommand{\arraystretch}{1.5}
\begin{tabularx}{\textwidth}{|>{\raggedright\arraybackslash}l|X|}
\hline
\hspace{0pt}\mytexttt{\color{red} REAL8} & \textbf{NCK} \\
\hline
\multicolumn{2}{|>{\raggedright\arraybackslash}X|}{\hspace{0pt}\mytexttt{\color{param} (INTEGER2 N, INTEGER2 K)}} \\
\hline
\end{tabularx}
}

\par





No Documentation Found






\par
\begin{description}
\item [\colorbox{tagtype}{\color{white} \textbf{\textsf{PARAMETER}}}] \textbf{\underline{k}} ||| INTEGER2 --- No Doc
\item [\colorbox{tagtype}{\color{white} \textbf{\textsf{PARAMETER}}}] \textbf{\underline{n}} ||| INTEGER2 --- No Doc
\end{description}







\par
\begin{description}
\item [\colorbox{tagtype}{\color{white} \textbf{\textsf{RETURN}}}] \textbf{REAL8} --- 
\end{description}




\rule{\linewidth}{0.5pt}

\chapter*{\color{headfile}
{\large ML\_Core\slash\hspace{0pt}}
{\large Math\slash\hspace{0pt}}
 \\
Poly
}
\hypertarget{ecldoc:toc:ML_Core.Math.Poly}{}
\hyperlink{ecldoc:toc:root/ML_Core/Math}{Go Up}


\section*{\underline{\textsf{DESCRIPTIONS}}}
\subsection*{\textsf{\colorbox{headtoc}{\color{white} EMBED}
Poly}}

\hypertarget{ecldoc:ml_core.math.poly}{}

{\renewcommand{\arraystretch}{1.5}
\begin{tabularx}{\textwidth}{|>{\raggedright\arraybackslash}l|X|}
\hline
\hspace{0pt}\mytexttt{\color{red} REAL8} & \textbf{Poly} \\
\hline
\multicolumn{2}{|>{\raggedright\arraybackslash}X|}{\hspace{0pt}\mytexttt{\color{param} (REAL8 x, SET OF REAL8 Coeffs)}} \\
\hline
\end{tabularx}
}

\par





Evaluate a polynomial from a set of co-effs. Co-effs 1 is assumed to be the HIGH order of the equation. Thus for ax\^{}2+bx+c - the set would need to be Coef := [a,b,c];






\par
\begin{description}
\item [\colorbox{tagtype}{\color{white} \textbf{\textsf{PARAMETER}}}] \textbf{\underline{Coeffs}} ||| SET ( REAL8 ) --- a set of coefficients forthe polynomial. The ALL set is considered to be all zero values
\item [\colorbox{tagtype}{\color{white} \textbf{\textsf{PARAMETER}}}] \textbf{\underline{x}} ||| REAL8 --- the value of x in the polynomial
\end{description}







\par
\begin{description}
\item [\colorbox{tagtype}{\color{white} \textbf{\textsf{RETURN}}}] \textbf{REAL8} --- value of the polynomial at x
\end{description}




\rule{\linewidth}{0.5pt}

\chapter*{\color{headfile}
{\large ML\_Core\slash\hspace{0pt}}
{\large Math\slash\hspace{0pt}}
 \\
StirlingFormula
}
\hypertarget{ecldoc:toc:ML_Core.Math.StirlingFormula}{}
\hyperlink{ecldoc:toc:root/ML_Core/Math}{Go Up}

\section*{\underline{\textsf{IMPORTS}}}
\begin{doublespace}
{\large
ML\_Core.Math |
ML\_Core.Constants |
}
\end{doublespace}

\section*{\underline{\textsf{DESCRIPTIONS}}}
\subsection*{\textsf{\colorbox{headtoc}{\color{white} FUNCTION}
StirlingFormula}}

\hypertarget{ecldoc:ml_core.math.stirlingformula}{}

{\renewcommand{\arraystretch}{1.5}
\begin{tabularx}{\textwidth}{|>{\raggedright\arraybackslash}l|X|}
\hline
\hspace{0pt}\mytexttt{\color{red} } & \textbf{StirlingFormula} \\
\hline
\multicolumn{2}{|>{\raggedright\arraybackslash}X|}{\hspace{0pt}\mytexttt{\color{param} (REAL x)}} \\
\hline
\end{tabularx}
}

\par





Stirling's formula






\par
\begin{description}
\item [\colorbox{tagtype}{\color{white} \textbf{\textsf{PARAMETER}}}] \textbf{\underline{x}} ||| REAL8 --- the point of evaluation
\end{description}







\par
\begin{description}
\item [\colorbox{tagtype}{\color{white} \textbf{\textsf{RETURN}}}] \textbf{REAL8} --- evaluation result
\end{description}




\rule{\linewidth}{0.5pt}

\chapter*{\color{headfile}
{\large ML\_Core\slash\hspace{0pt}}
{\large Math\slash\hspace{0pt}}
 \\
upperGamma
}
\hypertarget{ecldoc:toc:ML_Core.Math.upperGamma}{}
\hyperlink{ecldoc:toc:root/ML_Core/Math}{Go Up}


\section*{\underline{\textsf{DESCRIPTIONS}}}
\subsection*{\textsf{\colorbox{headtoc}{\color{white} EMBED}
upperGamma}}

\hypertarget{ecldoc:ml_core.math.uppergamma}{}

{\renewcommand{\arraystretch}{1.5}
\begin{tabularx}{\textwidth}{|>{\raggedright\arraybackslash}l|X|}
\hline
\hspace{0pt}\mytexttt{\color{red} REAL8} & \textbf{upperGamma} \\
\hline
\multicolumn{2}{|>{\raggedright\arraybackslash}X|}{\hspace{0pt}\mytexttt{\color{param} (REAL8 x, REAL8 y)}} \\
\hline
\end{tabularx}
}

\par





Return the upper incomplete gamma value of two real numbers, x and y.






\par
\begin{description}
\item [\colorbox{tagtype}{\color{white} \textbf{\textsf{PARAMETER}}}] \textbf{\underline{y}} ||| REAL8 --- the value of the second number
\item [\colorbox{tagtype}{\color{white} \textbf{\textsf{PARAMETER}}}] \textbf{\underline{x}} ||| REAL8 --- the value of the first number
\end{description}







\par
\begin{description}
\item [\colorbox{tagtype}{\color{white} \textbf{\textsf{RETURN}}}] \textbf{REAL8} --- the upper incomplete gamma value
\end{description}




\rule{\linewidth}{0.5pt}


\chapter*{\color{headtoc} Math}
\hypertarget{ecldoc:toc:root/ML_Core/Math}{}
\hyperlink{ecldoc:toc:root/ML_Core}{Go Up}


\section*{Table of Contents}
{\renewcommand{\arraystretch}{1.5}
\begin{longtable}{|p{\textwidth}|}
\hline
\hyperlink{ecldoc:toc:ML_Core.Math.Beta}{Beta.ecl} \\
Return the beta value of two positive real numbers, x and y \\
\hline
\hyperlink{ecldoc:toc:ML_Core.Math.Distributions}{Distributions.ecl} \\
\hline
\hyperlink{ecldoc:toc:ML_Core.Math.DoubleFac}{DoubleFac.ecl} \\
The 'double' factorial is defined for ODD n and is the product of all the odd numbers up to and including that number \\
\hline
\hyperlink{ecldoc:toc:ML_Core.Math.Fac}{Fac.ecl} \\
Factorial function \\
\hline
\hyperlink{ecldoc:toc:ML_Core.Math.gamma}{gamma.ecl} \\
Return the value of gamma function of real number x A wrapper for the standard C tgamma function \\
\hline
\hyperlink{ecldoc:toc:ML_Core.Math.log_gamma}{log\_gamma.ecl} \\
Return the value of the log gamma function of the absolute value of X \\
\hline
\hyperlink{ecldoc:toc:ML_Core.Math.lowerGamma}{lowerGamma.ecl} \\
Return the lower incomplete gamma value of two real numbers, \\
\hline
\hyperlink{ecldoc:toc:ML_Core.Math.NCK}{NCK.ecl} \\
\hline
\hyperlink{ecldoc:toc:ML_Core.Math.Poly}{Poly.ecl} \\
Evaluate a polynomial from a set of co-effs \\
\hline
\hyperlink{ecldoc:toc:ML_Core.Math.StirlingFormula}{StirlingFormula.ecl} \\
Stirling's formula \\
\hline
\hyperlink{ecldoc:toc:ML_Core.Math.upperGamma}{upperGamma.ecl} \\
Return the upper incomplete gamma value of two real numbers, x and y \\
\hline
\end{longtable}
}

\chapter*{\color{headfile}
{\large ML\_Core\slash\hspace{0pt}}
{\large Math\slash\hspace{0pt}}
 \\
Beta
}
\hypertarget{ecldoc:toc:ML_Core.Math.Beta}{}
\hyperlink{ecldoc:toc:root/ML_Core/Math}{Go Up}

\section*{\underline{\textsf{IMPORTS}}}
\begin{doublespace}
{\large
ML\_Core.Math |
}
\end{doublespace}

\section*{\underline{\textsf{DESCRIPTIONS}}}
\subsection*{\textsf{\colorbox{headtoc}{\color{white} FUNCTION}
Beta}}

\hypertarget{ecldoc:ml_core.math.beta}{}

{\renewcommand{\arraystretch}{1.5}
\begin{tabularx}{\textwidth}{|>{\raggedright\arraybackslash}l|X|}
\hline
\hspace{0pt}\mytexttt{\color{red} } & \textbf{Beta} \\
\hline
\multicolumn{2}{|>{\raggedright\arraybackslash}X|}{\hspace{0pt}\mytexttt{\color{param} (REAL8 x, REAL8 y)}} \\
\hline
\end{tabularx}
}

\par





Return the beta value of two positive real numbers, x and y






\par
\begin{description}
\item [\colorbox{tagtype}{\color{white} \textbf{\textsf{PARAMETER}}}] \textbf{\underline{y}} ||| REAL8 --- the value of the second number
\item [\colorbox{tagtype}{\color{white} \textbf{\textsf{PARAMETER}}}] \textbf{\underline{x}} ||| REAL8 --- the value of the first number
\end{description}







\par
\begin{description}
\item [\colorbox{tagtype}{\color{white} \textbf{\textsf{RETURN}}}] \textbf{REAL8} --- the beta value
\end{description}




\rule{\linewidth}{0.5pt}

\chapter*{\color{headfile}
{\large LogisticRegression\slash\hspace{0pt}}
 \\
Distributions
}
\hypertarget{ecldoc:toc:LogisticRegression.Distributions}{}
\hyperlink{ecldoc:toc:root/LogisticRegression}{Go Up}

\section*{\underline{\textsf{IMPORTS}}}
\begin{doublespace}
{\large
ML\_Core.Constants |
ML\_Core.Math |
}
\end{doublespace}

\section*{\underline{\textsf{DESCRIPTIONS}}}
\subsection*{\textsf{\colorbox{headtoc}{\color{white} MODULE}
Distributions}}

\hypertarget{ecldoc:LogisticRegression.Distributions}{}

{\renewcommand{\arraystretch}{1.5}
\begin{tabularx}{\textwidth}{|>{\raggedright\arraybackslash}l|X|}
\hline
\hspace{0pt}\mytexttt{\color{red} } & \textbf{Distributions} \\
\hline
\end{tabularx}
}

\par





No Documentation Found







\textbf{Children}
\begin{enumerate}
\item \hyperlink{ecldoc:logisticregression.distributions.normal_cdf}{Normal\_CDF}
: Cumulative Distribution of the standard normal distribution, the probability that a normal random variable will be smaller than x standard deviations above or below the mean
\item \hyperlink{ecldoc:logisticregression.distributions.normal_ppf}{Normal\_PPF}
: Normal Distribution Percentage Point Function
\item \hyperlink{ecldoc:logisticregression.distributions.t_cdf}{T\_CDF}
: Students t distribution integral evaluated between negative infinity and x
\item \hyperlink{ecldoc:logisticregression.distributions.t_ppf}{T\_PPF}
: Percentage point function for the T distribution
\item \hyperlink{ecldoc:logisticregression.distributions.chi2_cdf}{Chi2\_CDF}
: The cumulative distribution function for the Chi Square distribution
\item \hyperlink{ecldoc:logisticregression.distributions.chi2_ppf}{Chi2\_PPF}
: The Chi Squared PPF function
\end{enumerate}

\rule{\linewidth}{0.5pt}

\subsection*{\textsf{\colorbox{headtoc}{\color{white} FUNCTION}
Normal\_CDF}}

\hypertarget{ecldoc:logisticregression.distributions.normal_cdf}{}
\hspace{0pt} \hyperlink{ecldoc:LogisticRegression.Distributions}{Distributions} \textbackslash 

{\renewcommand{\arraystretch}{1.5}
\begin{tabularx}{\textwidth}{|>{\raggedright\arraybackslash}l|X|}
\hline
\hspace{0pt}\mytexttt{\color{red} REAL8} & \textbf{Normal\_CDF} \\
\hline
\multicolumn{2}{|>{\raggedright\arraybackslash}X|}{\hspace{0pt}\mytexttt{\color{param} (REAL8 x)}} \\
\hline
\end{tabularx}
}

\par





Cumulative Distribution of the standard normal distribution, the probability that a normal random variable will be smaller than x standard deviations above or below the mean. Taken from C/C++ Mathematical Algorithms for Scientists and Engineers, n. Shammas, McGraw-Hill, 1995






\par
\begin{description}
\item [\colorbox{tagtype}{\color{white} \textbf{\textsf{PARAMETER}}}] \textbf{\underline{x}} ||| REAL8 --- the number of standard deviations
\end{description}







\par
\begin{description}
\item [\colorbox{tagtype}{\color{white} \textbf{\textsf{RETURN}}}] \textbf{REAL8} --- 
\end{description}






\par
\begin{description}
\item [\colorbox{tagtype}{\color{white} \textbf{\textsf{RETURNS}}}] probability of exceeding x.
\end{description}




\rule{\linewidth}{0.5pt}
\subsection*{\textsf{\colorbox{headtoc}{\color{white} FUNCTION}
Normal\_PPF}}

\hypertarget{ecldoc:logisticregression.distributions.normal_ppf}{}
\hspace{0pt} \hyperlink{ecldoc:LogisticRegression.Distributions}{Distributions} \textbackslash 

{\renewcommand{\arraystretch}{1.5}
\begin{tabularx}{\textwidth}{|>{\raggedright\arraybackslash}l|X|}
\hline
\hspace{0pt}\mytexttt{\color{red} REAL8} & \textbf{Normal\_PPF} \\
\hline
\multicolumn{2}{|>{\raggedright\arraybackslash}X|}{\hspace{0pt}\mytexttt{\color{param} (REAL8 x)}} \\
\hline
\end{tabularx}
}

\par





Normal Distribution Percentage Point Function. Translated from C/C++ Mathematical Algorithms for Scientists and Engineers, N. Shammas, McGraw-Hill, 1995






\par
\begin{description}
\item [\colorbox{tagtype}{\color{white} \textbf{\textsf{PARAMETER}}}] \textbf{\underline{x}} ||| REAL8 --- probability
\end{description}







\par
\begin{description}
\item [\colorbox{tagtype}{\color{white} \textbf{\textsf{RETURN}}}] \textbf{REAL8} --- 
\end{description}






\par
\begin{description}
\item [\colorbox{tagtype}{\color{white} \textbf{\textsf{RETURNS}}}] number of standard deviations from the mean
\end{description}




\rule{\linewidth}{0.5pt}
\subsection*{\textsf{\colorbox{headtoc}{\color{white} FUNCTION}
T\_CDF}}

\hypertarget{ecldoc:logisticregression.distributions.t_cdf}{}
\hspace{0pt} \hyperlink{ecldoc:LogisticRegression.Distributions}{Distributions} \textbackslash 

{\renewcommand{\arraystretch}{1.5}
\begin{tabularx}{\textwidth}{|>{\raggedright\arraybackslash}l|X|}
\hline
\hspace{0pt}\mytexttt{\color{red} REAL8} & \textbf{T\_CDF} \\
\hline
\multicolumn{2}{|>{\raggedright\arraybackslash}X|}{\hspace{0pt}\mytexttt{\color{param} (REAL8 x, REAL8 df)}} \\
\hline
\end{tabularx}
}

\par





Students t distribution integral evaluated between negative infinity and x. Translated from NIST SEL DATAPAC Fortran TCDF.f source






\par
\begin{description}
\item [\colorbox{tagtype}{\color{white} \textbf{\textsf{PARAMETER}}}] \textbf{\underline{df}} ||| REAL8 --- degrees of freedom
\item [\colorbox{tagtype}{\color{white} \textbf{\textsf{PARAMETER}}}] \textbf{\underline{x}} ||| REAL8 --- value of the evaluation
\end{description}







\par
\begin{description}
\item [\colorbox{tagtype}{\color{white} \textbf{\textsf{RETURN}}}] \textbf{REAL8} --- 
\end{description}






\par
\begin{description}
\item [\colorbox{tagtype}{\color{white} \textbf{\textsf{RETURNS}}}] the probability that a value will be less than the specified value
\end{description}




\rule{\linewidth}{0.5pt}
\subsection*{\textsf{\colorbox{headtoc}{\color{white} FUNCTION}
T\_PPF}}

\hypertarget{ecldoc:logisticregression.distributions.t_ppf}{}
\hspace{0pt} \hyperlink{ecldoc:LogisticRegression.Distributions}{Distributions} \textbackslash 

{\renewcommand{\arraystretch}{1.5}
\begin{tabularx}{\textwidth}{|>{\raggedright\arraybackslash}l|X|}
\hline
\hspace{0pt}\mytexttt{\color{red} REAL8} & \textbf{T\_PPF} \\
\hline
\multicolumn{2}{|>{\raggedright\arraybackslash}X|}{\hspace{0pt}\mytexttt{\color{param} (REAL8 x, REAL8 df)}} \\
\hline
\end{tabularx}
}

\par





Percentage point function for the T distribution. Translated from NIST SEL DATAPAC Fortran TPPF.f source






\par
\begin{description}
\item [\colorbox{tagtype}{\color{white} \textbf{\textsf{PARAMETER}}}] \textbf{\underline{df}} ||| REAL8 --- No Doc
\item [\colorbox{tagtype}{\color{white} \textbf{\textsf{PARAMETER}}}] \textbf{\underline{x}} ||| REAL8 --- No Doc
\end{description}







\par
\begin{description}
\item [\colorbox{tagtype}{\color{white} \textbf{\textsf{RETURN}}}] \textbf{REAL8} --- 
\end{description}




\rule{\linewidth}{0.5pt}
\subsection*{\textsf{\colorbox{headtoc}{\color{white} FUNCTION}
Chi2\_CDF}}

\hypertarget{ecldoc:logisticregression.distributions.chi2_cdf}{}
\hspace{0pt} \hyperlink{ecldoc:LogisticRegression.Distributions}{Distributions} \textbackslash 

{\renewcommand{\arraystretch}{1.5}
\begin{tabularx}{\textwidth}{|>{\raggedright\arraybackslash}l|X|}
\hline
\hspace{0pt}\mytexttt{\color{red} REAL8} & \textbf{Chi2\_CDF} \\
\hline
\multicolumn{2}{|>{\raggedright\arraybackslash}X|}{\hspace{0pt}\mytexttt{\color{param} (REAL8 x, REAL8 df)}} \\
\hline
\end{tabularx}
}

\par





The cumulative distribution function for the Chi Square distribution. the CDF for the specfied degrees of freedom. Translated from the NIST SEL DATAPAC Fortran subroutine CHSCDF.






\par
\begin{description}
\item [\colorbox{tagtype}{\color{white} \textbf{\textsf{PARAMETER}}}] \textbf{\underline{df}} ||| REAL8 --- No Doc
\item [\colorbox{tagtype}{\color{white} \textbf{\textsf{PARAMETER}}}] \textbf{\underline{x}} ||| REAL8 --- No Doc
\end{description}







\par
\begin{description}
\item [\colorbox{tagtype}{\color{white} \textbf{\textsf{RETURN}}}] \textbf{REAL8} --- 
\end{description}




\rule{\linewidth}{0.5pt}
\subsection*{\textsf{\colorbox{headtoc}{\color{white} FUNCTION}
Chi2\_PPF}}

\hypertarget{ecldoc:logisticregression.distributions.chi2_ppf}{}
\hspace{0pt} \hyperlink{ecldoc:LogisticRegression.Distributions}{Distributions} \textbackslash 

{\renewcommand{\arraystretch}{1.5}
\begin{tabularx}{\textwidth}{|>{\raggedright\arraybackslash}l|X|}
\hline
\hspace{0pt}\mytexttt{\color{red} REAL8} & \textbf{Chi2\_PPF} \\
\hline
\multicolumn{2}{|>{\raggedright\arraybackslash}X|}{\hspace{0pt}\mytexttt{\color{param} (REAL8 x, REAL8 df)}} \\
\hline
\end{tabularx}
}

\par





The Chi Squared PPF function. Translated from the NIST SEL DATAPAC Fortran subroutine CHSPPF.






\par
\begin{description}
\item [\colorbox{tagtype}{\color{white} \textbf{\textsf{PARAMETER}}}] \textbf{\underline{df}} ||| REAL8 --- No Doc
\item [\colorbox{tagtype}{\color{white} \textbf{\textsf{PARAMETER}}}] \textbf{\underline{x}} ||| REAL8 --- No Doc
\end{description}







\par
\begin{description}
\item [\colorbox{tagtype}{\color{white} \textbf{\textsf{RETURN}}}] \textbf{REAL8} --- 
\end{description}




\rule{\linewidth}{0.5pt}



\chapter*{\color{headfile}
{\large ML\_Core\slash\hspace{0pt}}
{\large Math\slash\hspace{0pt}}
 \\
DoubleFac
}
\hypertarget{ecldoc:toc:ML_Core.Math.DoubleFac}{}
\hyperlink{ecldoc:toc:root/ML_Core/Math}{Go Up}


\section*{\underline{\textsf{DESCRIPTIONS}}}
\subsection*{\textsf{\colorbox{headtoc}{\color{white} EMBED}
DoubleFac}}

\hypertarget{ecldoc:ml_core.math.doublefac}{}

{\renewcommand{\arraystretch}{1.5}
\begin{tabularx}{\textwidth}{|>{\raggedright\arraybackslash}l|X|}
\hline
\hspace{0pt}\mytexttt{\color{red} REAL8} & \textbf{DoubleFac} \\
\hline
\multicolumn{2}{|>{\raggedright\arraybackslash}X|}{\hspace{0pt}\mytexttt{\color{param} (INTEGER2 i)}} \\
\hline
\end{tabularx}
}

\par





The 'double' factorial is defined for ODD n and is the product of all the odd numbers up to and including that number. We are extending the meaning to even numbers to mean the product of the even numbers up to and including that number. Thus DoubleFac(8) = 8*6*4*2 We also defend against i < 2 (returning 1.0)






\par
\begin{description}
\item [\colorbox{tagtype}{\color{white} \textbf{\textsf{PARAMETER}}}] \textbf{\underline{i}} ||| INTEGER2 --- the value used in the calculation
\end{description}







\par
\begin{description}
\item [\colorbox{tagtype}{\color{white} \textbf{\textsf{RETURN}}}] \textbf{REAL8} --- the factorial of the sequence, declining by 2
\end{description}




\rule{\linewidth}{0.5pt}

\chapter*{\color{headfile}
{\large ML\_Core\slash\hspace{0pt}}
{\large Math\slash\hspace{0pt}}
 \\
Fac
}
\hypertarget{ecldoc:toc:ML_Core.Math.Fac}{}
\hyperlink{ecldoc:toc:root/ML_Core/Math}{Go Up}


\section*{\underline{\textsf{DESCRIPTIONS}}}
\subsection*{\textsf{\colorbox{headtoc}{\color{white} EMBED}
Fac}}

\hypertarget{ecldoc:ml_core.math.fac}{}

{\renewcommand{\arraystretch}{1.5}
\begin{tabularx}{\textwidth}{|>{\raggedright\arraybackslash}l|X|}
\hline
\hspace{0pt}\mytexttt{\color{red} REAL8} & \textbf{Fac} \\
\hline
\multicolumn{2}{|>{\raggedright\arraybackslash}X|}{\hspace{0pt}\mytexttt{\color{param} (UNSIGNED2 i)}} \\
\hline
\end{tabularx}
}

\par





Factorial function






\par
\begin{description}
\item [\colorbox{tagtype}{\color{white} \textbf{\textsf{PARAMETER}}}] \textbf{\underline{i}} ||| UNSIGNED2 --- the value used, (i)(i-1)(i-2)\ldots(2)
\end{description}







\par
\begin{description}
\item [\colorbox{tagtype}{\color{white} \textbf{\textsf{RETURN}}}] \textbf{REAL8} --- the factorial i!
\end{description}




\rule{\linewidth}{0.5pt}

\chapter*{\color{headfile}
{\large ML\_Core\slash\hspace{0pt}}
{\large Math\slash\hspace{0pt}}
 \\
gamma
}
\hypertarget{ecldoc:toc:ML_Core.Math.gamma}{}
\hyperlink{ecldoc:toc:root/ML_Core/Math}{Go Up}


\section*{\underline{\textsf{DESCRIPTIONS}}}
\subsection*{\textsf{\colorbox{headtoc}{\color{white} EMBED}
gamma}}

\hypertarget{ecldoc:ml_core.math.gamma}{}

{\renewcommand{\arraystretch}{1.5}
\begin{tabularx}{\textwidth}{|>{\raggedright\arraybackslash}l|X|}
\hline
\hspace{0pt}\mytexttt{\color{red} REAL8} & \textbf{gamma} \\
\hline
\multicolumn{2}{|>{\raggedright\arraybackslash}X|}{\hspace{0pt}\mytexttt{\color{param} (REAL8 x)}} \\
\hline
\end{tabularx}
}

\par





Return the value of gamma function of real number x A wrapper for the standard C tgamma function.






\par
\begin{description}
\item [\colorbox{tagtype}{\color{white} \textbf{\textsf{PARAMETER}}}] \textbf{\underline{x}} ||| REAL8 --- the input x
\end{description}







\par
\begin{description}
\item [\colorbox{tagtype}{\color{white} \textbf{\textsf{RETURN}}}] \textbf{REAL8} --- the value of GAMMA evaluated at x
\end{description}




\rule{\linewidth}{0.5pt}

\chapter*{\color{headfile}
{\large ML\_Core\slash\hspace{0pt}}
{\large Math\slash\hspace{0pt}}
 \\
log_gamma
}
\hypertarget{ecldoc:toc:ML_Core.Math.log_gamma}{}
\hyperlink{ecldoc:toc:root/ML_Core/Math}{Go Up}


\section*{\underline{\textsf{DESCRIPTIONS}}}
\subsection*{\textsf{\colorbox{headtoc}{\color{white} EMBED}
log\_gamma}}

\hypertarget{ecldoc:ml_core.math.log_gamma}{}

{\renewcommand{\arraystretch}{1.5}
\begin{tabularx}{\textwidth}{|>{\raggedright\arraybackslash}l|X|}
\hline
\hspace{0pt}\mytexttt{\color{red} REAL8} & \textbf{log\_gamma} \\
\hline
\multicolumn{2}{|>{\raggedright\arraybackslash}X|}{\hspace{0pt}\mytexttt{\color{param} (REAL8 x)}} \\
\hline
\end{tabularx}
}

\par





Return the value of the log gamma function of the absolute value of X. A wrapper for the standard C lgamma function. Avoids the race condition found on some platforms by taking the absolute value of the of the input argument.






\par
\begin{description}
\item [\colorbox{tagtype}{\color{white} \textbf{\textsf{PARAMETER}}}] \textbf{\underline{x}} ||| REAL8 --- the input x
\end{description}







\par
\begin{description}
\item [\colorbox{tagtype}{\color{white} \textbf{\textsf{RETURN}}}] \textbf{REAL8} --- the value of the log of the GAMMA evaluated at ABS(x)
\end{description}




\rule{\linewidth}{0.5pt}

\chapter*{\color{headfile}
{\large ML\_Core\slash\hspace{0pt}}
{\large Math\slash\hspace{0pt}}
 \\
lowerGamma
}
\hypertarget{ecldoc:toc:ML_Core.Math.lowerGamma}{}
\hyperlink{ecldoc:toc:root/ML_Core/Math}{Go Up}


\section*{\underline{\textsf{DESCRIPTIONS}}}
\subsection*{\textsf{\colorbox{headtoc}{\color{white} EMBED}
lowerGamma}}

\hypertarget{ecldoc:ml_core.math.lowergamma}{}

{\renewcommand{\arraystretch}{1.5}
\begin{tabularx}{\textwidth}{|>{\raggedright\arraybackslash}l|X|}
\hline
\hspace{0pt}\mytexttt{\color{red} REAL8} & \textbf{lowerGamma} \\
\hline
\multicolumn{2}{|>{\raggedright\arraybackslash}X|}{\hspace{0pt}\mytexttt{\color{param} (REAL8 x, REAL8 y)}} \\
\hline
\end{tabularx}
}

\par





Return the lower incomplete gamma value of two real numbers, x and y






\par
\begin{description}
\item [\colorbox{tagtype}{\color{white} \textbf{\textsf{PARAMETER}}}] \textbf{\underline{y}} ||| REAL8 --- the value of the second number
\item [\colorbox{tagtype}{\color{white} \textbf{\textsf{PARAMETER}}}] \textbf{\underline{x}} ||| REAL8 --- the value of the first number
\end{description}







\par
\begin{description}
\item [\colorbox{tagtype}{\color{white} \textbf{\textsf{RETURN}}}] \textbf{REAL8} --- the lower incomplete gamma value
\end{description}




\rule{\linewidth}{0.5pt}

\chapter*{\color{headfile}
{\large ML\_Core\slash\hspace{0pt}}
{\large Math\slash\hspace{0pt}}
 \\
NCK
}
\hypertarget{ecldoc:toc:ML_Core.Math.NCK}{}
\hyperlink{ecldoc:toc:root/ML_Core/Math}{Go Up}

\section*{\underline{\textsf{IMPORTS}}}
\begin{doublespace}
{\large
ML\_Core.Math |
}
\end{doublespace}

\section*{\underline{\textsf{DESCRIPTIONS}}}
\subsection*{\textsf{\colorbox{headtoc}{\color{white} FUNCTION}
NCK}}

\hypertarget{ecldoc:ml_core.math.nck}{}

{\renewcommand{\arraystretch}{1.5}
\begin{tabularx}{\textwidth}{|>{\raggedright\arraybackslash}l|X|}
\hline
\hspace{0pt}\mytexttt{\color{red} REAL8} & \textbf{NCK} \\
\hline
\multicolumn{2}{|>{\raggedright\arraybackslash}X|}{\hspace{0pt}\mytexttt{\color{param} (INTEGER2 N, INTEGER2 K)}} \\
\hline
\end{tabularx}
}

\par





No Documentation Found






\par
\begin{description}
\item [\colorbox{tagtype}{\color{white} \textbf{\textsf{PARAMETER}}}] \textbf{\underline{k}} ||| INTEGER2 --- No Doc
\item [\colorbox{tagtype}{\color{white} \textbf{\textsf{PARAMETER}}}] \textbf{\underline{n}} ||| INTEGER2 --- No Doc
\end{description}







\par
\begin{description}
\item [\colorbox{tagtype}{\color{white} \textbf{\textsf{RETURN}}}] \textbf{REAL8} --- 
\end{description}




\rule{\linewidth}{0.5pt}

\chapter*{\color{headfile}
{\large ML\_Core\slash\hspace{0pt}}
{\large Math\slash\hspace{0pt}}
 \\
Poly
}
\hypertarget{ecldoc:toc:ML_Core.Math.Poly}{}
\hyperlink{ecldoc:toc:root/ML_Core/Math}{Go Up}


\section*{\underline{\textsf{DESCRIPTIONS}}}
\subsection*{\textsf{\colorbox{headtoc}{\color{white} EMBED}
Poly}}

\hypertarget{ecldoc:ml_core.math.poly}{}

{\renewcommand{\arraystretch}{1.5}
\begin{tabularx}{\textwidth}{|>{\raggedright\arraybackslash}l|X|}
\hline
\hspace{0pt}\mytexttt{\color{red} REAL8} & \textbf{Poly} \\
\hline
\multicolumn{2}{|>{\raggedright\arraybackslash}X|}{\hspace{0pt}\mytexttt{\color{param} (REAL8 x, SET OF REAL8 Coeffs)}} \\
\hline
\end{tabularx}
}

\par





Evaluate a polynomial from a set of co-effs. Co-effs 1 is assumed to be the HIGH order of the equation. Thus for ax\^{}2+bx+c - the set would need to be Coef := [a,b,c];






\par
\begin{description}
\item [\colorbox{tagtype}{\color{white} \textbf{\textsf{PARAMETER}}}] \textbf{\underline{Coeffs}} ||| SET ( REAL8 ) --- a set of coefficients forthe polynomial. The ALL set is considered to be all zero values
\item [\colorbox{tagtype}{\color{white} \textbf{\textsf{PARAMETER}}}] \textbf{\underline{x}} ||| REAL8 --- the value of x in the polynomial
\end{description}







\par
\begin{description}
\item [\colorbox{tagtype}{\color{white} \textbf{\textsf{RETURN}}}] \textbf{REAL8} --- value of the polynomial at x
\end{description}




\rule{\linewidth}{0.5pt}

\chapter*{\color{headfile}
{\large ML\_Core\slash\hspace{0pt}}
{\large Math\slash\hspace{0pt}}
 \\
StirlingFormula
}
\hypertarget{ecldoc:toc:ML_Core.Math.StirlingFormula}{}
\hyperlink{ecldoc:toc:root/ML_Core/Math}{Go Up}

\section*{\underline{\textsf{IMPORTS}}}
\begin{doublespace}
{\large
ML\_Core.Math |
ML\_Core.Constants |
}
\end{doublespace}

\section*{\underline{\textsf{DESCRIPTIONS}}}
\subsection*{\textsf{\colorbox{headtoc}{\color{white} FUNCTION}
StirlingFormula}}

\hypertarget{ecldoc:ml_core.math.stirlingformula}{}

{\renewcommand{\arraystretch}{1.5}
\begin{tabularx}{\textwidth}{|>{\raggedright\arraybackslash}l|X|}
\hline
\hspace{0pt}\mytexttt{\color{red} } & \textbf{StirlingFormula} \\
\hline
\multicolumn{2}{|>{\raggedright\arraybackslash}X|}{\hspace{0pt}\mytexttt{\color{param} (REAL x)}} \\
\hline
\end{tabularx}
}

\par





Stirling's formula






\par
\begin{description}
\item [\colorbox{tagtype}{\color{white} \textbf{\textsf{PARAMETER}}}] \textbf{\underline{x}} ||| REAL8 --- the point of evaluation
\end{description}







\par
\begin{description}
\item [\colorbox{tagtype}{\color{white} \textbf{\textsf{RETURN}}}] \textbf{REAL8} --- evaluation result
\end{description}




\rule{\linewidth}{0.5pt}

\chapter*{\color{headfile}
{\large ML\_Core\slash\hspace{0pt}}
{\large Math\slash\hspace{0pt}}
 \\
upperGamma
}
\hypertarget{ecldoc:toc:ML_Core.Math.upperGamma}{}
\hyperlink{ecldoc:toc:root/ML_Core/Math}{Go Up}


\section*{\underline{\textsf{DESCRIPTIONS}}}
\subsection*{\textsf{\colorbox{headtoc}{\color{white} EMBED}
upperGamma}}

\hypertarget{ecldoc:ml_core.math.uppergamma}{}

{\renewcommand{\arraystretch}{1.5}
\begin{tabularx}{\textwidth}{|>{\raggedright\arraybackslash}l|X|}
\hline
\hspace{0pt}\mytexttt{\color{red} REAL8} & \textbf{upperGamma} \\
\hline
\multicolumn{2}{|>{\raggedright\arraybackslash}X|}{\hspace{0pt}\mytexttt{\color{param} (REAL8 x, REAL8 y)}} \\
\hline
\end{tabularx}
}

\par





Return the upper incomplete gamma value of two real numbers, x and y.






\par
\begin{description}
\item [\colorbox{tagtype}{\color{white} \textbf{\textsf{PARAMETER}}}] \textbf{\underline{y}} ||| REAL8 --- the value of the second number
\item [\colorbox{tagtype}{\color{white} \textbf{\textsf{PARAMETER}}}] \textbf{\underline{x}} ||| REAL8 --- the value of the first number
\end{description}







\par
\begin{description}
\item [\colorbox{tagtype}{\color{white} \textbf{\textsf{RETURN}}}] \textbf{REAL8} --- the upper incomplete gamma value
\end{description}




\rule{\linewidth}{0.5pt}


